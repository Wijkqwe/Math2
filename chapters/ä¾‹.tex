
\chapter{例}

\section{函数}

\subsubsection{间断连续}
设\(f(x)\)与\(g(x)\)在\((-\infty. +\infty)\)内有定义,\(x = x_1\)为f(x)唯一间断点,\(x = x_2\)为g(x)唯一间断点,则:
\subparagraph{解}
当\(x_1 \neq x_2\)时,\(f(x) + g(x)\)必有两个间断点。

令\(w(x) = f(x) + g(x)\),设w(x)在\(x_1\)处连续,\(f(x) = w(x) - g(x)\),由题知\(g(x)\)仅在\(x_2\)间断,则\(f(x)\)在\(x_1\)亦应连续,矛盾。


\subsubsection{间断判断}
设f(x)在\((a, b)\)内可导,\(x_0 \in (a, b)\)是\(f'(x)\)的间断点,则该间断点一定是:非无穷型第二类间断点(振荡间断点)
\subparagraph{解}
\begin{itemize}
    \item 若\(\displaystyle\lim_{x \to x_0^{\pm}}f(x) = A_{\pm}\)均存在 \(\Rightarrow\) \(f_{\pm}'(x_0) = \displaystyle\lim_{x \to x_0^{\pm}}\dfrac{f(x) - f(x_0)}{x - x_0} = \displaystyle\lim_{x \to x_0^{\pm}}f'(x) = A_{\pm}\),则在\(x_0\)连续,矛盾。
    \item 若\(\displaystyle\lim_{x \to x_0^{\pm}}f(x) = \infty\),则在\(x_0\)处不存在,矛盾
    \item 故振荡间断点。
\end{itemize}


\section{极限}

\subsubsection{洛必达条件}
设f(x)在\(x = a\)处二阶导数存在,则\[I = \lim_{h \to 0}\dfrac{\dfrac{f(a + h) - f(a)}{h} - f'(a)}{h} = ?\]

\subparagraph{方法1}
\begin{flalign}
    I & = \lim_{h \to 0}\dfrac{f(a + h) - f(a) + hf'(a)}{h^2} \nonumber \tag{1} \\ 
    & = \lim_{h \to 0}\dfrac{f'(a + h) - f'(a)}{2h} \nonumber \tag{2} \\ 
    & = \dfrac{1}{2}f''(a) \nonumber \tag{3}
\end{flalign}
(1)到(2)使用洛必达,(2)到(3)使用导数定义。\(\exists f''(a) \Rightarrow f'(x)\)在\(x = a\)某邻域有定义,故(1)可用洛必达;没有明确\(f''(x)\)在\(x = a\)某邻域内存在及\(f''(x)\)在\(x = a\)是连续的,故(2)到(3)不能用洛必达。

\subparagraph{方法2}
泰勒公式
\[f(a + h) = f(a) + f'(a)h + \dfrac{1}{2}f''(a)h^2 + o(h^2)\]
\begin{flalign}
    I & = \lim_{h \to 0}\dfrac{\dfrac{f'(a)h + \dfrac{1}{2}f''(a)h^2 + o(h^2)}{h} - f'(a)}{h} \nonumber \\ 
    & = \lim_{h \to 0}(\dfrac{1}{2}f''(a) + \dfrac{o(h^2)}{h^2}) = \dfrac{1}{2}f''(a) \nonumber
\end{flalign}


\subsubsection{夹逼准则}
若\(x >= 0\),则\(\displaystyle\lim_{n \to \infty}\sqrt[n]{1 + x^n + (\dfrac{x^2}{2})^n} = \)?
\subparagraph{解}
\begin{itemize}
    \item 当\(0 <= x < 1\)时,\(\max\{1, x^n, (\dfrac{x^2}{2})^n\} = 1\)
    \item 当\(1 <= x < 2\)时,\(\max\{1, x^n, (\dfrac{x^2}{2})^n\} = x^n\)
    \item 当\(x >= 2\)时,\(\max\{1, x^n, (\dfrac{x^2}{2})^n\} = (\dfrac{x^2}{2})^n\)
\end{itemize}


\section{导数}

\subsubsection{极值二阶导}
设\(f(x)\)在\(x = x_0\)处二阶可导,且\(f'(x_0) = 0, f''(x_0) \neq 0\)证明:
\begin{enumerate}
    \item \(f''(x_0) < 0\)时,f(x)在\(x_0\)处取极大值;
    \item \(f''(x_0) > 0\)时,f(x)在\(x_0\)处取极小值;
\end{enumerate}
\subparagraph{证明}
1):\(f''(x_0) = \displaystyle \lim_{x \to x_0}\dfrac{f'(x) - f'(x_0)}{x - x_0} < 0\),根据函数极限的局部保号性,存在\(x_0\)的去心邻域\(U(x_0, \delta)\),当\(x \in U(x_0, \delta)\),有\(\dfrac{f'(x) - f'(x_0)}{x - x_0} < 0\),因为\(f'(x_0) = 0\),\(f'(x)\)与\(x - x_0\)符号相反,根据判别极值第一充分条件,\(f(x)\)在点\(x_0\)处极大值。


\subsubsection{导函数连续}
已知\(g(x)\)在\(x = 0\)处二阶可导,且\(g(0) = g'(0) = 0\),设\(f(x) = \begin{cases}
\dfrac{g(x)}{x}, x\neq 0, \\ 
0, x = 0,
\end{cases}\),证明:\(f(x)\)导函数在x = 0处连续
\subparagraph{证明}
\(f'(0) = \displaystyle \lim_{x \to 0}\dfrac{\dfrac{g(x)}{x} - 0}{x - 0} = \lim_{x \to 0}\dfrac{g(x)}{x^2} = \lim_{x \to 0}\dfrac{g'(x)}{2x} = \dfrac{1}{2}\lim_{x \to 0}\dfrac{g'(x) - g'(0)}{x - 0} = \dfrac{1}{2}g''(0)\)

当\(x \neq 0\)时,\(f'(x) = \dfrac{xg'(x) - g(x)}{x^2}\),则\(\displaystyle\lim_{x \to 0}f'(x) = \lim_{x \to 0}\dfrac{xg'(x) - g(x)}{x^2} = \lim_{x \to 0}\dfrac{g'(x)}{x} - \lim_{x \to 0}\dfrac{g(x)}{x^2} = g''(0) - \dfrac{1}{2}g''(0) = f'(0)\),因此连续。




