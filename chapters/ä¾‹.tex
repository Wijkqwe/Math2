
\chapter{例}

\section{函数}

\subsubsection{间断连续}
设\(f(x)\)与\(g(x)\)在\((-\infty. +\infty)\)内有定义,\(x = x_1\)为f(x)唯一间断点,\(x = x_2\)为g(x)唯一间断点,则:
\subparagraph{解}
当\(x_1 \neq x_2\)时,\(f(x) + g(x)\)必有两个间断点。

令\(w(x) = f(x) + g(x)\),设w(x)在\(x_1\)处连续,\(f(x) = w(x) - g(x)\),由题知\(g(x)\)仅在\(x_2\)间断,则\(f(x)\)在\(x_1\)亦应连续,矛盾。


\subsubsection{间断判断}
设f(x)在\((a, b)\)内可导,\(x_0 \in (a, b)\)是\(f'(x)\)的间断点,则该间断点一定是:非无穷型第二类间断点(振荡间断点)
\subparagraph{解}
\begin{itemize}
    \item 若\(\displaystyle\lim_{x \to x_0^{\pm}}f(x) = A_{\pm}\)均存在 \(\Rightarrow\) \(f_{\pm}'(x_0) = \displaystyle\lim_{x \to x_0^{\pm}}\dfrac{f(x) - f(x_0)}{x - x_0} = \displaystyle\lim_{x \to x_0^{\pm}}f'(x) = A_{\pm}\),则在\(x_0\)连续,矛盾。
    \item 若\(\displaystyle\lim_{x \to x_0^{\pm}}f(x) = \infty\),则在\(x_0\)处不存在,矛盾
    \item 故振荡间断点。
\end{itemize}


\section{极限}

\subsubsection{洛必达条件}
设f(x)在\(x = a\)处二阶导数存在,则\[I = \lim_{h \to 0}\dfrac{\dfrac{f(a + h) - f(a)}{h} - f'(a)}{h} = ?\]

\subparagraph{方法1}
\begin{flalign}
    I & = \lim_{h \to 0}\dfrac{f(a + h) - f(a) + hf'(a)}{h^2} \nonumber \tag{1} \\ 
    & = \lim_{h \to 0}\dfrac{f'(a + h) - f'(a)}{2h} \nonumber \tag{2} \\ 
    & = \dfrac{1}{2}f''(a) \nonumber \tag{3}
\end{flalign}
(1)到(2)使用洛必达,(2)到(3)使用导数定义。\(\exists f''(a) \Rightarrow f'(x)\)在\(x = a\)某邻域有定义,故(1)可用洛必达;没有明确\(f''(x)\)在\(x = a\)某邻域内存在及\(f''(x)\)在\(x = a\)是连续的,故(2)到(3)不能用洛必达。

\subparagraph{方法2}
泰勒公式
\[f(a + h) = f(a) + f'(a)h + \dfrac{1}{2}f''(a)h^2 + o(h^2)\]
\begin{flalign}
    I & = \lim_{h \to 0}\dfrac{\dfrac{f'(a)h + \dfrac{1}{2}f''(a)h^2 + o(h^2)}{h} - f'(a)}{h} \nonumber \\ 
    & = \lim_{h \to 0}(\dfrac{1}{2}f''(a) + \dfrac{o(h^2)}{h^2}) = \dfrac{1}{2}f''(a) \nonumber
\end{flalign}


\subsubsection{夹逼准则}
若\(x >= 0\),则\(\displaystyle\lim_{n \to \infty}\sqrt[n]{1 + x^n + (\dfrac{x^2}{2})^n} = \)?
\subparagraph{解}
\begin{itemize}
    \item 当\(0 <= x < 1\)时,\(\max\{1, x^n, (\dfrac{x^2}{2})^n\} = 1\)
    \item 当\(1 <= x < 2\)时,\(\max\{1, x^n, (\dfrac{x^2}{2})^n\} = x^n\)
    \item 当\(x >= 2\)时,\(\max\{1, x^n, (\dfrac{x^2}{2})^n\} = (\dfrac{x^2}{2})^n\)
\end{itemize}


\section{导数}

\subsubsection{极值二阶导}
设\(f(x)\)在\(x = x_0\)处二阶可导,且\(f'(x_0) = 0, f''(x_0) \neq 0\)证明:
\begin{enumerate}
    \item \(f''(x_0) < 0\)时,f(x)在\(x_0\)处取极大值;
    \item \(f''(x_0) > 0\)时,f(x)在\(x_0\)处取极小值;
\end{enumerate}
\subparagraph{证明}
1):\(f''(x_0) = \displaystyle \lim_{x \to x_0}\dfrac{f'(x) - f'(x_0)}{x - x_0} < 0\),根据函数极限的局部保号性,存在\(x_0\)的去心邻域\(U(x_0, \delta)\),当\(x \in U(x_0, \delta)\),有\(\dfrac{f'(x) - f'(x_0)}{x - x_0} < 0\),因为\(f'(x_0) = 0\),\(f'(x)\)与\(x - x_0\)符号相反,根据判别极值第一充分条件,\(f(x)\)在点\(x_0\)处极大值。


\subsubsection{导函数连续}
已知\(g(x)\)在\(x = 0\)处二阶可导,且\(g(0) = g'(0) = 0\),设\(f(x) = \begin{cases}
\dfrac{g(x)}{x}, x\neq 0, \\ 
0, x = 0,
\end{cases}\),证明:\(f(x)\)导函数在x = 0处连续
\subparagraph{证明}
\(f'(0) = \displaystyle \lim_{x \to 0}\dfrac{\dfrac{g(x)}{x} - 0}{x - 0} = \lim_{x \to 0}\dfrac{g(x)}{x^2} = \lim_{x \to 0}\dfrac{g'(x)}{2x} = \dfrac{1}{2}\lim_{x \to 0}\dfrac{g'(x) - g'(0)}{x - 0} = \dfrac{1}{2}g''(0)\)

当\(x \neq 0\)时,\(f'(x) = \dfrac{xg'(x) - g(x)}{x^2}\),则\(\displaystyle\lim_{x \to 0}f'(x) = \lim_{x \to 0}\dfrac{xg'(x) - g(x)}{x^2} = \lim_{x \to 0}\dfrac{g'(x)}{x} - \lim_{x \to 0}\dfrac{g(x)}{x^2} = g''(0) - \dfrac{1}{2}g''(0) = f'(0)\),因此连续。


\section{中值定理}

\subsubsection{罗尔定理}
设实数\(a_0, a_1, ..., a_n\)满足\(a_0 + \dfrac{a_1}{2} + ... + \dfrac{a_n}{n + 1} = 0\),证明:方程\(a_0 + a_1x + a_2x^2 + ... + a_nx^n = 0\)在 \((0, 1)\)内至少一根。

\subparagraph{解}
设\(F(x) = a_0x + \dfrac{a_1}{2}x^2 + \dfrac{a_2}{3}x^3 + ... + \dfrac{a_n}{n + 1}x^{n + 1}, 0 <= x <= 1\),显然\(F(0) = 0\),且\(F(1) = a_0 + ... + \dfrac{a_n}{n + 1} = 0\),由罗尔定理知,存在\(\xi \in (0, 1), F'(\xi) = 0\)


\subsubsection{拉格朗日中值}
已知\(f(x)\)在\([0, 1]\)上连续,\((0, 1)\)内可导,且\(f(0) = 0, f(1) = 1\),证明:存在\(\eta, \tau \in (0, 1), \eta \neq \tau\),使得\(f'(\eta)f'(\tau) = 1\)

\subparagraph{证明}
易知存在\(\xi \in (0, 1)\),使得\(f(\xi) = 1 - \xi\),用\(\xi\)将\([0, 1]\)划分为\([0, \xi], [\xi, 1]\)由拉格朗日中值定理有\[f(\xi) - f(0) = f'(\eta)(\xi - 0), \eta \in (0, \xi)\]
\[f(1) - f(\xi) = f'(\tau)(1 - \xi), \tau \in (\xi, 1)\]
则\(f'(\eta) = \dfrac{f(\xi) - f(0)}{\xi} = \dfrac{1 - \xi}{\xi}, f'(\tau) = \dfrac{f(1) - f(\xi)}{1 - \xi} = \dfrac{\xi}{1 - \xi}\),故\(f'(\eta)f'(\tau) = 1\)


\subsubsection{泰勒公式}
设函数\(f(x)\)在\([0, 1]\)上二阶可导,且\(f(0) = f(1) = 0, \displaystyle \min_{x \in [0, 1]}\{f(x)\} = -1\),证明:存在\(\xi \in (0, 1)\),使得\(f''(\xi) >= 8\)

\subparagraph{证明}
利用一阶泰勒公式,由于\(f(x)\)在[0, 1]上连续,因此\(\exists x_0 \in [0, 1]\)使得\[f(x_0) = \displaystyle\min_{x \in [0, 1]}\{f(x)\} = -1\]
由于\(f(0) = f(1) = 0 > f(x_0)\),因此\(x_0 \in (0, 1), f'(x_0) = 0\);由一阶泰勒公式得\[f(x) = f(x_0) + f'(x_0)(x - x_0) + \dfrac{1}{2!}f''(\eta)(x - x_0)^2 = -1 + \dfrac{1}{2}f''(\eta)(x - x_0)^2\]

于是当\(x = 0\)时,\(f(0) = -1 + \dfrac{1}{2}f''(\xi_1)x_0^2\),即\(f''(\xi_1) = \dfrac{2}{x_0^2}, \xi_1\)是对应\(x = 0\)是的\(\eta\)

当\(x = 1\)时,\(f(1) = -1 + \dfrac{1}{2}f''(\xi_2)(1 - x_0)^2\),即\(f''(\xi_2) = \dfrac{2}{(1 - x_0)^2}, \xi_2\)是对应\(x = 1\)是的\(\eta\)

记\(f''(\xi) = \max\{f''(\xi_1), f''(\xi_2)\}, \xi = \xi_1 or \xi_2\),于是\(\exists \xi \in (0, 1)\)使得\[f''(\xi) = 2\max\{\dfrac{1}{x_0^2}, \dfrac{1}{(1 - x_0)^2}\} >= 2 * \dfrac{1}{(\dfrac{1}{2})^2} = 8\]

若要证明\(\exists \xi \in (a, b)\)使得\(f''(\xi)\)大于或小于某个非零常数时,往往利用泰勒公式;


\section{积分}

\subsection{概念性质}

\subsubsection{判断原函数与定积分}
\begin{enumerate}
    \item \(f(x) = \begin{cases}
    2,\ x > 0 \\ 
    1,\ x = 0 \\ 
    -1,\ x < 0
    \end{cases}\),没有原函数,定积分存在;
    \item \(f(x) = \begin{cases}
        2x\sin\dfrac{1}{x^2} - \dfrac{2}{x}\cos\dfrac{1}{x^2},\ x \neq 0 \\ 
        0,\ x = 0
    \end{cases}\),
    \item \(f(x) = \begin{cases}
        \dfrac{1}{x},\ x \neq 0 \\ 
        0,\ x = 0
    \end{cases}\)
    \item \(f(x) = \begin{cases}
        2x\cos\dfrac{1}{x} + sin\dfrac{1}{x},\ x \neq 0 \\ 
        0,\ \ x = 0
    \end{cases}\)
\end{enumerate}

\paragraph{解}
\begin{enumerate}
    \item x = 0为跳跃间断点,任意包含x = 0的区间上不存在原函数;满足定积分存在定理,定积分存在
    \item x = 0,是振荡间断点,设\(F(x) = \begin{cases}
        x^2\sin\dfrac{1}{x^2},\ x \neq 0 \\ 
        0,\ x = 0
    \end{cases}\),则\(F'(x) = f(x), -\infty < x < +\infty\),故存在原函数;由于\(\infty\cos\infty\)为无界振荡,在任一包含x = 0的区间上定积分不存在;
    \item x = 0无穷间断点,在任一包含x = 0的区间上不存在原函数;定积分不存在
    \item 存在原函数\(F(x) = \begin{cases}
        x^2\cos\dfrac{1}{x},\ x \neq 0 \\ 
        0,\ \ x = 0
    \end{cases}\);有界且只有一个振荡间断点,定积分存在
\end{enumerate}

\subsubsection{定积分定义}
先提\(\dfrac{1}{n}\),再凑\(\dfrac{i}{n}\),由于\(\dfrac{i}{n} = 0 + \dfrac{1 - 0}{n}i\),\(\dfrac{i}{n}\)可读作0到1上的x,\(\dfrac{1}{n}\)可读作0到1上的dx
\begin{flalign}
\lim_{x \to \infty}(\dfrac{n + 1}{n^2 + 1} + ... + \dfrac{n + n}{n^2 + n^2}) & = \lim_{x \to \infty}\sum_{i = 1}^n\dfrac{n + i}{n^2 + i^2} \\ 
& = \lim_{x \to \infty}\sum_{i = 1}^n\dfrac{n^2 + ni}{n^2 + i^2} * \dfrac{1}{n} \\ 
& = \lim_{x \to \infty}\sum_{i = 1}^n\dfrac{1 + \dfrac{i}{n}}{1 + (\dfrac{i}{n})^2} * \dfrac{1}{n} \\ 
& = \int_0^1\dfrac{1 + x}{1 + x^2}dx
\end{flalign}

\subsubsection{函数有界}
设\(a > 0, f(x)\)在\([0, \infty)\)内连续有界,C为常数;证明\(y = e^{-ax}(\int_0^xf(t)e^{at}dt + C)\)有界。
\paragraph{证明}
设\(|f(x)| <= M\),当\(x >= 0\)时,有
\begin{flalign}
    |y(x)| & = |e^{-ax}(C + \int_0^xf(t)e^{at}dt)| <= |Ce^{-ax}| + e^{-ax}|\int_0^xf(t)e^{at}dt| \\ 
    & <= |C| + e^{-ax}\int_0^x|f(t)e^{at}|dt <= |C| + Me^{-ax}\int_0^xe^{at}dt \\ 
    & = |C| + \dfrac{M}{a}(1 - e^{-ax}) <= |C| + \dfrac{M}{a}
\end{flalign}


\subsubsection{敛散性}
设\(a > b > 0\),反常积分\(\displaystyle\int_0^{+\infty}\dfrac{1}{x^a + x^b}dx\)收敛,则?

\paragraph{解}
\(I = \displaystyle\int_0^1\dfrac{1}{x^a + x^b}dx + \int_1^{+\infty}\dfrac{1}{x^a + x^b}dx = I_1 + I_2\),

对\(I_1\)看\(x \to 0^+\),由于\(a > b > 0\),\(x^b\)趋于0的“速度”慢于\(x^a\)趋于0的“速度”,\(x^a + x^b \sim x^b\),则\(b < 1\),

对\(I_2\)看\(x \to +\infty\),由于\(a > b > 0\),\(x^a\)趋于\(+\infty\)“速度”大于\(x^b\)趋于\(+\infty\)的“速度”,\(x^a + x^b \sim x^a\),\(a > 1\)


\subsubsection{敛散性,比较判别}
已知\(a > 0\),则对反常积分\(\displaystyle\int_0^1\dfrac{\ln x}{x^a}dx\)敛散性的判别:

\paragraph{解}
当\(a < 1\)时,取充分小正数\(\varepsilon\)使得\(a + \varepsilon < 1\),由于\[\lim_{x \to 0^+}\dfrac{\dfrac{\ln x}{x^a}}{\dfrac{1}{x^{a + \varepsilon}}} = \lim_{x \to 0^+}\dfrac{\ln x}{x^{-\varepsilon}} = \lim_{x \to 0^+}\dfrac{\dfrac{1}{x}}{-\varepsilon x^{-\varepsilon - 1}} = \lim_{x \to 0^+}(-\dfrac{1}{\varepsilon}x^\varepsilon) = 0\]
由于\(\int_0^1\dfrac{1}{x^{a + \varepsilon}}dx\)收敛,故\(\int_0^1\dfrac{\ln x}{x^a}dx\)收敛,

当\(a >= 1\)时,由于\(\lim_{x \to 0^+}x^a\dfrac{\ln x}{x^a} = \infty,\ \int_0^1\dfrac{1}{x^a}dx\)发散,故\(\int_0^1\dfrac{\ln x}{x^a}dx\)发散


\subsubsection{敛散性,比较判别}
已知\(a > 0\),则对反常积分\(\displaystyle\int_1^{+\infty}\dfrac{\ln x}{x^a}dx\)敛散性的判别:

\paragraph{解}
当\(a <= 1\)且x充分大时,\(\dfrac{\ln x}{x^a} > \dfrac{1}{x^a}\),由于\(\int_1^{+\infty}\dfrac{1}{x^a}dx\)发散,故\(\int_1^{+\infty}\dfrac{\ln x}{x^a}dx\)发散

当\(a > 1\)时,取充分小正数\(\varepsilon\)使得\(a - \varepsilon > 1\),由于\(\lim_{x \to +\infty}\dfrac{\dfrac{\ln x}{x^a}}{\dfrac{1}{x^{a - \varepsilon}}} = \lim_{x \to +\infty}\dfrac{\ln x}{x^\varepsilon} = 0,\ \int_1^{+\infty}\dfrac{1}{x^{a - \varepsilon}}dx\)收敛,故\(\int_1^{+\infty}\dfrac{\ln x}{x^a}dx\)收敛


\subsubsection{定积分定义放缩}
\(\displaystyle\lim_{n \to \infty}\sum_{i = 1}^n\dfrac{\sin\dfrac{i\pi}{n}}{n + \dfrac{1}{i}} = ?\)

\paragraph{解}
当各项分母均为n时,\(\displaystyle\lim_{n \to \infty}\sum_{i = 1}^n\dfrac{\sin\dfrac{i\pi}{n}}{n} = \int_0^1\sin\pi xdx\)。因此先进行放缩
\[\sum_{i = 1}^n\dfrac{\sin\dfrac{i\pi}{n}}{n + 1} <= \sum_{i = 1}^n\dfrac{\sin\dfrac{i\pi}{n}}{n + \dfrac{1}{i}} <= \sum_{i = 1}^n\dfrac{\sin\dfrac{i\pi}{n}}{n}\]
\[\because\ \lim_{n \to \infty}\sum_{i = 1}^n\dfrac{\sin\dfrac{i\pi}{n}}{n + 1} = \lim_{n \to \infty}\dfrac{n}{n + 1} * \dfrac{1}{n}\sum_{i = 1}^n\sin\dfrac{i}{n}\pi = \int_0^1\sin\pi x\,dx\]
\[\lim_{n \to \infty}\sum_{i = 1}^n\dfrac{\sin\dfrac{i\pi}{n}}{n} = \lim_{n \to \infty}\dfrac{1}{n}\sum_{i = 1}^n\sin\dfrac{i}{n}\pi = \int_0^1\sin\pi x\,dx\]
\[\therefore\ \lim_{n \to \infty}\sum_{i = 1}^n\dfrac{\sin\dfrac{i\pi}{n}}{n + \dfrac{1}{i}} = \dfrac{2}{\pi}\]


\subsubsection{反常积分敛散性判别}
讨论\(\displaystyle\int_2^{+\infty}\dfrac{1}{x\ln^px}dx\)敛散性,其中p为任意实数

\paragraph{解}
\begin{enumerate}
    \item 当p = 1时,\(\displaystyle\int_2^{+\infty}\dfrac{1}{x\ln x}dx = \ln|\ln x|\bigg|_2^{+\infty} = +\infty\),发散
    \item 当\(p \neq 1\)时,\[\int_2^{+\infty}\dfrac{dx}{x\ln^px} = \dfrac{1}{1 - p}(\ln x)^{1 - p}\bigg|_2^{+\infty}\]\begin{itemize}
        \item 当\(p > 1\)时,\(\displaystyle\lim_{x \to +\infty}(\ln x)^{1 - p} = 0\),收敛
        \item 当\(p < 1\)时,\(\displaystyle\lim_{x \to +\infty}(\ln x)^{1 - p} = +\infty\),发散
    \end{itemize}
\end{enumerate}
综上,\(\displaystyle\int_2^{+\infty}\dfrac{1}{x\ln^px}dx\begin{cases}
    \text{收敛},\ p > 1 \\ 
    \text{发散},\ p <= 1
\end{cases}\)


\subsection{计算}

\subsubsection{换元,分部积分}
求\(\displaystyle\int\dfrac{xe^x}{\sqrt{e^x - 1}}dx\)

\paragraph{解}
令\(u = \sqrt{e^x - 1},\ x = \ln(1 + u^2),\ dx = \dfrac{2u}{1 + u^2}du\),则
\begin{flalign}
    \int\dfrac{xe^x}{\sqrt{e^x - 1}}dx & = \int\dfrac{(1 + u^2)\ln(1 + u^2)}{u} * \dfrac{2u}{1 + u^2}du = 2\int\ln(1 + u^2)du \nonumber \\ 
    & = 2u\ln(1 + u^2) - \int\dfrac{4u^2}{1 + u^2}du
\end{flalign}


\subsubsection{换元}
求\(\displaystyle\int\dfrac{xe^{\arctan x}}{(1 + x^2)^{\frac{3}{2}}}dx\)

\paragraph{解}
令\(x = \tan t\),则\[\int\dfrac{xe^{\arctan x}}{(1 + x^2)^{\frac{3}{2}}}dx = \int\dfrac{e^t\tan t}{(1 + \tan^2t)^\frac{3}{2}}\sec^2tdt = \int e^t\sin tdt\]


\subsubsection{}
\(\displaystyle\int\dfrac{1}{1 + e^x}dx\)

\paragraph{解}
\[\int\dfrac{1}{1 + e^x}dx = \int(1 - \dfrac{e^x}{1 + e^x})dx = x - \ln(1 + e^x) + C\]


\subsubsection{分部积分}
求\(\displaystyle\int e^{2x}(\tan x + 1)^2dx\)

\paragraph{解}
\begin{flalign}
    \int e^{2x}(\tan x + 1)^2dx & = \int e^{2x}(\sec^2x + 2\tan x)dx \nonumber \\ 
    & = \int e^{2x}\sec^2xdx + 2\int e^{2x}\tan xdx \nonumber \\ 
    & = e^{2x}\tan x - 2\int e^{2x}\tan xdx + 2\int e^{2x}\tan xdx \nonumber \\ 
    & = e^{2x}\tan x + C \nonumber
\end{flalign}


\subsubsection{分式积分}
\(\displaystyle\int\dfrac{2x + 3}{x^2 - x + 1}dx\)

\paragraph{解}
\begin{flalign}
    \text{上式} & = \int\dfrac{2x - 1}{x^2 - x + 1}dx + \int\dfrac{4}{x^2 -x + 1}dx \nonumber \\ 
    & = \ln(x^2 - x + 1) + 4\int\dfrac{1}{(x - \frac{1}{2})^2 + (\frac{\sqrt{3}}{2})^2}d(x - \dfrac{1}{2}) \nonumber \\ 
    & = \ln(x^2 - x + 1) + (\dfrac{8}{\sqrt{3}}\arctan \dfrac{2x - 1}{\sqrt{3}}) + C
\end{flalign}


\subsubsection{换元为奇函数}
\(\displaystyle\lim_{n \to \infty}\dfrac{1}{n}\sum_{i = 1}^n[\ln(3n - 2i) - \ln(n + 2i)] = \)

\paragraph{解}
\begin{flalign}
    \text{上式} & = \int_0^1\ln\dfrac{3 - 2x}{1 + 2x}dx = \int_0^1\ln\dfrac{\frac{3}{2} - x}{\frac{1}{2} + x}dx \nonumber \\ 
    & = \int_{-\frac{1}{2}}^{\frac{1}{2}}\ln\dfrac{1 - t}{1 + t}dt = 0
\end{flalign}


\subsubsection{换元}
\(\displaystyle\int_0^1x\arcsin\sqrt{4x - 4x^2}dx\)

\paragraph{解}
\begin{flalign}
    \int_0^1x\arcsin\sqrt{4x - 4x^2}dx & = \int_0^1x\arcsin\sqrt{1 - (1 - 2x)^2}dx \nonumber \\ 
    & = \dfrac{1}{2}\int_1^{-1}(1 - t)\arcsin\sqrt{1 - t^2}(-\dfrac{1}{2}dt) = \dfrac{1}{4}\int_{-1}^1(1 - t)\arcsin\sqrt{1 - t^2}dt \nonumber \\ 
    & = \dfrac{1}{2}\int_0^1\arcsin\sqrt{1 - t^2}dt = \dfrac{1}{2}
\end{flalign}


\subsubsection{变限两个未知数}
\(F(x) = \displaystyle\int_0^{\frac{\pi}{2}}|\sin x - \sin t|dt,\ (x >= 0)\)在\(x \to 0^+\)处的二次泰勒多项式为\(a + bx + cx^2\),则\(abc = ?\)

\paragraph{解}
当\(x \to 0^+\)时,
\begin{flalign}
    F(x) & = \displaystyle\int_0^x(\sin x - \sin t)dt + \int_x^{\frac{\pi}{2}}(\sin t - \sin x)dt \nonumber \\ 
    & = x\sin x + (\cos x - 1) + \cos x - \sin x * (\dfrac{\pi}{2} - x) \nonumber \\ 
    & = (2x - \dfrac{\pi}{2})\sin x + 2\cos x - 1
\end{flalign}

\subparagraph{方式1}
直接展开
\[\sin x = x + o(x^2)\]
\[\cos x = 1 - \dfrac{1}{2}x^2 + o(x^2)\]
\[F(x) = 1 - \dfrac{\pi}{2}x + x^2 + o(x^2)\]
得\(abc = -\dfrac{\pi}{2}\)

\subparagraph{方式2}
\[F'(x) = (2x - \dfrac{\pi}{2})\cos x\]
\[F''(x) = 2\cos x - (2x - \dfrac{\pi}{2})\sin x\]
\[\therefore F(0) = 1, F'_+(0) = -\dfrac{\pi}{2}, F''_+(0) = 2\]
\[F(x) = F(0) + F'_+(0)x + \dfrac{F''_+(0)}{2!}x^2 + ...\]
得\(abc = -\dfrac{\pi}{2}\)


\subsubsection{无穷区间,换元}
\(\displaystyle\int_3^{+\infty}\dfrac{dx}{(x - 1)^4\sqrt{x^2 - 2x}}\)

\paragraph{解}
\begin{flalign}
    & = \int_3^{+\infty}\dfrac{dx}{(x - 1)^4\sqrt{(x - 1)^2 - 1}} \xrightarrow{x - 1 = \sec \theta}\int_\frac{\pi}{3}^\frac{\pi}{2}\dfrac{\sec\theta\tan\theta}{\sec^4\theta\tan\theta}d\theta \nonumber \\ 
    & = \int_\frac{\pi}{3}^\frac{\pi}{2}(1 - \sin^2\theta)\cos\theta d\theta = \dfrac{2}{3} - \dfrac{3\sqrt{3}}{8} \nonumber
\end{flalign}


\subsubsection{\(\Gamma\)函数}
设\(f(x) = \begin{cases}
    \dfrac{4x^2}{a^3\sqrt{\pi}}e^{-\frac{x^2}{a^2}}\ ,\ x > 0 \\ 
    0\ ,\ x <= 0
\end{cases}\),a为正常数,则\(\displaystyle\int_0^{+\infty}x^2f(x)dx =\)?

\paragraph{解}
\begin{flalign}
    \int_0^{+\infty}x^2f(x)dx & = \dfrac{2a^2}{\sqrt{\pi}} * 2\int_0^{+\infty}(\dfrac{x}{a})^{2 * \frac{5}{2} - 1}e^{-(\frac{x}{a})^2}\ d(\dfrac{x}{a}) \nonumber \\ 
    & = \dfrac{2a^2}{\sqrt{\pi}} * \Gamma(\dfrac{5}{2}) = \dfrac{3}{2}a^2 \nonumber
\end{flalign}


\subsubsection{分段函数}
\(\displaystyle\int\ \max\{1, |x|\}\ dx\)

\paragraph{解}
\(\max\{1, |x|\} = \begin{cases}
    -x,\ x < -1 \\ 
    1,\ -1 <= x <= 1 \\ 
    x,\ x > 1
\end{cases}\)
由于f(x)连续,则必存在原函数\(F(x) = \begin{cases}
    -\dfrac{x^2}{2} + C_1,\ x < -1 \\ 
    x + C_2,\ -1 <= x <= 1 \\ 
    \dfrac{x^2}{2} + C_3,\ x > 1
\end{cases}\)
又F(x)连续,则\(\begin{cases}
    -\dfrac{1}{2} + C_1 = -1 + C_2 \\ 
    1 + C_2 = \dfrac{1}{2} + C_3
\end{cases}\),得原式\(= \begin{cases}
    -\dfrac{x^2}{2} + C,\ x < -1 \\ 
    x + \dfrac{1}{2} + C,\ -1 <= x <= 1 \\ 
    \dfrac{x^2}{2} + 1 + C,\ x > 1
\end{cases}\)


\subsubsection{换元}
\(\displaystyle\int\arcsin\sqrt{\dfrac{x}{a + x}}dx\)

\paragraph{解}
令\(\arcsin\sqrt{\dfrac{x}{a + x}} = t,\ x = \dfrac{a\sin^2t}{1 - \sin^2t} = a\tan^2t\)
\begin{flalign}
    \int\arcsin\sqrt{\dfrac{x}{a + x}}dx = & \int td(a\tan^2t) = at\tan^2t - a\int\tan^2tdt \nonumber \\ 
    = & at\tan^2t + a\int(1 - \sec^2t)dt = at\tan^2t + at - a\tan t + C \nonumber \\ 
    = & (a + x)\arcsin\sqrt{\dfrac{x}{a + x}} - \sqrt{ax} + C \nonumber
\end{flalign}


\subsubsection{求满足条件函数,换元}
求连续函数\(f(x)\)使其满足\(\int_0^1f(tx)dt = f(x) + x\sin x\)

\paragraph{解}
令\(tx = u\),则原式化为\(\displaystyle\dfrac{1}{x}\int_0^xf(u)du = f(x) + x\sin x\),即\[\displaystyle\int_0^xf(u)du = xf(x) + x^2\sin x\]
两边对x求导得:\[f(x) = f(x) + xf'(x) + 2x\sin x + x^2\cos x\]
\[f'(x) = -2\sin x - x\cos x\]
积分得\[f(x) = \cos x - x\sin x + C\]


\subsubsection{换元,三角函数}
设\(f(x) = \begin{cases}
    \dfrac{1}{1 + \sin x},\ x >= 0 \\ 
    \dfrac{1}{1 + e^x},\ x < 0
\end{cases}\),求\(\displaystyle\int_{-1}^{\frac{\pi}{4}}f(x)dx\)

\paragraph{解}
\begin{flalign}
    \int_{-1}^0\dfrac{dx}{1 + e^x} & \xrightarrow{e^x = t} \int_{e^{-1}}^1\dfrac{1}{1 + t} * \dfrac{1}{t}dt = \int_{e^{-1}}^1(\dfrac{1}{t} - \dfrac{1}{1 + t})dt \nonumber \\ 
    & = \ln\dfrac{t}{1 + t}\bigg|_{e^{-1}}^1 = -\ln 2 + \ln(1 + e) \nonumber
\end{flalign}

\begin{flalign}
    \int_0^{\frac{\pi}{4}}\dfrac{dx}{1 + \sin x} & = \int_0^{\frac{\pi}{4}}\dfrac{1 - \sin x}{\cos^2x}dx = \int_0^{\frac{\pi}{4}}\sec^2xdx - \int_0^{\frac{\pi}{4}}\dfrac{\sin x}{\cos^2x}dx \nonumber \\ 
    & =\tan x\bigg|_0^{\frac{\pi}{4}} - \dfrac{1}{\cos x}\bigg|_0^{\frac{\pi}{4}} = 2 - \sqrt{2} \nonumber
\end{flalign}


\subsubsection{换元,奇偶性}
\(\displaystyle\int_{-1}^1\dfrac{x + 1}{1 + \sqrt[3]{x^2}}dx\)

\paragraph{解}
\begin{flalign}
    \int_{-1}^1\dfrac{x + 1}{1 + \sqrt[3]{x^2}}dx & = \int_{-1}^1\dfrac{x}{1 + \sqrt[3]{x^2}}dx + \int_{-1}^1\dfrac{1}{1 + \sqrt[3]{x^2}}dx = 0 + 2\int_0^1\dfrac{1}{1 + \sqrt[3]{x^2}}dx \nonumber \\ 
    & \xrightarrow{\sqrt[3]{x^2} = t} 3\int_0^1\dfrac{\sqrt{t}}{1 + t}dt \xrightarrow{\sqrt{t} = u} 6\int_0^1\dfrac{u^2}{1 + u^2}du \nonumber \\ 
    & = 6 - 6\arctan 1 = 6 - \dfrac{3}{2}\pi \nonumber
\end{flalign}


\subsubsection{三角函数}
\(\displaystyle\int_0^{\frac{3}{4}\pi}\dfrac{1}{1 + \cos^2x}dx\)

\paragraph{解}
\begin{flalign}
    \int_0^{\frac{3}{4}\pi}\dfrac{1}{1 + \cos^2x}dx & = \int_0^{\frac{\pi}{2}}\dfrac{1}{1 + \cos^2x}dx + \int_{\frac{\pi}{2}}^{\frac{3}{4}\pi}\dfrac{1}{1 + \cos^2x}dx \nonumber \\ 
    & = \lim_{x \to (\frac{\pi}{2})^-}F(x) - F(0) + F(\dfrac{3}{4}\pi) - \lim_{x \to (\frac{\pi}{2})^+}F(x) \nonumber \\ 
    & = \dfrac{\pi}{\sqrt{2}} - \dfrac{1}{\sqrt{2}}\arctan\dfrac{1}{\sqrt{2}} \nonumber
\end{flalign}
其中,\begin{flalign}
    F(x) &  = \int\dfrac{1}{1 + \cos^2x}dx = \int\dfrac{\sec^2x}{2 + \tan^2x}dx \nonumber \\ 
    & = \int\dfrac{\sqrt{2}d(\dfrac{\tan x}{\sqrt{2}})}{2[1 + (\dfrac{\tan x}{\sqrt{2}})^2]} = \dfrac{1}{\sqrt{2}}\arctan\dfrac{\tan x}{\sqrt{2}} + C \nonumber
\end{flalign}


\subsubsection{三角函数,换元}
\(\displaystyle\int_0^\pi\dfrac{x\sin x}{1 + \cos^2x}dx\)

\paragraph{解}
\begin{flalign}
    \int_0^\pi\dfrac{x\sin x}{1 + \cos^2x}dx & \xrightarrow{x = \pi - t} \int_\pi^0\dfrac{(\pi - t)\sin(\pi - t)}{1 + \cos^2(\pi - t)}(-dt) \nonumber \\ 
    & = \int_0^\pi\dfrac{(\pi - t)\sin t}{1 + \cos^2t}dt = \pi\int_0^\pi\dfrac{\sin t}{1 + \cos^2t}dt - \int_0^\pi\dfrac{t\sin t}{1 + \cos^2t}dt \nonumber \\ 
    & = \pi\int_0^\pi\dfrac{\sin t}{1 + \cos^2t}dt - \int_0^\pi\dfrac{x\sin x}{1 + \cos^2x}dx \nonumber \\ 
    \int_0^\pi\dfrac{x\sin x}{1 + \cos^2x}dx & = \dfrac{\pi}{2}\int_0^\pi\dfrac{\sin t}{1 + \cos^2t}dt = -\dfrac{\pi}{2}\int_0^\pi\dfrac{1}{1 + \cos^2t}d(\cos t) \nonumber \\ 
    & = -\dfrac{\pi}{2}\arctan(\cos t)\bigg|_0^\pi = \dfrac{\pi^2}{4} \nonumber
\end{flalign}


\subsubsection{三角函数}
\(\displaystyle\int_{-\dfrac{\pi}{4}}^{\dfrac{\pi}{4}}e^{\dfrac{x}{2}}\dfrac{\cos x - \sin x}{\sqrt{\cos x}}dx\)

\paragraph{解}
\begin{flalign}
    \int_{-\dfrac{\pi}{4}}^{\dfrac{\pi}{4}}e^{\dfrac{x}{2}}\dfrac{\cos x - \sin x}{\sqrt{\cos x}}dx & = \int_{-\dfrac{\pi}{4}}^{\dfrac{\pi}{4}}e^{\dfrac{x}{2}}\sqrt{\cos x}dx - \int_{-\dfrac{\pi}{4}}^{\dfrac{\pi}{4}}e^{\dfrac{x}{2}}\dfrac{\sin x}{\sqrt{\cos x}}dx \nonumber \\ 
    & = \int_{-\dfrac{\pi}{4}}^{\dfrac{\pi}{4}}e^{\dfrac{x}{2}}\sqrt{\cos x}dx + 2\int_{-\dfrac{\pi}{4}}^{\dfrac{\pi}{4}}e^{\dfrac{x}{2}}d(\sqrt{\cos x}) \nonumber \\ 
    & = \int_{-\dfrac{\pi}{4}}^{\dfrac{\pi}{4}}e^{\dfrac{x}{2}}\sqrt{\cos x}dx + 2e^{\dfrac{x}{2}}\sqrt{\cos x}\bigg|_{-\dfrac{\pi}{4}}^{\dfrac{\pi}{4}} - \int_{-\dfrac{\pi}{4}}^{\dfrac{\pi}{4}}e^{\dfrac{x}{2}}\sqrt{\cos x}dx \nonumber \\ 
    & = \sqrt[4]{8}(e^{\frac{\pi}{8}} - e^{-\frac{\pi}{8}}) \nonumber
\end{flalign}


\subsubsection{三角函数,换元}
\(\displaystyle\int_{\frac{1}{2}}^{\frac{3}{2}}\dfrac{(1 - x)\arcsin(1 - x)}{\sqrt{2x - x^2}}dx\)

\paragraph{解}
\begin{flalign}
    \int_{\frac{1}{2}}^{\frac{3}{2}}\dfrac{(1 - x)\arcsin(1 - x)}{\sqrt{2x - x^2}}dx & \xrightarrow{1 - x = \sin t} \int_{-\dfrac{\pi}{6}}^{\dfrac{\pi}{6}}\dfrac{t\sin t}{\cos t}\cos tdt = \int_{-\dfrac{\pi}{6}}^{\dfrac{\pi}{6}}t\sin tdt \nonumber \\ 
    & = -\int_{-\dfrac{\pi}{6}}^{\dfrac{\pi}{6}}td(\cos t) = -(t\cos t - \sin t)\bigg|_{-\dfrac{\pi}{6}}^{\dfrac{\pi}{6}} \nonumber \\ 
    & = 1 - \dfrac{\sqrt{3}\pi}{6} \nonumber
\end{flalign}


\subsubsection{定积分定义夹逼}
\(\displaystyle\lim_{n \to \infty}\sum_{i = 1}^n\dfrac{n}{n^2 + i^2 + 1} = \)

\subparagraph{解}
\[\sum_{i = 1}^n\dfrac{1}{n}\dfrac{1}{1 + \dfrac{(i + 1)^2}{n^2}} <= \sum_{i = 1}^n\dfrac{1}{n}\dfrac{1}{1 + \dfrac{i^2 + 1}{n^2}} <= \sum_{i = 1}^n\dfrac{1}{n}\dfrac{1}{1 + \dfrac{i^2}{n^2}}\]
\[\sum_{i = 1}^n\dfrac{1}{n}\dfrac{1}{1 + \dfrac{(1 + i)^2}{n^2}} = \sum_{i = 1}^n\dfrac{1}{n}\dfrac{1}{1 + \dfrac{i^2}{n^2}} - \dfrac{1}{n}\dfrac{1}{1 + \dfrac{1}{n^2}} + \dfrac{1}{n}\dfrac{1}{1 + \dfrac{(n + 1)^2}{n^2}}\]
\[\therefore\ \lim_{n \to \infty}\sum_{i = 1}^n\dfrac{n}{n^2 + i^2 + 1} = \dfrac{\pi}{4}\]


\subsubsection{反常函数积分}
\(\displaystyle\int_0^{+\infty}\dfrac{xe^{-x}}{(1 + e^{-x})^2}dx = \)

\subparagraph{解}
\begin{flalign}
    \text{原式} & = \lim_{b \to +\infty}\int_0^bxd(\dfrac{1}{1 + e^{-x}}) = \lim_{b \to +\infty}(\dfrac{x}{1 + e^{-x}}\bigg|_0^b - \int_0^b\dfrac{dx}{1 + e^{-x}}) \nonumber \\ 
    & = \lim_{b \to +\infty}(\dfrac{b}{1 + e^{-b}} - \ln(1 + e^{-x})\bigg|_0^b) = \ln2 + \lim_{b \to +\infty}(\dfrac{b}{1 + e^{-b}} - \ln(e^b + 1)) \nonumber \\ 
    & = \ln2 + \lim_{b \to +\infty}\dfrac{1}{1 + e^{-b}}(b - (1 + e^{-b})\ln(e^b + 1)) \nonumber \\ 
    & = \ln2 + \lim_{b \to +\infty}(\ln e^b - (1 + e^b)\dfrac{\ln(1 + e^b)}{e^b}) \nonumber \\ 
    & = \lim_{b \to +\infty}(\ln\dfrac{e^b}{e^b + 1} - \dfrac{\ln(1 + e^b)}{e^b}) = \ln2 + \ln1 - 0 \nonumber \\
    & = \ln2 \nonumber
\end{flalign}


\subsubsection{\(e^{t^2x}\)换元}
对\(\displaystyle\int e^{t^2x}dt\),令\(t\sqrt{x} = u, dt = \dfrac{1}{\sqrt{x}}du\),则\[\int e^{t^2x}dt = \dfrac{1}{\sqrt{x}}\int e^{u^2}du\]


\subsubsection{绝对值奇偶性}
\(\displaystyle\int_{-1}^1(x + 2|x|)^2dx\),展开得\(x^2 + 2x|x| + 4x^2\),对称性有\[ = 10\int_0^1x^2dx\]


\subsubsection{三角函数}
\(\displaystyle\int\dfrac{dx}{\cos x + \sin x}\)
\paragraph{解}
令\(\tan\dfrac{x}{2} = t\),则\(x = 2\arctan t,\ dx = \dfrac{2dt}{1 + t^2},\ \sin x = \dfrac{2t}{1 + t^2}, \cos x = \dfrac{1 - t^2}{1 + t^2}\),代入得
\[ = \int\dfrac{2dt}{1 + 2t - t^2} = \int\dfrac{2dt}{2 - (1 - t)^2}\]


\subsubsection{\(e^{tx - t^2}\)变限积分换元}
\(F(x) = \displaystyle\int_0^xe^{tx - t^2}dt\),求\(F'(x)\)
\subparagraph{解}
\begin{flalign}
    F(x) & = \int_0^xe^{\frac{x^2}{4} - (\frac{x}{2} - t)^2}dt = e^{\frac{x^2}{4}}\int_0^xe^{- (\frac{x}{2} - t)^2}d(t - \dfrac{x}{2}) \nonumber \\ 
    & \xrightarrow{u = \frac{x}{2} - t} -e^{\frac{x^2}{4}}\int_{\frac{x}{2}}^{-\frac{x}{2}}e^{-u^2}du = 2e^{\frac{x^2}{4}}\int_{0}^{\frac{x}{2}}e^{-u^2}du \nonumber
\end{flalign}


\subsubsection{三角函数,原函数}
\(\displaystyle f(x) = \dfrac{1}{1 + \sin^2x}, x\in[0, \pi]\),则\(f(x)\)在\([0, \pi]\)上全体原函数为
\subparagraph{解}
\begin{flalign}
    \int\dfrac{dx}{1 + \sin^2x} & = \int\dfrac{\dfrac{1}{\cos^2x}dx}{\dfrac{1}{\cos^2x} + \tan^2x} \nonumber \\ 
    & = \int\dfrac{d\tan x}{1 + 2\tan^2x} \nonumber \\ 
    & = \dfrac{1}{\sqrt{2}}\arctan(\sqrt{2}\tan x) + C \nonumber
\end{flalign}

\(\because\)在\(x = \dfrac{\pi}{2}\)上无定义,

\(\therefore\)左右分别求极限,得\(F(x) + C\),其中
\[F(x) = \begin{cases}
    \dfrac{1}{\sqrt{2}}\arctan(\sqrt{2}\tan x) - \dfrac{\pi}{2\sqrt{2}}, 0 <= x < \dfrac{\pi}{2} \\ 
    0,\ \ x = \dfrac{\pi}{2} \\ 
    \dfrac{1}{\sqrt{2}}\arctan(\sqrt{2}\tan x) + \dfrac{\pi}{2\sqrt{2}}, \dfrac{\pi}{2} < x <= \pi
\end{cases}\]


\section{二重积分}

\subsubsection{极坐标平移}
设D为圆域\(x^2 + y^2 <= 2x + 2y\),则\(\displaystyle\iint_Dxydxdy = \) ?
\subparagraph{解}
\((x - 1)^2 + (y - 1)^2 = 2\),令\(x = 1 + \rho\cos\theta, y = 1 + \rho\sin\theta\),则
\begin{flalign}
    \iint_Dxydxdy & = \int_0^{2\pi}d\theta\int_0^{\sqrt{2}}(1 + \rho\cos\theta)(1 + \rho\sin\theta)\rho d\rho \nonumber \\ 
    & = \int_0^{2\pi}d\theta\int_0^{\sqrt{2}}(1 + \rho\cos\theta + \rho\sin\theta + \rho^2\sin\theta\cos\theta)\rho d\rho \nonumber \\ 
    & = \int_0^{2\pi}d\theta\int_0^{\sqrt{2}}\rho d\rho = 2\pi \nonumber
\end{flalign}


\subsubsection{\(x^x\)}
设积分区域D由曲线\(y = \ln x\)及直线\(x = 2, y = 0\)围成,则\(\displaystyle\iint_D\dfrac{e^{xy}}{x^x - 1}\mathrm{d}\sigma = \)?
\subparagraph{解}
由题设知积分区域\(D=\{(x,y)\mid1\leqslant x\leqslant2,\ 0\leqslant y\leqslant\ln x\}\),从而

\(\begin{aligned}
    \iint_{D}\frac{e^{xy}}{x^{x} - 1}\mathrm{d}\sigma & = \int_{1}^{2}\mathrm{d}x\int_{0}^{\ln x}\frac{\mathrm{e}^{xy}}{x^{x} - 1}\mathrm{d}y \\
    & = \int_{1}^{2}\frac{\mathrm{d}x}{x^{x} - 1}\int_{0}^{\ln x}\mathrm{e}^{xy}\mathrm{d}y \\
    & = \int_{1}^{2}\frac{\mathrm{d}x}{x(x^{x} - 1)}\int_{0}^{\ln x}\mathrm{e}^{xy}\mathrm{d}\left(xy\right) \\
    & = \int_{1}^{2}\frac{\mathrm{e}^{xy}}{x(x^{x} - 1)}\bigg|_{y=0}^{y=\ln x}\mathrm{d}x \\ 
    & = \int_{1}^{2}\frac{e^{x\ln x} - 1}{x(x^{x} - 1)}dx=\int_{1}^{2}\frac{dx}{x}=\ln2
\end{aligned}\)



