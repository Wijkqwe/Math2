
\chapter{例}

\section{函数}

\subsubsection{间断连续}
设\(f(x)\)与\(g(x)\)在\((-\infty. +\infty)\)内有定义,\(x = x_1\)为f(x)唯一间断点,\(x = x_2\)为g(x)唯一间断点,则:
\subparagraph{解}
当\(x_1 \neq x_2\)时,\(f(x) + g(x)\)必有两个间断点。

令\(w(x) = f(x) + g(x)\),设w(x)在\(x_1\)处连续,\(f(x) = w(x) - g(x)\),由题知\(g(x)\)仅在\(x_2\)间断,则\(f(x)\)在\(x_1\)亦应连续,矛盾。


\subsubsection{间断判断}
设f(x)在\((a, b)\)内可导,\(x_0 \in (a, b)\)是\(f'(x)\)的间断点,则该间断点一定是:非无穷型第二类间断点(振荡间断点)
\subparagraph{解}
\begin{itemize}
    \item 若\(\displaystyle\lim_{x \to x_0^{\pm}}f(x) = A_{\pm}\)均存在 \(\Rightarrow\) \(f_{\pm}'(x_0) = \displaystyle\lim_{x \to x_0^{\pm}}\dfrac{f(x) - f(x_0)}{x - x_0} = \displaystyle\lim_{x \to x_0^{\pm}}f'(x) = A_{\pm}\),则在\(x_0\)连续,矛盾。
    \item 若\(\displaystyle\lim_{x \to x_0^{\pm}}f(x) = \infty\),则在\(x_0\)处不存在,矛盾
    \item 故振荡间断点。
\end{itemize}


\section{极限}

\subsubsection{洛必达条件}
设f(x)在\(x = a\)处二阶导数存在,则\[I = \lim_{h \to 0}\dfrac{\dfrac{f(a + h) - f(a)}{h} - f'(a)}{h} = ?\]

\subparagraph{方法1}
\begin{flalign}
    I & = \lim_{h \to 0}\dfrac{f(a + h) - f(a) + hf'(a)}{h^2} \nonumber \tag{1} \\ 
    & = \lim_{h \to 0}\dfrac{f'(a + h) - f'(a)}{2h} \nonumber \tag{2} \\ 
    & = \dfrac{1}{2}f''(a) \nonumber \tag{3}
\end{flalign}
(1)到(2)使用洛必达,(2)到(3)使用导数定义。\(\exists f''(a) \Rightarrow f'(x)\)在\(x = a\)某邻域有定义,故(1)可用洛必达;没有明确\(f''(x)\)在\(x = a\)某邻域内存在及\(f''(x)\)在\(x = a\)是连续的,故(2)到(3)不能用洛必达。

\subparagraph{方法2}
泰勒公式
\[f(a + h) = f(a) + f'(a)h + \dfrac{1}{2}f''(a)h^2 + o(h^2)\]
\begin{flalign}
    I & = \lim_{h \to 0}\dfrac{\dfrac{f'(a)h + \dfrac{1}{2}f''(a)h^2 + o(h^2)}{h} - f'(a)}{h} \nonumber \\ 
    & = \lim_{h \to 0}(\dfrac{1}{2}f''(a) + \dfrac{o(h^2)}{h^2}) = \dfrac{1}{2}f''(a) \nonumber
\end{flalign}


\subsubsection{夹逼准则}
若\(x >= 0\),则\(\displaystyle\lim_{n \to \infty}\sqrt[n]{1 + x^n + (\dfrac{x^2}{2})^n} = \)?
\subparagraph{解}
\begin{itemize}
    \item 当\(0 <= x < 1\)时,\(\max\{1, x^n, (\dfrac{x^2}{2})^n\} = 1\)
    \item 当\(1 <= x < 2\)时,\(\max\{1, x^n, (\dfrac{x^2}{2})^n\} = x^n\)
    \item 当\(x >= 2\)时,\(\max\{1, x^n, (\dfrac{x^2}{2})^n\} = (\dfrac{x^2}{2})^n\)
\end{itemize}


\section{数列}

\subsubsection{单调有界收敛}
设数列\(\{x_n\}\)满足:\(x_1 > 0, x_ne^{x_{n + 1}} = e^{x_{n}} - 1\,\,\,(n = 1, 2, ...)\),证明\(\{x_n\}\)收敛,并求\(\displaystyle\lim_{n \to \infty}x_n\)

\subparagraph{解}
先证有界。 \\
已知\(x_1 > 0\),设\(x_k > 0\),由\(x > 0\)时\(e^x - 1 > x\),得\(x_{k + 1} = \ln\dfrac{e^{x_{k}} - 1}{x_k} > \ln 1 > 0\),故数列有下界0. \\
而\(e^{x_{n + 1}} = \dfrac{e^{x_{n}} - 1}{x_n} = \dfrac{e^{x_{n}} - e^0}{x_n} = e^\xi\,\,(0 <= \xi <= x_{n})\),所以\(x_{n + 1} = \xi < x_n\),即数列递减,故收敛。 \\
设\(\displaystyle\lim_{n \to \infty}x_n = A\),得\(A = 0\)


\section{导数}

\subsubsection{极值二阶导}
设\(f(x)\)在\(x = x_0\)处二阶可导,且\(f'(x_0) = 0, f''(x_0) \neq 0\)证明:
\begin{enumerate}
    \item \(f''(x_0) < 0\)时,f(x)在\(x_0\)处取极大值;
    \item \(f''(x_0) > 0\)时,f(x)在\(x_0\)处取极小值;
\end{enumerate}
\subparagraph{证明}
1):\(f''(x_0) = \displaystyle \lim_{x \to x_0}\dfrac{f'(x) - f'(x_0)}{x - x_0} < 0\),根据函数极限的局部保号性,存在\(x_0\)的去心邻域\(U(x_0, \delta)\),当\(x \in U(x_0, \delta)\),有\(\dfrac{f'(x) - f'(x_0)}{x - x_0} < 0\),因为\(f'(x_0) = 0\),\(f'(x)\)与\(x - x_0\)符号相反,根据判别极值第一充分条件,\(f(x)\)在点\(x_0\)处极大值。


\subsubsection{导函数连续}
已知\(g(x)\)在\(x = 0\)处二阶可导,且\(g(0) = g'(0) = 0\),设\(f(x) = \begin{cases}
\dfrac{g(x)}{x}, x\neq 0, \\ 
0, x = 0,
\end{cases}\),证明:\(f(x)\)导函数在x = 0处连续
\subparagraph{证明}
\(f'(0) = \displaystyle \lim_{x \to 0}\dfrac{\dfrac{g(x)}{x} - 0}{x - 0} = \lim_{x \to 0}\dfrac{g(x)}{x^2} = \lim_{x \to 0}\dfrac{g'(x)}{2x} = \dfrac{1}{2}\lim_{x \to 0}\dfrac{g'(x) - g'(0)}{x - 0} = \dfrac{1}{2}g''(0)\)

当\(x \neq 0\)时,\(f'(x) = \dfrac{xg'(x) - g(x)}{x^2}\),则\(\displaystyle\lim_{x \to 0}f'(x) = \lim_{x \to 0}\dfrac{xg'(x) - g(x)}{x^2} = \lim_{x \to 0}\dfrac{g'(x)}{x} - \lim_{x \to 0}\dfrac{g(x)}{x^2} = g''(0) - \dfrac{1}{2}g''(0) = f'(0)\),因此连续。


\section{中值定理}

\subsubsection{罗尔定理}
设实数\(a_0, a_1, ..., a_n\)满足\(a_0 + \dfrac{a_1}{2} + ... + \dfrac{a_n}{n + 1} = 0\),证明:方程\(a_0 + a_1x + a_2x^2 + ... + a_nx^n = 0\)在 \((0, 1)\)内至少一根。

\subparagraph{解}
设\(F(x) = a_0x + \dfrac{a_1}{2}x^2 + \dfrac{a_2}{3}x^3 + ... + \dfrac{a_n}{n + 1}x^{n + 1}, 0 <= x <= 1\),显然\(F(0) = 0\),且\(F(1) = a_0 + ... + \dfrac{a_n}{n + 1} = 0\),由罗尔定理知,存在\(\xi \in (0, 1), F'(\xi) = 0\)


\subsubsection{拉格朗日中值}
已知\(f(x)\)在\([0, 1]\)上连续,\((0, 1)\)内可导,且\(f(0) = 0, f(1) = 1\),证明:存在\(\eta, \tau \in (0, 1), \eta \neq \tau\),使得\(f'(\eta)f'(\tau) = 1\)

\subparagraph{证明}
易知存在\(\xi \in (0, 1)\),使得\(f(\xi) = 1 - \xi\),用\(\xi\)将\([0, 1]\)划分为\([0, \xi], [\xi, 1]\)由拉格朗日中值定理有\[f(\xi) - f(0) = f'(\eta)(\xi - 0), \eta \in (0, \xi)\]
\[f(1) - f(\xi) = f'(\tau)(1 - \xi), \tau \in (\xi, 1)\]
则\(f'(\eta) = \dfrac{f(\xi) - f(0)}{\xi} = \dfrac{1 - \xi}{\xi}, f'(\tau) = \dfrac{f(1) - f(\xi)}{1 - \xi} = \dfrac{\xi}{1 - \xi}\),故\(f'(\eta)f'(\tau) = 1\)


\subsubsection{泰勒公式}
设函数\(f(x)\)在\([0, 1]\)上二阶可导,且\(f(0) = f(1) = 0, \displaystyle \min_{x \in [0, 1]}\{f(x)\} = -1\),证明:存在\(\xi \in (0, 1)\),使得\(f''(\xi) >= 8\)

\subparagraph{证明}
利用一阶泰勒公式,由于\(f(x)\)在[0, 1]上连续,因此\(\exists x_0 \in [0, 1]\)使得\[f(x_0) = \displaystyle\min_{x \in [0, 1]}\{f(x)\} = -1\]
由于\(f(0) = f(1) = 0 > f(x_0)\),因此\(x_0 \in (0, 1), f'(x_0) = 0\);由一阶泰勒公式得\[f(x) = f(x_0) + f'(x_0)(x - x_0) + \dfrac{1}{2!}f''(\eta)(x - x_0)^2 = -1 + \dfrac{1}{2}f''(\eta)(x - x_0)^2\]

于是当\(x = 0\)时,\(f(0) = -1 + \dfrac{1}{2}f''(\xi_1)x_0^2\),即\(f''(\xi_1) = \dfrac{2}{x_0^2}, \xi_1\)是对应\(x = 0\)是的\(\eta\)

当\(x = 1\)时,\(f(1) = -1 + \dfrac{1}{2}f''(\xi_2)(1 - x_0)^2\),即\(f''(\xi_2) = \dfrac{2}{(1 - x_0)^2}, \xi_2\)是对应\(x = 1\)是的\(\eta\)

记\(f''(\xi) = \max\{f''(\xi_1), f''(\xi_2)\}, \xi = \xi_1 or \xi_2\),于是\(\exists \xi \in (0, 1)\)使得\[f''(\xi) = 2\max\{\dfrac{1}{x_0^2}, \dfrac{1}{(1 - x_0)^2}\} >= 2 * \dfrac{1}{(\dfrac{1}{2})^2} = 8\]

若要证明\(\exists \xi \in (a, b)\)使得\(f''(\xi)\)大于或小于某个非零常数时,往往利用泰勒公式;


\subsubsection{拉格朗日证明不等式}
已知f(x)在[0, 1]上有二阶导数,且\(f(0) = 0, f(1) = 1, \displaystyle\int_0^1f(x)dx = 1\),证明:\begin{enumerate}
    \item 存在\(\xi \in (0 ,1)\),使得\(f'(\xi) = 0\)
    \item 存在\(\eta \in (0, 1)\),使得\(f''(\eta) < -2\)
\end{enumerate}

\subparagraph{证明1}
由拉格朗日中值定理得\(\exists c \in (0, 1)\),使得\(F(1) - F(0) = F'(c) = f(c) = 1\);\(f(c) = f(1)\)由罗尔定理得\(\exists\xi\)

\subparagraph{证明2}
令\(\varphi(x) = f(x) + x^2\),\(\varphi(0) = 0, \varphi(c) = 1 + c^2, \varphi(1) = 2\)。由拉格朗日中值定理得\(\exists\eta_1 \in (0, c), \eta_2 \in(c, 1)\),使得
\[\varphi'(\eta_1) = \dfrac{\varphi(c) - \varphi(0)}{c} = c + \dfrac{1}{c}\]
\[\varphi'(\eta_2) = \dfrac{\varphi(1) - \varphi(c)}{1 - c} = 1 + c\]
再由拉格朗日中值定理得\(\exists\eta\in(\eta_1, \eta_2)\),使得\[\varphi''(\eta) = \dfrac{\varphi'(\eta_2) - \varphi'(\eta_1)}{\eta_2 - \eta_1} = \dfrac{1 - \frac{1}{c}}{\eta_2 - \eta_1} < 0\]
又因为\(\varphi''(x) = f''(x) + 2\),故\(\varphi''(\eta) = f''(\eta) + 2 < 0, f''(\eta) < 2\)


\subsubsection{罗尔定理}
设\(f(x)\)在\([0, 1]\)上有2阶导数,且\(f(1) > 0\),\(\displaystyle\lim_{x \to 0^+}\dfrac{f(x)}{x} < 0\),证明:\begin{enumerate}
    \item 方程\(f(x) = 0\)在\((0, 1)\)内至少存在一个实根
    \item 方程\(f(x)f''(x) + [f'(x)]^2 = 0\)在区间\((0, 1)\)内至少存在2个不同的实根
\end{enumerate}

\subparagraph{证明1}
由题可知\(f(x)\)连续且\(\displaystyle\lim_{x \to 0^+}\dfrac{f(x)}{x}\)存在,故\(f(0) = 0\)。由极限保号性,\(\exists a \in (0, 1)\)使\(f(a) < 0\),\(\because f(1) > 0, \exists b \in (0, 1)\),得证。

\subparagraph{证明2}
由\(f(0) = f(b) = 0\),得\(\exists c \in (0, b)\),使\(f'(c) = 0\),设\(F(x) = f(x)f'(x)\),得证。


\section{积分}

\subsection{概念性质}

\subsubsection{判断原函数与定积分}
\begin{enumerate}
    \item \(f(x) = \begin{cases}
    2,\ x > 0 \\ 
    1,\ x = 0 \\ 
    -1,\ x < 0
    \end{cases}\),没有原函数,定积分存在;
    \item \(f(x) = \begin{cases}
        2x\sin\dfrac{1}{x^2} - \dfrac{2}{x}\cos\dfrac{1}{x^2},\ x \neq 0 \\ 
        0,\ x = 0
    \end{cases}\),
    \item \(f(x) = \begin{cases}
        \dfrac{1}{x},\ x \neq 0 \\ 
        0,\ x = 0
    \end{cases}\)
    \item \(f(x) = \begin{cases}
        2x\cos\dfrac{1}{x} + sin\dfrac{1}{x},\ x \neq 0 \\ 
        0,\ \ x = 0
    \end{cases}\)
\end{enumerate}

\paragraph{解}
\begin{enumerate}
    \item x = 0为跳跃间断点,任意包含x = 0的区间上不存在原函数;满足定积分存在定理,定积分存在
    \item x = 0,是振荡间断点,设\(F(x) = \begin{cases}
        x^2\sin\dfrac{1}{x^2},\ x \neq 0 \\ 
        0,\ x = 0
    \end{cases}\),则\(F'(x) = f(x), -\infty < x < +\infty\),故存在原函数;由于\(\infty\cos\infty\)为无界振荡,在任一包含x = 0的区间上定积分不存在;
    \item x = 0无穷间断点,在任一包含x = 0的区间上不存在原函数;定积分不存在
    \item 存在原函数\(F(x) = \begin{cases}
        x^2\cos\dfrac{1}{x},\ x \neq 0 \\ 
        0,\ \ x = 0
    \end{cases}\);有界且只有一个振荡间断点,定积分存在
\end{enumerate}

\subsubsection{定积分定义}
先提\(\dfrac{1}{n}\),再凑\(\dfrac{i}{n}\),由于\(\dfrac{i}{n} = 0 + \dfrac{1 - 0}{n}i\),\(\dfrac{i}{n}\)可读作0到1上的x,\(\dfrac{1}{n}\)可读作0到1上的dx
\begin{flalign}
\lim_{x \to \infty}(\dfrac{n + 1}{n^2 + 1} + ... + \dfrac{n + n}{n^2 + n^2}) & = \lim_{x \to \infty}\sum_{i = 1}^n\dfrac{n + i}{n^2 + i^2} \\ 
& = \lim_{x \to \infty}\sum_{i = 1}^n\dfrac{n^2 + ni}{n^2 + i^2} * \dfrac{1}{n} \\ 
& = \lim_{x \to \infty}\sum_{i = 1}^n\dfrac{1 + \dfrac{i}{n}}{1 + (\dfrac{i}{n})^2} * \dfrac{1}{n} \\ 
& = \int_0^1\dfrac{1 + x}{1 + x^2}dx
\end{flalign}

\subsubsection{函数有界}
设\(a > 0, f(x)\)在\([0, \infty)\)内连续有界,C为常数;证明\(y = e^{-ax}(\int_0^xf(t)e^{at}dt + C)\)有界。
\paragraph{证明}
设\(|f(x)| <= M\),当\(x >= 0\)时,有
\begin{flalign}
    |y(x)| & = |e^{-ax}(C + \int_0^xf(t)e^{at}dt)| <= |Ce^{-ax}| + e^{-ax}|\int_0^xf(t)e^{at}dt| \\ 
    & <= |C| + e^{-ax}\int_0^x|f(t)e^{at}|dt <= |C| + Me^{-ax}\int_0^xe^{at}dt \\ 
    & = |C| + \dfrac{M}{a}(1 - e^{-ax}) <= |C| + \dfrac{M}{a}
\end{flalign}


\subsubsection{敛散性}
设\(a > b > 0\),反常积分\(\displaystyle\int_0^{+\infty}\dfrac{1}{x^a + x^b}dx\)收敛,则?

\paragraph{解}
\(I = \displaystyle\int_0^1\dfrac{1}{x^a + x^b}dx + \int_1^{+\infty}\dfrac{1}{x^a + x^b}dx = I_1 + I_2\),

对\(I_1\)看\(x \to 0^+\),由于\(a > b > 0\),\(x^b\)趋于0的“速度”慢于\(x^a\)趋于0的“速度”,\(x^a + x^b \sim x^b\),则\(b < 1\),

对\(I_2\)看\(x \to +\infty\),由于\(a > b > 0\),\(x^a\)趋于\(+\infty\)“速度”大于\(x^b\)趋于\(+\infty\)的“速度”,\(x^a + x^b \sim x^a\),\(a > 1\)


\subsubsection{敛散性,比较判别}
已知\(a > 0\),则对反常积分\(\displaystyle\int_0^1\dfrac{\ln x}{x^a}dx\)敛散性的判别:

\paragraph{解}
当\(a < 1\)时,取充分小正数\(\varepsilon\)使得\(a + \varepsilon < 1\),由于\[\lim_{x \to 0^+}\dfrac{\dfrac{\ln x}{x^a}}{\dfrac{1}{x^{a + \varepsilon}}} = \lim_{x \to 0^+}\dfrac{\ln x}{x^{-\varepsilon}} = \lim_{x \to 0^+}\dfrac{\dfrac{1}{x}}{-\varepsilon x^{-\varepsilon - 1}} = \lim_{x \to 0^+}(-\dfrac{1}{\varepsilon}x^\varepsilon) = 0\]
由于\(\int_0^1\dfrac{1}{x^{a + \varepsilon}}dx\)收敛,故\(\int_0^1\dfrac{\ln x}{x^a}dx\)收敛,

当\(a >= 1\)时,由于\(\lim_{x \to 0^+}x^a\dfrac{\ln x}{x^a} = \infty,\ \int_0^1\dfrac{1}{x^a}dx\)发散,故\(\int_0^1\dfrac{\ln x}{x^a}dx\)发散


\subsubsection{敛散性,比较判别}
已知\(a > 0\),则对反常积分\(\displaystyle\int_1^{+\infty}\dfrac{\ln x}{x^a}dx\)敛散性的判别:

\paragraph{解}
当\(a <= 1\)且x充分大时,\(\dfrac{\ln x}{x^a} > \dfrac{1}{x^a}\),由于\(\int_1^{+\infty}\dfrac{1}{x^a}dx\)发散,故\(\int_1^{+\infty}\dfrac{\ln x}{x^a}dx\)发散

当\(a > 1\)时,取充分小正数\(\varepsilon\)使得\(a - \varepsilon > 1\),由于\(\lim_{x \to +\infty}\dfrac{\dfrac{\ln x}{x^a}}{\dfrac{1}{x^{a - \varepsilon}}} = \lim_{x \to +\infty}\dfrac{\ln x}{x^\varepsilon} = 0,\ \int_1^{+\infty}\dfrac{1}{x^{a - \varepsilon}}dx\)收敛,故\(\int_1^{+\infty}\dfrac{\ln x}{x^a}dx\)收敛


\subsubsection{定积分定义放缩}
\(\displaystyle\lim_{n \to \infty}\sum_{i = 1}^n\dfrac{\sin\dfrac{i\pi}{n}}{n + \dfrac{1}{i}} = ?\)

\paragraph{解}
当各项分母均为n时,\(\displaystyle\lim_{n \to \infty}\sum_{i = 1}^n\dfrac{\sin\dfrac{i\pi}{n}}{n} = \int_0^1\sin\pi xdx\)。因此先进行放缩
\[\sum_{i = 1}^n\dfrac{\sin\dfrac{i\pi}{n}}{n + 1} <= \sum_{i = 1}^n\dfrac{\sin\dfrac{i\pi}{n}}{n + \dfrac{1}{i}} <= \sum_{i = 1}^n\dfrac{\sin\dfrac{i\pi}{n}}{n}\]
\[\because\ \lim_{n \to \infty}\sum_{i = 1}^n\dfrac{\sin\dfrac{i\pi}{n}}{n + 1} = \lim_{n \to \infty}\dfrac{n}{n + 1} * \dfrac{1}{n}\sum_{i = 1}^n\sin\dfrac{i}{n}\pi = \int_0^1\sin\pi x\,dx\]
\[\lim_{n \to \infty}\sum_{i = 1}^n\dfrac{\sin\dfrac{i\pi}{n}}{n} = \lim_{n \to \infty}\dfrac{1}{n}\sum_{i = 1}^n\sin\dfrac{i}{n}\pi = \int_0^1\sin\pi x\,dx\]
\[\therefore\ \lim_{n \to \infty}\sum_{i = 1}^n\dfrac{\sin\dfrac{i\pi}{n}}{n + \dfrac{1}{i}} = \dfrac{2}{\pi}\]


\subsubsection{反常积分敛散性判别}
讨论\(\displaystyle\int_2^{+\infty}\dfrac{1}{x\ln^px}dx\)敛散性,其中p为任意实数

\subparagraph{解}
\begin{enumerate}
    \item 当p = 1时,\(\displaystyle\int_2^{+\infty}\dfrac{1}{x\ln x}dx = \ln|\ln x|\bigg|_2^{+\infty} = +\infty\),发散
    \item 当\(p \neq 1\)时,\[\int_2^{+\infty}\dfrac{dx}{x\ln^px} = \dfrac{1}{1 - p}(\ln x)^{1 - p}\bigg|_2^{+\infty}\]\begin{itemize}
        \item 当\(p > 1\)时,\(\displaystyle\lim_{x \to +\infty}(\ln x)^{1 - p} = 0\),收敛
        \item 当\(p < 1\)时,\(\displaystyle\lim_{x \to +\infty}(\ln x)^{1 - p} = +\infty\),发散
    \end{itemize}
\end{enumerate}
综上,\(\displaystyle\int_2^{+\infty}\dfrac{1}{x\ln^px}dx\begin{cases}
    \text{收敛},\ p > 1 \\ 
    \text{发散},\ p <= 1
\end{cases}\)


\subsubsection{反常积分敛散性判别}
反常积分\(1)\displaystyle\int_{-\infty}^0\dfrac{1}{x^2}e^{\frac{1}{x}}dx, 2)\int_0^{+\infty}\dfrac{1}{x^2}e^{\frac{1}{x}}dx\)的敛散性为?

\subparagraph{解}
设\(R < c < 0.\)\begin{align*}
    \int_{-\infty}^0\dfrac{1}{x^2}e^{\frac{1}{x}}dx & = \lim_{c \to 0^- R \to -\infty}\int_{R}^c\dfrac{1}{x^2}e^{\frac{1}{x}}dx \\
    & = \lim_{c \to 0^- R \to -\infty}\int_{R}^c-e^{\frac{1}{x}}d(\dfrac{1}{x}) \\
    & = \lim_{x \to 0^-}(-e^{\frac{1}{x}}) - \lim_{x \to -\infty}(-e^{\frac{1}{x}}) = 1
\end{align*}
设\(0 < c < R.\)\begin{align*}
    \int_0^{+\infty}\dfrac{1}{x^2}e^{\frac{1}{x}}dx = \lim_{x \to +\infty}(-e^{\frac{1}{x}}) - \lim_{x \to 0^+}(-e^{\frac{1}{x}})
\end{align*}
发散


\subsection{计算}

\subsubsection{换元,分部积分}
求\(\displaystyle\int\dfrac{xe^x}{\sqrt{e^x - 1}}dx\)

\paragraph{解}
令\(u = \sqrt{e^x - 1},\ x = \ln(1 + u^2),\ dx = \dfrac{2u}{1 + u^2}du\),则
\begin{flalign}
    \int\dfrac{xe^x}{\sqrt{e^x - 1}}dx & = \int\dfrac{(1 + u^2)\ln(1 + u^2)}{u} * \dfrac{2u}{1 + u^2}du = 2\int\ln(1 + u^2)du \nonumber \\ 
    & = 2u\ln(1 + u^2) - \int\dfrac{4u^2}{1 + u^2}du \nonumber
\end{flalign}


\subsubsection{换元}
求\(\displaystyle\int\dfrac{xe^{\arctan x}}{(1 + x^2)^{\frac{3}{2}}}dx\)

\paragraph{解}
令\(x = \tan t\),则\[\int\dfrac{xe^{\arctan x}}{(1 + x^2)^{\frac{3}{2}}}dx = \int\dfrac{e^t\tan t}{(1 + \tan^2t)^\frac{3}{2}}\sec^2tdt = \int e^t\sin tdt\]


\subsubsection{}
\(\displaystyle\int\dfrac{1}{1 + e^x}dx\)

\paragraph{解}
\[\int\dfrac{1}{1 + e^x}dx = \int(1 - \dfrac{e^x}{1 + e^x})dx = x - \ln(1 + e^x) + C\]


\subsubsection{分部积分}
求\(\displaystyle\int e^{2x}(\tan x + 1)^2dx\)

\paragraph{解}
\begin{flalign}
    \int e^{2x}(\tan x + 1)^2dx & = \int e^{2x}(\sec^2x + 2\tan x)dx \nonumber \\ 
    & = \int e^{2x}\sec^2xdx + 2\int e^{2x}\tan xdx \nonumber \\ 
    & = e^{2x}\tan x - 2\int e^{2x}\tan xdx + 2\int e^{2x}\tan xdx \nonumber \\ 
    & = e^{2x}\tan x + C \nonumber
\end{flalign}


\subsubsection{分式积分}
\(\displaystyle\int\dfrac{2x + 3}{x^2 - x + 1}dx\)

\paragraph{解}
\begin{flalign}
    \text{上式} & = \int\dfrac{2x - 1}{x^2 - x + 1}dx + \int\dfrac{4}{x^2 -x + 1}dx \nonumber \\ 
    & = \ln(x^2 - x + 1) + 4\int\dfrac{1}{(x - \frac{1}{2})^2 + (\frac{\sqrt{3}}{2})^2}d(x - \dfrac{1}{2}) \nonumber \\ 
    & = \ln(x^2 - x + 1) + (\dfrac{8}{\sqrt{3}}\arctan \dfrac{2x - 1}{\sqrt{3}}) + C
\end{flalign}


\subsubsection{换元为奇函数}
\(\displaystyle\lim_{n \to \infty}\dfrac{1}{n}\sum_{i = 1}^n[\ln(3n - 2i) - \ln(n + 2i)] = \)

\paragraph{解}
\begin{flalign}
    \text{上式} & = \int_0^1\ln\dfrac{3 - 2x}{1 + 2x}dx = \int_0^1\ln\dfrac{\frac{3}{2} - x}{\frac{1}{2} + x}dx \nonumber \\ 
    & = \int_{-\frac{1}{2}}^{\frac{1}{2}}\ln\dfrac{1 - t}{1 + t}dt = 0
\end{flalign}


\subsubsection{换元}
\(\displaystyle\int_0^1x\arcsin\sqrt{4x - 4x^2}dx\)

\paragraph{解}
\begin{flalign}
    \int_0^1x\arcsin\sqrt{4x - 4x^2}dx & = \int_0^1x\arcsin\sqrt{1 - (1 - 2x)^2}dx \nonumber \\ 
    & = \dfrac{1}{2}\int_1^{-1}(1 - t)\arcsin\sqrt{1 - t^2}(-\dfrac{1}{2}dt) = \dfrac{1}{4}\int_{-1}^1(1 - t)\arcsin\sqrt{1 - t^2}dt \nonumber \\ 
    & = \dfrac{1}{2}\int_0^1\arcsin\sqrt{1 - t^2}dt = \dfrac{1}{2}
\end{flalign}


\subsubsection{变限两个未知数}
\(F(x) = \displaystyle\int_0^{\frac{\pi}{2}}|\sin x - \sin t|dt,\ (x >= 0)\)在\(x \to 0^+\)处的二次泰勒多项式为\(a + bx + cx^2\),则\(abc = ?\)

\paragraph{解}
当\(x \to 0^+\)时,
\begin{flalign}
    F(x) & = \displaystyle\int_0^x(\sin x - \sin t)dt + \int_x^{\frac{\pi}{2}}(\sin t - \sin x)dt \nonumber \\ 
    & = x\sin x + (\cos x - 1) + \cos x - \sin x * (\dfrac{\pi}{2} - x) \nonumber \\ 
    & = (2x - \dfrac{\pi}{2})\sin x + 2\cos x - 1
\end{flalign}

\subparagraph{方式1}
直接展开
\[\sin x = x + o(x^2)\]
\[\cos x = 1 - \dfrac{1}{2}x^2 + o(x^2)\]
\[F(x) = 1 - \dfrac{\pi}{2}x + x^2 + o(x^2)\]
得\(abc = -\dfrac{\pi}{2}\)

\subparagraph{方式2}
\[F'(x) = (2x - \dfrac{\pi}{2})\cos x\]
\[F''(x) = 2\cos x - (2x - \dfrac{\pi}{2})\sin x\]
\[\therefore F(0) = 1, F'_+(0) = -\dfrac{\pi}{2}, F''_+(0) = 2\]
\[F(x) = F(0) + F'_+(0)x + \dfrac{F''_+(0)}{2!}x^2 + ...\]
得\(abc = -\dfrac{\pi}{2}\)


\subsubsection{无穷区间,换元}
\(\displaystyle\int_3^{+\infty}\dfrac{dx}{(x - 1)^4\sqrt{x^2 - 2x}}\)

\paragraph{解}
\begin{flalign}
    & = \int_3^{+\infty}\dfrac{dx}{(x - 1)^4\sqrt{(x - 1)^2 - 1}} \xrightarrow{x - 1 = \sec \theta}\int_\frac{\pi}{3}^\frac{\pi}{2}\dfrac{\sec\theta\tan\theta}{\sec^4\theta\tan\theta}d\theta \nonumber \\ 
    & = \int_\frac{\pi}{3}^\frac{\pi}{2}(1 - \sin^2\theta)\cos\theta d\theta = \dfrac{2}{3} - \dfrac{3\sqrt{3}}{8} \nonumber
\end{flalign}


\subsubsection{\(\Gamma\)函数}
设\(f(x) = \begin{cases}
    \dfrac{4x^2}{a^3\sqrt{\pi}}e^{-\frac{x^2}{a^2}}\ ,\ x > 0 \\ 
    0\ ,\ x <= 0
\end{cases}\),a为正常数,则\(\displaystyle\int_0^{+\infty}x^2f(x)dx =\)?

\paragraph{解}
\begin{flalign}
    \int_0^{+\infty}x^2f(x)dx & = \dfrac{2a^2}{\sqrt{\pi}} * 2\int_0^{+\infty}(\dfrac{x}{a})^{2 * \frac{5}{2} - 1}e^{-(\frac{x}{a})^2}\ d(\dfrac{x}{a}) \nonumber \\ 
    & = \dfrac{2a^2}{\sqrt{\pi}} * \Gamma(\dfrac{5}{2}) = \dfrac{3}{2}a^2 \nonumber
\end{flalign}


\subsubsection{分段函数}
\(\displaystyle\int\ \max\{1, |x|\}\ dx\)

\paragraph{解}
\(\max\{1, |x|\} = \begin{cases}
    -x,\ x < -1 \\ 
    1,\ -1 <= x <= 1 \\ 
    x,\ x > 1
\end{cases}\)
由于f(x)连续,则必存在原函数\(F(x) = \begin{cases}
    -\dfrac{x^2}{2} + C_1,\ x < -1 \\ 
    x + C_2,\ -1 <= x <= 1 \\ 
    \dfrac{x^2}{2} + C_3,\ x > 1
\end{cases}\)
又F(x)连续,则\(\begin{cases}
    -\dfrac{1}{2} + C_1 = -1 + C_2 \\ 
    1 + C_2 = \dfrac{1}{2} + C_3
\end{cases}\),得原式\(= \begin{cases}
    -\dfrac{x^2}{2} + C,\ x < -1 \\ 
    x + \dfrac{1}{2} + C,\ -1 <= x <= 1 \\ 
    \dfrac{x^2}{2} + 1 + C,\ x > 1
\end{cases}\)


\subsubsection{换元}
\(\displaystyle\int\arcsin\sqrt{\dfrac{x}{a + x}}dx\)

\paragraph{解}
令\(\arcsin\sqrt{\dfrac{x}{a + x}} = t,\ x = \dfrac{a\sin^2t}{1 - \sin^2t} = a\tan^2t\)
\begin{flalign}
    \int\arcsin\sqrt{\dfrac{x}{a + x}}dx = & \int td(a\tan^2t) = at\tan^2t - a\int\tan^2tdt \nonumber \\ 
    = & at\tan^2t + a\int(1 - \sec^2t)dt = at\tan^2t + at - a\tan t + C \nonumber \\ 
    = & (a + x)\arcsin\sqrt{\dfrac{x}{a + x}} - \sqrt{ax} + C \nonumber
\end{flalign}


\subsubsection{求满足条件函数,换元}
求连续函数\(f(x)\)使其满足\(\int_0^1f(tx)dt = f(x) + x\sin x\)

\paragraph{解}
令\(tx = u\),则原式化为\(\displaystyle\dfrac{1}{x}\int_0^xf(u)du = f(x) + x\sin x\),即\[\displaystyle\int_0^xf(u)du = xf(x) + x^2\sin x\]
两边对x求导得:\[f(x) = f(x) + xf'(x) + 2x\sin x + x^2\cos x\]
\[f'(x) = -2\sin x - x\cos x\]
积分得\[f(x) = \cos x - x\sin x + C\]


\subsubsection{换元,三角函数}
设\(f(x) = \begin{cases}
    \dfrac{1}{1 + \sin x},\ x >= 0 \\ 
    \dfrac{1}{1 + e^x},\ x < 0
\end{cases}\),求\(\displaystyle\int_{-1}^{\frac{\pi}{4}}f(x)dx\)

\paragraph{解}
\begin{flalign}
    \int_{-1}^0\dfrac{dx}{1 + e^x} & \xrightarrow{e^x = t} \int_{e^{-1}}^1\dfrac{1}{1 + t} * \dfrac{1}{t}dt = \int_{e^{-1}}^1(\dfrac{1}{t} - \dfrac{1}{1 + t})dt \nonumber \\ 
    & = \ln\dfrac{t}{1 + t}\bigg|_{e^{-1}}^1 = -\ln 2 + \ln(1 + e) \nonumber
\end{flalign}

\begin{flalign}
    \int_0^{\frac{\pi}{4}}\dfrac{dx}{1 + \sin x} & = \int_0^{\frac{\pi}{4}}\dfrac{1 - \sin x}{\cos^2x}dx = \int_0^{\frac{\pi}{4}}\sec^2xdx - \int_0^{\frac{\pi}{4}}\dfrac{\sin x}{\cos^2x}dx \nonumber \\ 
    & =\tan x\bigg|_0^{\frac{\pi}{4}} - \dfrac{1}{\cos x}\bigg|_0^{\frac{\pi}{4}} = 2 - \sqrt{2} \nonumber
\end{flalign}


\subsubsection{换元,奇偶性}
\(\displaystyle\int_{-1}^1\dfrac{x + 1}{1 + \sqrt[3]{x^2}}dx\)

\paragraph{解}
\begin{flalign}
    \int_{-1}^1\dfrac{x + 1}{1 + \sqrt[3]{x^2}}dx & = \int_{-1}^1\dfrac{x}{1 + \sqrt[3]{x^2}}dx + \int_{-1}^1\dfrac{1}{1 + \sqrt[3]{x^2}}dx = 0 + 2\int_0^1\dfrac{1}{1 + \sqrt[3]{x^2}}dx \nonumber \\ 
    & \xrightarrow{\sqrt[3]{x^2} = t} 3\int_0^1\dfrac{\sqrt{t}}{1 + t}dt \xrightarrow{\sqrt{t} = u} 6\int_0^1\dfrac{u^2}{1 + u^2}du \nonumber \\ 
    & = 6 - 6\arctan 1 = 6 - \dfrac{3}{2}\pi \nonumber
\end{flalign}


\subsubsection{三角函数}
\(\displaystyle\int_0^{\frac{3}{4}\pi}\dfrac{1}{1 + \cos^2x}dx\)

\paragraph{解}
\begin{flalign}
    \int_0^{\frac{3}{4}\pi}\dfrac{1}{1 + \cos^2x}dx & = \int_0^{\frac{\pi}{2}}\dfrac{1}{1 + \cos^2x}dx + \int_{\frac{\pi}{2}}^{\frac{3}{4}\pi}\dfrac{1}{1 + \cos^2x}dx \nonumber \\ 
    & = \lim_{x \to (\frac{\pi}{2})^-}F(x) - F(0) + F(\dfrac{3}{4}\pi) - \lim_{x \to (\frac{\pi}{2})^+}F(x) \nonumber \\ 
    & = \dfrac{\pi}{\sqrt{2}} - \dfrac{1}{\sqrt{2}}\arctan\dfrac{1}{\sqrt{2}} \nonumber
\end{flalign}
其中,\begin{flalign}
    F(x) &  = \int\dfrac{1}{1 + \cos^2x}dx = \int\dfrac{\sec^2x}{2 + \tan^2x}dx \nonumber \\ 
    & = \int\dfrac{\sqrt{2}d(\dfrac{\tan x}{\sqrt{2}})}{2[1 + (\dfrac{\tan x}{\sqrt{2}})^2]} = \dfrac{1}{\sqrt{2}}\arctan\dfrac{\tan x}{\sqrt{2}} + C \nonumber
\end{flalign}


\subsubsection{三角函数,换元}
\(\displaystyle\int_0^\pi\dfrac{x\sin x}{1 + \cos^2x}dx\)

\paragraph{解}
\begin{flalign}
    \int_0^\pi\dfrac{x\sin x}{1 + \cos^2x}dx & \xrightarrow{x = \pi - t} \int_\pi^0\dfrac{(\pi - t)\sin(\pi - t)}{1 + \cos^2(\pi - t)}(-dt) \nonumber \\ 
    & = \int_0^\pi\dfrac{(\pi - t)\sin t}{1 + \cos^2t}dt = \pi\int_0^\pi\dfrac{\sin t}{1 + \cos^2t}dt - \int_0^\pi\dfrac{t\sin t}{1 + \cos^2t}dt \nonumber \\ 
    & = \pi\int_0^\pi\dfrac{\sin t}{1 + \cos^2t}dt - \int_0^\pi\dfrac{x\sin x}{1 + \cos^2x}dx \nonumber \\ 
    \int_0^\pi\dfrac{x\sin x}{1 + \cos^2x}dx & = \dfrac{\pi}{2}\int_0^\pi\dfrac{\sin t}{1 + \cos^2t}dt = -\dfrac{\pi}{2}\int_0^\pi\dfrac{1}{1 + \cos^2t}d(\cos t) \nonumber \\ 
    & = -\dfrac{\pi}{2}\arctan(\cos t)\bigg|_0^\pi = \dfrac{\pi^2}{4} \nonumber
\end{flalign}


\subsubsection{三角函数}
\(\displaystyle\int_{-\dfrac{\pi}{4}}^{\dfrac{\pi}{4}}e^{\dfrac{x}{2}}\dfrac{\cos x - \sin x}{\sqrt{\cos x}}dx\)

\paragraph{解}
\begin{flalign}
    \int_{-\dfrac{\pi}{4}}^{\dfrac{\pi}{4}}e^{\dfrac{x}{2}}\dfrac{\cos x - \sin x}{\sqrt{\cos x}}dx & = \int_{-\dfrac{\pi}{4}}^{\dfrac{\pi}{4}}e^{\dfrac{x}{2}}\sqrt{\cos x}dx - \int_{-\dfrac{\pi}{4}}^{\dfrac{\pi}{4}}e^{\dfrac{x}{2}}\dfrac{\sin x}{\sqrt{\cos x}}dx \nonumber \\ 
    & = \int_{-\dfrac{\pi}{4}}^{\dfrac{\pi}{4}}e^{\dfrac{x}{2}}\sqrt{\cos x}dx + 2\int_{-\dfrac{\pi}{4}}^{\dfrac{\pi}{4}}e^{\dfrac{x}{2}}d(\sqrt{\cos x}) \nonumber \\ 
    & = \int_{-\dfrac{\pi}{4}}^{\dfrac{\pi}{4}}e^{\dfrac{x}{2}}\sqrt{\cos x}dx + 2e^{\dfrac{x}{2}}\sqrt{\cos x}\bigg|_{-\dfrac{\pi}{4}}^{\dfrac{\pi}{4}} - \int_{-\dfrac{\pi}{4}}^{\dfrac{\pi}{4}}e^{\dfrac{x}{2}}\sqrt{\cos x}dx \nonumber \\ 
    & = \sqrt[4]{8}(e^{\frac{\pi}{8}} - e^{-\frac{\pi}{8}}) \nonumber
\end{flalign}


\subsubsection{三角函数,换元}
\(\displaystyle\int_{\frac{1}{2}}^{\frac{3}{2}}\dfrac{(1 - x)\arcsin(1 - x)}{\sqrt{2x - x^2}}dx\)

\paragraph{解}
\begin{flalign}
    \int_{\frac{1}{2}}^{\frac{3}{2}}\dfrac{(1 - x)\arcsin(1 - x)}{\sqrt{2x - x^2}}dx & \xrightarrow{1 - x = \sin t} \int_{-\dfrac{\pi}{6}}^{\dfrac{\pi}{6}}\dfrac{t\sin t}{\cos t}\cos tdt = \int_{-\dfrac{\pi}{6}}^{\dfrac{\pi}{6}}t\sin tdt \nonumber \\ 
    & = -\int_{-\dfrac{\pi}{6}}^{\dfrac{\pi}{6}}td(\cos t) = -(t\cos t - \sin t)\bigg|_{-\dfrac{\pi}{6}}^{\dfrac{\pi}{6}} \nonumber \\ 
    & = 1 - \dfrac{\sqrt{3}\pi}{6} \nonumber
\end{flalign}


\subsubsection{定积分定义夹逼}
\(\displaystyle\lim_{n \to \infty}\sum_{i = 1}^n\dfrac{n}{n^2 + i^2 + 1} = \)

\subparagraph{解}
\[\sum_{i = 1}^n\dfrac{1}{n}\dfrac{1}{1 + \dfrac{(i + 1)^2}{n^2}} <= \sum_{i = 1}^n\dfrac{1}{n}\dfrac{1}{1 + \dfrac{i^2 + 1}{n^2}} <= \sum_{i = 1}^n\dfrac{1}{n}\dfrac{1}{1 + \dfrac{i^2}{n^2}}\]
\[\sum_{i = 1}^n\dfrac{1}{n}\dfrac{1}{1 + \dfrac{(1 + i)^2}{n^2}} = \sum_{i = 1}^n\dfrac{1}{n}\dfrac{1}{1 + \dfrac{i^2}{n^2}} - \dfrac{1}{n}\dfrac{1}{1 + \dfrac{1}{n^2}} + \dfrac{1}{n}\dfrac{1}{1 + \dfrac{(n + 1)^2}{n^2}}\]
\[\therefore\ \lim_{n \to \infty}\sum_{i = 1}^n\dfrac{n}{n^2 + i^2 + 1} = \dfrac{\pi}{4}\]


\subsubsection{反常函数积分}
\(\displaystyle\int_0^{+\infty}\dfrac{xe^{-x}}{(1 + e^{-x})^2}dx = \)

\subparagraph{解}
\begin{flalign}
    \text{原式} & = \lim_{b \to +\infty}\int_0^bxd(\dfrac{1}{1 + e^{-x}}) = \lim_{b \to +\infty}(\dfrac{x}{1 + e^{-x}}\bigg|_0^b - \int_0^b\dfrac{dx}{1 + e^{-x}}) \nonumber \\ 
    & = \lim_{b \to +\infty}(\dfrac{b}{1 + e^{-b}} - \ln(1 + e^{-x})\bigg|_0^b) = \ln2 + \lim_{b \to +\infty}(\dfrac{b}{1 + e^{-b}} - \ln(e^b + 1)) \nonumber \\ 
    & = \ln2 + \lim_{b \to +\infty}\dfrac{1}{1 + e^{-b}}(b - (1 + e^{-b})\ln(e^b + 1)) \nonumber \\ 
    & = \ln2 + \lim_{b \to +\infty}(\ln e^b - (1 + e^b)\dfrac{\ln(1 + e^b)}{e^b}) \nonumber \\ 
    & = \lim_{b \to +\infty}(\ln\dfrac{e^b}{e^b + 1} - \dfrac{\ln(1 + e^b)}{e^b}) = \ln2 + \ln1 - 0 \nonumber \\
    & = \ln2 \nonumber
\end{flalign}


\subsubsection{\(e^{t^2x}\)换元}
对\(\displaystyle\int e^{t^2x}dt\),令\(t\sqrt{x} = u, dt = \dfrac{1}{\sqrt{x}}du\),则\[\int e^{t^2x}dt = \dfrac{1}{\sqrt{x}}\int e^{u^2}du\]


\subsubsection{绝对值奇偶性}
\(\displaystyle\int_{-1}^1(x + 2|x|)^2dx\),展开得\(x^2 + 2x|x| + 4x^2\),对称性有\[ = 10\int_0^1x^2dx\]


\subsubsection{三角函数}
\(\displaystyle\int\dfrac{dx}{\cos x + \sin x}\)
\paragraph{解}
令\(\tan\dfrac{x}{2} = t\),则\(x = 2\arctan t,\ dx = \dfrac{2dt}{1 + t^2},\ \sin x = \dfrac{2t}{1 + t^2}, \cos x = \dfrac{1 - t^2}{1 + t^2}\),代入得
\[ = \int\dfrac{2dt}{1 + 2t - t^2} = \int\dfrac{2dt}{2 - (1 - t)^2}\]


\subsubsection{\(e^{tx - t^2}\)变限积分换元}
\(F(x) = \displaystyle\int_0^xe^{tx - t^2}dt\),求\(F'(x)\)
\subparagraph{解}
\begin{flalign}
    F(x) & = \int_0^xe^{\frac{x^2}{4} - (\frac{x}{2} - t)^2}dt = e^{\frac{x^2}{4}}\int_0^xe^{- (\frac{x}{2} - t)^2}d(t - \dfrac{x}{2}) \nonumber \\ 
    & \xrightarrow{u = \frac{x}{2} - t} -e^{\frac{x^2}{4}}\int_{\frac{x}{2}}^{-\frac{x}{2}}e^{-u^2}du = 2e^{\frac{x^2}{4}}\int_{0}^{\frac{x}{2}}e^{-u^2}du \nonumber
\end{flalign}


\subsubsection{三角函数,原函数}
\(\displaystyle f(x) = \dfrac{1}{1 + \sin^2x}, x\in[0, \pi]\),则\(f(x)\)在\([0, \pi]\)上全体原函数为
\subparagraph{解}
\begin{flalign}
    \int\dfrac{dx}{1 + \sin^2x} & = \int\dfrac{\dfrac{1}{\cos^2x}dx}{\dfrac{1}{\cos^2x} + \tan^2x} \nonumber \\ 
    & = \int\dfrac{d\tan x}{1 + 2\tan^2x} \nonumber \\ 
    & = \dfrac{1}{\sqrt{2}}\arctan(\sqrt{2}\tan x) + C \nonumber
\end{flalign}

\(\because\)在\(x = \dfrac{\pi}{2}\)上无定义,

\(\therefore\)左右分别求极限,得\(F(x) + C\),其中
\[F(x) = \begin{cases}
    \dfrac{1}{\sqrt{2}}\arctan(\sqrt{2}\tan x) - \dfrac{\pi}{2\sqrt{2}}, 0 <= x < \dfrac{\pi}{2} \\ 
    0,\ \ x = \dfrac{\pi}{2} \\ 
    \dfrac{1}{\sqrt{2}}\arctan(\sqrt{2}\tan x) + \dfrac{\pi}{2\sqrt{2}}, \dfrac{\pi}{2} < x <= \pi
\end{cases}\]


\subsubsection{分段三角函数}
设n正整数,\(S_n\)为曲线\(y = e^{-x}\sin x\,(0 <= x <= n\pi)\)与x轴所围图形的面积,求\(S_n\)及\(\displaystyle\lim_{n \to \infty}S_n\)
\subparagraph{解}
\begin{align}
    S_n & = \sum_{k = 0}^{n  -1}(-1)^k\int_{k\pi}^{(k + 1)\pi}e^{-x}\sin xdx \nonumber \\
    & = \sum_{k = 0}^{n  -1}(-1)^k[-\dfrac{1}{2}e^{-x}(\sin x + \cos x)]\bigg|_{k\pi}^{(k + 1)\pi} \nonumber \\ 
    & = \dfrac{1}{2}\sum_{k = 0}^{n  -1}(-1)^{k + 1}[e^{-(k + 1)\pi}(-1)^{k + 1} - e^{-k\pi}(-1)^{k}] \nonumber \\ 
    & = \dfrac{1}{2}\sum_{k = 0}^{n  -1}(e^{-(k + 1)\pi} + e^{-k\pi}) = \dfrac{1}{2}(1 + 2\sum_{k = 1}^{n}e^{-k\pi} - e^{-n\pi}) \nonumber \\
    & = \dfrac{1}{2}[1 + \dfrac{2e^{-\pi}(1 - e^{-n\pi})}{1 - e^{-\pi}} - e^{-n\pi}] \nonumber
\end{align}
故\(\displaystyle\lim_{n \to \infty}S_n = \lim_{n \to \infty}\dfrac{1}{2}[1 + \dfrac{2e^{-\pi}(1 - e^{-n\pi})}{1 - e^{-\pi}} - e^{-n\pi}] = \dfrac{1}{2} + \dfrac{1}{e^\pi - 1}\)


\section{二重积分}

\subsubsection{极坐标平移}
设D为圆域\(x^2 + y^2 <= 2x + 2y\),则\(\displaystyle\iint_Dxydxdy = \) ?
\subparagraph{解}
\((x - 1)^2 + (y - 1)^2 = 2\),令\(x = 1 + \rho\cos\theta, y = 1 + \rho\sin\theta\),则
\begin{flalign}
    \iint_Dxydxdy & = \int_0^{2\pi}d\theta\int_0^{\sqrt{2}}(1 + \rho\cos\theta)(1 + \rho\sin\theta)\rho d\rho \nonumber \\ 
    & = \int_0^{2\pi}d\theta\int_0^{\sqrt{2}}(1 + \rho\cos\theta + \rho\sin\theta + \rho^2\sin\theta\cos\theta)\rho d\rho \nonumber \\ 
    & = \int_0^{2\pi}d\theta\int_0^{\sqrt{2}}\rho d\rho = 2\pi \nonumber
\end{flalign}


\subsubsection{\(x^x\)}
设积分区域D由曲线\(y = \ln x\)及直线\(x = 2, y = 0\)围成,则\(\displaystyle\iint_D\dfrac{e^{xy}}{x^x - 1}\mathrm{d}\sigma = \)?
\subparagraph{解}
由题设知积分区域\(D=\{(x,y)\mid1\leqslant x\leqslant2,\ 0\leqslant y\leqslant\ln x\}\),从而 \\
\(\begin{aligned}
    \iint_{D}\frac{e^{xy}}{x^{x} - 1}\mathrm{d}\sigma & = \int_{1}^{2}\mathrm{d}x\int_{0}^{\ln x}\frac{\mathrm{e}^{xy}}{x^{x} - 1}\mathrm{d}y \\
    & = \int_{1}^{2}\frac{\mathrm{d}x}{x^{x} - 1}\int_{0}^{\ln x}\mathrm{e}^{xy}\mathrm{d}y \\
    & = \int_{1}^{2}\frac{\mathrm{d}x}{x(x^{x} - 1)}\int_{0}^{\ln x}\mathrm{e}^{xy}\mathrm{d}\left(xy\right) \\
    & = \int_{1}^{2}\frac{\mathrm{e}^{xy}}{x(x^{x} - 1)}\bigg|_{y=0}^{y=\ln x}\mathrm{d}x \\ 
    & = \int_{1}^{2}\frac{e^{x\ln x} - 1}{x(x^{x} - 1)}dx=\int_{1}^{2}\frac{dx}{x}=\ln2
\end{aligned}\)


\subsubsection{对称性}
已知平面区域\(D = \{(x, y)\bigg||x| <= y, (x^2 + y^2)^3 <= y^4\}\),计算\(\displaystyle\iint_D\dfrac{x + y}{\sqrt{x^2 + y^2}}dxdy\)
\subparagraph{解}
将\(x\)替换为\(-x\)得D关于y轴对称,被积函数\(f(x, y) = \dfrac{x + y}{\sqrt{x^2 + y^2}} = \dfrac{x}{\sqrt{x^2 + y^2}} + \dfrac{y}{\sqrt{x^2 + y^2}}\)。其中\(\dfrac{x}{\sqrt{x^2 + y^2}}\)是关于x的奇函数,\(\dfrac{y}{\sqrt{x^2 + y^2}}\)是关于x的偶函数。故原式\(= \displaystyle\iint_D\dfrac{y}{\sqrt{x^2 + y^2}}dxdy\)


\subsubsection{参数方程}
设平面区域D由\(\begin{cases}
    x = t - \sin t \\
    y = 1 - \cos t
\end{cases}(0 <= t <= 2\pi)\)与x轴围成,计算二重积分\(\displaystyle\iint_D(x + 2y)dxdy\)

\subparagraph{解}
\begin{align*}
    \text{原积分} & = \int_0^{2\pi}dx\int_0^{\varphi(x)}(x + 2y)dy = \int_0^{2\pi}[x\varphi(x) + \varphi^2(x)]dx \\
    & \rightarrow \int_0^{2\pi}[(t - \sin t)(1 - \cos t) + (1 - \cos t)^2]d(t - \sin t) \\
    & = \int_0^{2\pi}[(t - \sin t)(1 - \cos t)^2]dt + \int_0^{2\pi}[(1 - \cos t)^3]dt = 3\pi^2 + 5\pi
\end{align*}


\section{几何应用}

\subsubsection{参数方程/微分/旋转体体积}
摆线\(x = a(t - \sin t), y = a(1 - \cos t),\ (0 <= t <= 2\pi)\)与x轴围成图形绕\(y = 2a\)旋转体体积V = ?

\subparagraph{解}
摆线\(y = y(x),\ (0 <= x <= 2\pi a)\)。

设摆线与直线\(y = 2a, x = 0, x = 2\pi a\)围成图形绕\(y = 2a\)旋转一周所成的旋转体的体积\(V_1\)。任取\([x, x + dx] \subset [0, 2\pi a]\),对应部分相应体积微元为\(dV_1 = \pi[2a - y(x)]^2dx\),则\begin{flalign}
    V_1 & = \pi\int_0^{2\pi a}(2a - y)^2dx \xrightarrow{x = a(t - \sin t)} \pi\int_0^{2\pi}[2a - a(1 - \cos t)]^2a(1 - \cos t)dt \nonumber \\ 
    & = \pi a^3\int_0^{2\pi}(1 + \cos t)^2(1 - \cos t)dt = \pi a^3\int_0^{2\pi}(1 + \cos t - \cos^2 t - \cos^3t)dt \nonumber \\ 
    & = \pi^2a^3 \nonumber
\end{flalign}
故\(V = \pi(2a)^{2\pi a} - V_1 = 7\pi^2a^3\)


\subsubsection{参数方程/面积/弧长/旋转体体积/侧面积}
设星形线方程\(\begin{cases}
    x = a\cos^3t \\ 
    y = a\sin^3t
\end{cases}\),则围成的面积A为?弧长L为?绕x轴旋转得旋转体体积V为?旋转体侧面积S为?

\subparagraph{解}
\begin{flalign}
    A & = 4\int_0^aydx = 4\int_{\frac{\pi}{2}}^0a\sin^3t * 3a\cos^2t(-\sin t)dt \nonumber \\ 
    & = 12\int_0^\frac{\pi}{2}a^2(\sin^4t - \sin^6t)dt \nonumber \\ 
    & = 12a^2 * (\dfrac{1 * 3}{2 * 4} - \dfrac{1 * 3 * 5}{2 * 4 * 6})\dfrac{\pi}{2} = \dfrac{3}{8}\pi a^2 \nonumber
\end{flalign}
\begin{flalign}
    L & = 4\int_0^\frac{\pi}{2}\sqrt{(x')^2 + (y')^2}dt = 4\int_0^\frac{\pi}{2}\sqrt{3^2a^2(\cos^4t\sin^2t + \sin^4t\cos^2t)}dt \nonumber \\ 
    & = 4\int_0^\frac{\pi}{2}3a\cos t\sin tdt = 6a(\sin t)^2\bigg|_0^\frac{\pi}{2} = 6a \nonumber
\end{flalign}
\begin{flalign}
    V & = 2\int_0^a\pi y^2dx = 2\int_\frac{\pi}{2}^0\pi a^2\sin^6t * 3a\cos^2t(-\sin t)dt \nonumber \\ 
    & = 6\pi a^3\int_0^\frac{\pi}{2}\sin^7t(1 - \sin^2t)dt \nonumber \\ 
    & = 6\pi a^3[\dfrac{6 * 4 * 2}{7 * 5 * 3}(1 - \dfrac{8}{9})] = \dfrac{32}{105}\pi a^3 \nonumber
\end{flalign}
\begin{flalign}
    S & = 2\pi\int_0^\pi a\sin^3t\sqrt{x'^2(t) + y'^2(t)}dt \nonumber \\ 
    & = 2\pi\int_0^\pi3a^2\sin^3t\sqrt{\sin^2t\cos^2t}dt \nonumber \\ 
    & = 6\pi a^2\int_0^\pi\sin^4t|\cos t|dt = 6\pi a^2\int_{-\frac{\pi}{2}}^\frac{\pi}{2}\sin^4t\cos tdt \nonumber \\ 
    & = \dfrac{12}{5}\pi a^2 \nonumber
\end{flalign}

\subparagraph{隐函数求法}
给出星形线得隐函数\(x^\frac{2}{3} + y^\frac{2}{3} = a^\frac{2}{3}\)

则\(\dfrac{2}{3}x^{-\frac{1}{3}} + \dfrac{2}{3}y^{-\frac{1}{3}}y' = 0,\ \therefore\ y' = -\dfrac{y^{\frac{1}{3}}}{x^{\frac{1}{3}}}\)

\(\therefore\ \sqrt{1 + y'^2} = \sqrt{\dfrac{x^\frac{2}{3} + y^\frac{2}{3}}{x^\frac{2}{3}}} = \dfrac{a^\frac{1}{3}}{x^\frac{1}{3}}\)

\begin{flalign}
    \therefore S & = 2 * 2\pi\int_0^ay\sqrt{1 + y'^2}dx = 4\pi\int_0^a(a^\frac{2}{3} - x^\frac{2}{3})^\frac{3}{2}\dfrac{a^\frac{1}{3}}{x^{\frac{1}{3}}}dx \nonumber \\ 
    & = 4\pi * \dfrac{3}{2}\int_0^a(a^\frac{2}{3} - x^\frac{2}{3})^\frac{3}{2}a^\frac{1}{3}dx^\frac{2}{3} = 6\pi a^\frac{1}{3}(-\dfrac{2}{5})(a^\frac{2}{3} - x^\frac{2}{3})^\frac{5}{2}\bigg|_0^a = \dfrac{12}{5}\pi a^2 \nonumber
\end{flalign}
\begin{flalign}
    A & = 4\int_0^a(a^\frac{2}{3} - x^\frac{2}{3})^\frac{3}{2}dx \xrightarrow{x^\frac{1}{3} = a^\frac{1}{3}\cos t} 4\int_0^\frac{\pi}{2}a\sin^3t * 3a * \cos^2t\sin tdt \nonumber \\ 
    & = 12a^2\int_0^\frac{\pi}{2}(\sin^4t - \sin^6t)dt \nonumber
\end{flalign}


\subsubsection{条件判断}
已知f(x),g(x)2阶可导且2阶导数在\(x = a\)连续,则\(\displaystyle\lim_{x \to a}\dfrac{f(x) - g(x)}{(x - a)^2} = 0\)是曲线\(y = f(x)\)和\(y = g(x)\)在\(x = a\)处相切且曲率相等的:\underline{充分非必要条件}
\subparagraph{解}
由\(\displaystyle\lim_{x \to a}\dfrac{f(x) - g(x)}{(x - a)^2} = 0\)得\(f(a) = g(a)\);由\(\displaystyle\lim_{x \to a}\dfrac{f'(x) - g'(x)}{2(x - a)} = 0\)得\(f'(x) = g'(x)\);由\(0 = \displaystyle\lim_{x \to a}\dfrac{f'(x) - g'(x)}{x - a} = \lim_{x \to a}(\dfrac{f'(x) - f'(a)}{x - a} - \dfrac{g'(x) - g'(a)}{x - a}) = f''(a) - g''(a)\)得\(f''(a) = g''(a)\)。故可推出相切且曲率相等。
但\(|f''(a)| = |g''(a)|\)不能推出原式。


\subsubsection{平均值}
f(x)在\([0, \dfrac{3\pi}{2}]\)上连续,在\((0, \dfrac{3\pi}{2})\)上是函数\(\dfrac{\cos x}{2x - 3\pi}\)的一个原函数,且\(f(0) = 0\)\begin{enumerate}
    \item 求\(f(x)\)在\([0, \dfrac{3\pi}{2}]\)上的平均值
    \item 证明f(x)在区间\((0, \dfrac{3\pi}{2})\)内存在唯一零点
\end{enumerate}

\paragraph{解1}
不妨设\(f(x) = \displaystyle\int_0^x\dfrac{\cos t}{2t - 3\pi}dt + C\),由\(f(0) = 0\),得C=0,则平均值
\[A = \dfrac{\displaystyle\int_0^{\frac{3\pi}{2}}f(x)dx}{\dfrac{3\pi}{2}} = \dfrac{\displaystyle\int_0^{\frac{3\pi}{2}}(\int_0^x\dfrac{\cos t}{2t - 3\pi}dt)dx}{\dfrac{3\pi}{2}}\]
\subparagraph{法1}
分部积分,\begin{align*}
    \int_0^{\frac{3\pi}{2}}f(x)dx & = \dfrac{1}{2}\int_0^{\frac{3\pi}{2}}f(x)d(2x - 3\pi) \\
    & = \dfrac{1}{2}f(x)(2x - 3\pi)\bigg|_0^{\frac{3\pi}{2}} - \dfrac{1}{2}\int_0^{\frac{3\pi}{2}}(2x - 3\pi)d[f(x)] \\
    & = - \dfrac{1}{2}\int_0^{\frac{3\pi}{2}}(2x - 3\pi)f'(x)dx = - \dfrac{1}{2}\int_0^{\frac{3\pi}{2}}\cos xdx \\
    & = \dfrac{1}{2}
\end{align*}
故\(A = \dfrac{1}{3\pi}\)
\subparagraph{法2}
二重积分,交换\(x,t\)积分次序。

\paragraph{证明2}
\(f'(x) = \dfrac{\cos x}{2x - 3\pi}\),因\(f(0) = 0\),故\((0, \dfrac{\pi}{2})\)单调减少,\((\dfrac{\pi}{2}, \dfrac{3\pi}{2})\)单调增。若\(f(\dfrac{3\pi}{2}) > 0\)则唯一零点,故问题转化为证明\(f(\dfrac{3\pi}{2}) > 0\)
\subparagraph{法1}
由问题1得\[\int_0^{\frac{3\pi}{2}}f(x)dx = A * \dfrac{3\pi}{2} = \dfrac{1}{2} > 0\]
若\(f(\dfrac{3\pi}{2}) <= 0\),则\(\displaystyle\int_0^{\frac{3\pi}{2}}f(x)dx < 0\),矛盾,故\(f(\dfrac{3\pi}{2}) > 0\)
\subparagraph{法2}
换元估计。


\section{物理应用}

\subsubsection{对水作功}
一容器是由\(y = x^2,\ (0 <= x <= 2)\)绕y轴旋转而成,若容器内水量是容量的1/4,水密度为\(\rho\),则将容器中水全部抽出需作的功为?
\subparagraph{解}
\(V_{\text{容}} = \pi\displaystyle\int_0^4(\sqrt{y})^2dy = 8\pi\),容器内水面水量\(V(h) = \pi\displaystyle\int_0^hydy = \dfrac{\pi h}{2} = \dfrac{1}{4} * 8\pi\),故水面高度\(h = 2\)。
在高度y处取一层水厚度为dy,该层水体积\(dV = \pi r^2dy = \pi ydy\),质量\(dm = \rho dV = \rho\pi ydy\),该层水到出口高度为\(4 - y\),功的微元为\(dW = dm * g * (4 - y)\),故总功为\[W = \rho\pi g\int_0^2y(4 - y)dy = \dfrac{16}{3}\pi\rho g\]


\subsubsection{面积关于时间的变化率}
已知曲线\(L:y = \dfrac{4}{9}x^2\,(x >= 0)\),点O\((0, 0)\),点A\((0, 1)\),设P是L上动点,S是OA与AP与L所围图形的面积。若P移动到点\((3, 4)\)时,沿x轴正向的速度是4,求此时S关于时间t的变化率

\subparagraph{解}
t时刻,P坐标为\((x(t), \dfrac{4}{9}x^2(t))\),则
\[S(t) = \dfrac{1}{2}[1 + \dfrac{4}{9}x^2(t)]x(t) - \int_0^{x(t)}\dfrac{4}{9}u^2du = \dfrac{x(t)}{2} + \dfrac{2}{27}x^3(t)\]
\[S'(t) = \dfrac{1}{2}x'(t) + \dfrac{2}{9}x^2(t)x'(t)\]
由题可知\(x(t) = 3\)时,\(x'(t) = 4\),代入得\(S'(t)\bigg|_{x = 3} = 10\)


\subsubsection{温度变化率}
已知高温物体置低温物体中,任一时刻该物体温度对时间的变化率与该物体和介质的温差成正比,现将一初始温度为120的物体在20的恒温介质中冷却,30min后物体降至30,若要将物体温度继续降至21,还需冷却多长时间

\textbf{解:}
设t时刻温度\(x(t)\),比例常数k(>0),介质温度为m,则\[\dfrac{dx}{dt} = -k(x - m)\]
故\(x(t) = Ce^{-kt} + m\)。\(\because x(0) = 120, m = 20, \ \therefore C = 100, \ \therefore x(t) = 100e^{-kt} + 20\)。 \\
又\(\because x(\dfrac{1}{2}) = 30, \therefore k = 2\ln10\),当\(x(t) = 21\)时,\(t= 1\),故还需30min。


\section{微分方程}

\subsubsection{根据特解求方程}
已知\(y_1 = xe^x + e^{2x}, y_2 = xe^x + e^{-x}, y_3 = xe^x + e^{2x} - e^{-x}\)是某二阶线性非齐次微分方程的三个解,则此微分方程为:?
\paragraph{解}
\(y_1 - y_3 = e^{-x},\ y_1 - y_2 = e^{2x} - e^{-x}\)是对应齐次方程的解,\((y_1 - y_3) + (y_1 - y_2) = e^{2x}\)是对应齐次方程的解,\(e^{-x}, e^{2x}\)是对应齐次方程两个线性无关的特解,\(y_2 - e^{-x} = xe^x\)是非齐次方程的解。

\subparagraph{方法1}
由“\(e^{-x}, e^{2x}\)是对应齐次方程两个线性无关的特解”得\(\lambda_1 = -1, \lambda_2 = 2\)是特征方程的两个根,故特征方程\((\lambda + 1)(\lambda - 2) = 0\),对应齐次微分方程为\[y'' - y' - 2y = 0\]
设非齐次方程为\(y'' - y' - 2y = f(x)\),非齐次解\(xe^x\)代入得\(f(x) = (1 - 2x)e^x\)

\subparagraph{方法2}
由非齐次特解\(xe^x\)及对应齐次方程两个线性无关解得非齐次方程通解为\[y = C_1e^{-x} + C_2e^{2x} + xe^x\]
求得\[y' = -C_1e^{-x} + 2C_2e^{2x} + (x + 1)e^x\]
\[y'' = C_1e^{-x} + 4C_2e^{2x} + (x + 2)e^x\]
消去\(C_1, C_2\)\[y'' - y' = 2(C_1e^{-x} + C_2e^{2x} + xe^x) - 2xe^x + e^x = 2y + (1 - 2x)e^x\]
故方程为\(y'' - y' - 2y = (1 - 2x)e^x\)


\subsubsection{根据特解求方程}
已知\(y_1 = \cos2x - \dfrac{1}{4}x\cos2x,\ y_2 = \sin2x - \dfrac{1}{4}x\cos2x\)是某二阶线性常系数非齐次微分方程得两个解,\(y_3 = \cos2x\)是它对应得齐次方程得一个解,则该微分方程是?

\paragraph{解}
\(y_1 - y_2 = \cos2x - \sin2x\)是对应齐次方程的一个解,故\(\cos2x - (\cos2x - \sin2x) = \sin2x\)也是对应齐次方程的解。根据两个线性无关解,的特征根为\(\pm2i\),特征方程为\(\lambda^2 + 4 = 0\),原方程为\(y'' + 4y = f(x)\)。由叠加原理得非齐次解\(-\dfrac{x}{4}\cos2x\),代入得\(f(x) = y'' + 4y = \sin2x\)


\subsubsection{高阶求特解}
方程\(y''' - y' = 0\)满足条件\(y\bigg|_{x = 0} = 3, y'\bigg|_{x = 0} = -1, y''\bigg|_{x = 0} = 1\)的特解为?
\paragraph{解}
特征方程为\[r^3 - r = 0\]即\(r(r^2 - 1) = 0\),得\(r_1 = 0, r_2 = 1, r_3 = -1\),故通解为\(y = C_1 + C_2e^x + C_3e^{-x}\),条件代入得\(y = 2 + e^{-x}\)


\section{多元函数微分}

\subsubsection{隐函数求偏导数全微分}
若函数\(z = z(x, y)\)由方程\(e^{x + 2y + 3z} + xyz = 1\)确定,则\(dz\bigg|_{(0, 0)} = \)?
\paragraph{解}
x = 0, y = 0代入得z = 0
\subparagraph{方法1}
原式两端微分得\[e^{x + 2y + 3z}(dx + 2dy + 3dz) + yzdx + xzdy + zydz = 0\]
代入得\(dx + 2dy + 3dz = 0\)
\subparagraph{方法2}
隐函数求导公式得\[\dfrac{\vartheta z}{\vartheta x} = -\dfrac{e^{x + 2y + 3z} + yz}{3e^{x + 2y + 3z} + xy}\]
\[\dfrac{\vartheta z}{\vartheta y} = -\dfrac{2e^{x + 2y + 3z} + xz}{3e^{x + 2y + 3z} + xy}\]
代入得
\subparagraph{方法3}
将\(y = 0\)代入原式得\(e^{x + 3z} = 1\),两端对x求导得\[e^{x + 3z}(1 + 3z_x') = 0\]
代入得


\subsubsection{二元最值}
二元函数\(f(x, y) = x^2(2 + y^2) + y\ln y\)的极小值为?
\subparagraph{解}
\(f_x' = 2x(2 + y^2),\ f_y' = 2x^2y + \ln y + 1\)
令\(\begin{cases}
    f_x' = 0 \\ 
    f_y' = 0
\end{cases}\),解得驻点\((0, \dfrac{1}{e})\)
\begin{flalign}
    A & = f_{xx}''(x_0, y_0) \nonumber \\ 
    B & = f_{xy}''(x_0, y_0) \nonumber \\ 
    C & = f_{yy}''(x_0, y_0) \nonumber
\end{flalign}
故\(AC - B^2 > 0,\ A > 0\),\(\therefore f(0, \dfrac{1}{e})\)是极小值为\(-\dfrac{1}{e}\)


\subsubsection{导数定义求偏导}
设\(z = (y^x + \dfrac{\sin x}{\sqrt{x^2 + 2y^2}})^{\sqrt{x^2 + y^2}}\),则\(\dfrac{\vartheta z}{\vartheta x}\bigg|_{(0, 1)} = \)?
\subparagraph{解}
\(y = 1\)代入得\(z(x, 1) = (1 + \dfrac{\sin x}{\sqrt{x^2 + 2}})^{\sqrt{x^2 + 1}}\),设\(z(x, 1) = \varphi(x)\),故\[\dfrac{\vartheta z}{\vartheta x}\bigg|_{(0, 1)} = \varphi'(0) = \lim_{x \to 0}\dfrac{\varphi(x) - \varphi(0)}{x}\]


\subsubsection{隐函数全微分}
设z是方程\(x + y + z = \displaystyle\int_0^{xyz}e^{-t^2}dt\)确定的隐函数,则\(dz = \)?
\subparagraph{解}
两端一阶全微分得\[dx + dy + dz = e^{-x^2y^2z^2}d(xyz) = e^{-x^2y^2z^2}(yzdx + xzdy + xydz)\]


\subsubsection{极限、线性近似微分}
设连续函数\(z = f(x, y)\)满足\(\displaystyle\lim_{x \to 0, y \to 1}\dfrac{f(x, y) - 2x + y - 2}{\sqrt{x^2 + (y - 1)^2}} = 0\),则\(dz\bigg|_{(0, 1)} = \)?
\subparagraph{解}
由题可知\[\displaystyle\lim_{x \to 0, y \to 1}[f(x, y) - 2x + y - 2] = 0\]
由\(f(x, y)\)连续,则\[f(0, 1) = 1\]
从而有\(\displaystyle\lim_{x \to 0, y \to 1}\dfrac{f(x, y) - f(0, 1) - 2x + (y - 1)}{\sqrt{x^2 + (y - 1)^2}} = 0\)
,即\[f(x, y) - f(0, 1) = 2x - (y - 1) + o(\rho)\]
故\[f_x'(0, 1) = 2,\ f_y'(0, 1) = -1\]


\section{矩阵}

\subsection{对角化}

\paragraph{对角化判断}
设A为3阶矩阵,已知\(|E + A| = 0, (3E - A)x = 0\text{有非零解}, E - 3A\)不可逆,问A是否相似于对角矩阵,说明理由
\subparagraph{解}
由题可知:\(|E + A| = 0, |3E - A| = 0, |E - 3A| = 0\),得A得三个特征值:\(-1, 3, \dfrac{1}{3}\),故有三个线性无关特征向量,故A相似于对角矩阵

\paragraph{特征向量定义,对角化判断}
设A为2阶矩阵且\(A^2 - A = 2E, P = [\alpha, A\alpha]\),其中\(\alpha\)是非零向量且不是A得特征向量;(1)证明\(|P| \neq 0\);(2)求\(P^{-1}AP\),判断A是否相似于对角矩阵

\subparagraph{证明1}
若\(|P| = 0\),即P为不可逆矩阵,则\(\alpha, A\alpha\)线性相关,因为\(\alpha \neq 0\),所以\(\exists \lambda_0\),使得\(A\alpha = \lambda_0\alpha\),这与\(\alpha\)不是A得特征向量矛盾,故P可逆,\(|P| = 0\)

\subparagraph{解2}
\(\because A^2 - A - 2E = 0\),即\(A^2\alpha - A\alpha - 2\alpha = 0, A^2\alpha = 2\alpha + A\alpha, \)\[\therefore AP = [A\alpha, A^2\alpha] = [A\alpha, 2\alpha + A\alpha] = [\alpha, A\alpha]\begin{bmatrix}
    0 & 2 \\ 
    1 & 1
\end{bmatrix} = P\begin{bmatrix}
    0 & 2 \\ 
    1 & 1
\end{bmatrix}\]
\[\therefore P^{-1}AP = \begin{bmatrix}
    0 & 2 \\ 
    1 & 1
\end{bmatrix}\]
\[\therefore |\lambda E - A| = \begin{vmatrix}
    \lambda & -2 \\ 
    -1 & \lambda - 1
\end{vmatrix}\]
得A得特征值为\(2, -1\),故A相似于对角矩阵\(\begin{bmatrix}
    2 & 0 \\ 
    0 & -1
\end{bmatrix}\)

\paragraph{特征值特征向量求矩阵}
设A是3阶矩阵,已知\(A\xi_i = i\xi_i, (i = 1, 2, 3)\),其中\(\xi_1 = [1, 0, 0]^T, \xi_2 = [1, 1, 0]^T, \xi_3 = [1, 1, 1]^T\),则矩阵A = ?
\subparagraph{解法1}
由题可知,A有3个互不相同的特征值,故A相似于对角矩阵,且\(\xi_1, \xi_2, \xi_3\)是3个线性无关的特征向量,故存在可逆矩阵\(P = [\xi_1, \xi_2, \xi_3]\)使得\(P^{-1}AP = \begin{bmatrix}
    1 & & \\ 
    & 2 & \\ 
    & & 3
\end{bmatrix}\),故\(A = P\begin{bmatrix}
    1 & & \\ 
    & 2 & \\ 
    & & 3
\end{bmatrix}P^{-1}\)

\subparagraph{解法2}
由题得\[[A\xi_1, A\xi_2, A\xi_3] = A[\xi_1, \xi_2, \xi_3] = [\xi_1, 2\xi_2, 3\xi_3]\]
\[\therefore\, A = [\xi_1, 2\xi_2, 3\xi_3][\xi_1, \xi_2, \xi_3]^{-1}\]

\subsubsection{求特征向量}
已知\(P^{-1}AP = \begin{bmatrix}
    1 \\ 
     & 1 \\ 
     & & -1
\end{bmatrix}, P = (\alpha_1, \alpha_2, \alpha_3)\)可逆,则矩阵A关于特征值\(\lambda = 1\)的特征向量是?
\subparagraph{解}
当\(P^{-1}AP = \Lambda\)时,P的每一列都是A的相应的特征向量,故\(\lambda = 1\)的特征向量为\(k_1\alpha + k_2\alpha,\ \ k_1, k_2\)不全为0。


\subsubsection{求特征向量}
已知\(P^{-1}AP = B,\ B =  \begin{bmatrix}
    1 & -1 & 2 \\ 
    2 & -2 & 4 \\ 
    1 & -1 & 2
\end{bmatrix},\ P = (\alpha_1, \alpha_2, \alpha_3)\),则矩阵A关于特征值\(\lambda = 0\)的特征向量为?
\subparagraph{解}
\(0E - B = \begin{bmatrix}
    1 & -1 & 2 \\ 
    0 & 0 & 0 \\ 
    0 & 0 & 0
\end{bmatrix}\)得基础解系\[\beta_1 = (1, 1, 0)^T,\ \beta_2 = (-2, 0, 1)^T\]故特征向量为\(k_1P\beta_1 + k_2P\beta_2\)


\subsubsection{相似对角求矩阵}
已知矩阵\(A = \begin{bmatrix}
    3 & 1 & 2 \\ 
    0 & 2 & a \\ 
    0 & 0 & 3
\end{bmatrix}\)和对角矩阵相似,则\(a=\)?
\subparagraph{解}
\begin{flalign}
    A \sim \Lambda & \Leftrightarrow \lambda = 3\text{有两个线性无关特征向量} \nonumber \\ 
    & \Leftrightarrow (3E - A)x = 0\text{有两个线性无关解} \nonumber \\ 
    & \Leftrightarrow r(3E - A) = 1 \nonumber
\end{flalign}
故\(a = -2\)


\section{线性方程组}

\subsection{齐次}

\subsubsection{通解}
设\(A = \begin{bmatrix}
    1 & -2 & 0 \\ 
    2 & 1 & 5 \\ 
    0 & 1 & 1
\end{bmatrix}\),B是三阶矩阵,则满足AB = O的所有B=?
\subparagraph{解}
\(B = \begin{bmatrix}
    2k & 2l & 2\lambda \\ 
    k & l & \lambda \\ 
    -k & -l & -\lambda
\end{bmatrix}\)

\subsection{非齐次}

\subsubsection{例1}
设\(r(A_{4 * 4}) = 2, \eta_1, \eta_2, \eta_3\)是\(Ax = b\)的3个解向量,其中\(\begin{cases}
\eta_1 - \eta_2 = \alpha_1 \\ 
\eta_1 + \eta_2 = \alpha_2 \\ 
\eta_3 + 2\eta_2 = \alpha_3
\end{cases}\),求\(Ax = b\)通解

\subparagraph{解}
\(Ax = \beta\)通解结构为\[k_1\xi_1 + k_2\xi_2 + \eta\],因为\(A(\eta_1 - \eta_2) = b - b = 0, A[3(\eta_1 + \eta_2) - 2(\eta_3 + 2\eta_2)] = 6b - 6b = 0\),故\(\eta_1 - \eta_2, 3(\eta_1 + \eta_2) - 2(\eta_3 + 2\eta_2)\)是\(Ax = 0\)的解向量,又\(A[\dfrac{1}{2}(\eta_1 + \eta_2)] = \dfrac{1}{2}(b + b) = b\),因此\(\dfrac{1}{2}(\eta_1 + \eta_2)\)是\(Ax = b\)的一个特解,因此\(Ax = b\)通解为\[k_1\alpha_1 + k_2(3\alpha_2 - 2\alpha_3) + \dfrac{1}{2}\alpha_2\]

\subsubsection{例2}
由\([\eta_1 - \eta_2, \eta_1 + \eta_2, \eta_3 + 2\eta_2] = [\eta_1, \eta_2, \eta_3]\begin{bmatrix}
1 & 1 & 0 \\ 
-1 & 1 & 2 \\ 
0 & 0 & 1
\end{bmatrix}\),故\[[\eta_1, \eta_2, \eta_3] = [\eta_1 - \eta_2, \eta_1 + \eta_2, \eta_3 + 2\eta_2]\begin{bmatrix}
1 & 1 & 0 \\ 
-1 & 1 & 2 \\ 
0 & 0 & 1
\end{bmatrix}^{-1}\],求出\(\eta_1, \eta_2, \eta_3\),的通解\[k_1(\eta_1 - \eta_2) + k_2(\eta_2 - \eta_3) + \eta_3\]


\subsubsection{解性质}
设\(A_{3 * 3}x = b\),即\(\begin{cases}
    a_{11}x_1 + ... = b_1 \\ 
    ... \\ 
    ... \\ 
\end{cases}\)有唯一解\(\xi = [1, 2, 3]^T\)。

方程组\(B_{3 * 4}x = b\),即\(\begin{cases}
    a_{11}x_1 + ... + a_{14}x_4 = b_1 \\ 
    ... \\ 
    ... \\ 
\end{cases}\)有特解\(\eta = [-2, 1, 4, 2]^T\),则\(B_{3 * 4}x = b\)的通解为?

\subparagraph{解}
\(r(A) = r(A, b) = 3,\ \therefore\ r(B) = r(B, b) = 3, \eta_1 = [1, 2, 3, 0]^T\)是\(B_{3 * 4}x = b\)的另一特解。故Bx = 0基础解系仅一个向量,为\(\eta - \eta_1\),故通解为\(k(\eta - \eta_1) + \eta\)


\section{特征值特征向量}

\subsubsection{行元素之和求代数余子式和}
设\(A = (a_{ij})\)为3阶矩阵,\(A_{ij}\)为\(a_{ij}\)的代数余子式,若A每行元素之和均为2,且\(|A| = 3\),则\(A_{11} + A_{21} + A_{31} = ?\)

\subparagraph{解}
由\(A\begin{bmatrix}
    1 \\ 
    1 \\ 
    1
\end{bmatrix} = 2\begin{bmatrix}
    1 \\ 
    1 \\ 
    1
\end{bmatrix}\)得A的一个特征值\(\lambda = 2\),其对应特征向量为\(\alpha\begin{bmatrix}
    1 \\ 
    1 \\ 
    1
\end{bmatrix}\),则\(A^*\)的一个特征值为\(\dfrac{|A|}{\lambda}\),其对应特征向量也为\(\alpha\),由\(A^*\alpha = \dfrac{|A|}{\lambda}\alpha, A^* = \begin{bmatrix}
    A_{11} & A_{21} & A_{31} \\
    ... \\
    ...
\end{bmatrix}\)得\(A^*\begin{bmatrix}
    1 \\ 
    1 \\ 
    1
\end{bmatrix} = \begin{bmatrix}
    A_{11} + A_{21} + A_{31} \\ 
    ... \\ 
    ...
\end{bmatrix} = \dfrac{|A|}{\lambda}\begin{bmatrix}
    1 \\ 
    1 \\ 
    1
\end{bmatrix}\),即\(A_{11} + A_{21} + A_{31} = \dfrac{3}{2}\)

\subsubsection{特征向量求特征值}
已知\(\alpha = [1, k, 1]^T\)是\(A^{-1}\)的特征向量,其中\(A = \begin{bmatrix}
    2 & 1 & 1 \\ 
    1 & 2 & 1 \\ 
    1 & 1 & 2
\end{bmatrix}\),求k及\(\alpha\)对应的\(A^{-1}\)的特征值

\subparagraph{解}
由题可知\(A^{-1}\alpha = \lambda\alpha\),\(\lambda\)是\(A^{-1}\)对应\(\alpha\)的特征值,两端左乘A得\(\alpha = \lambda A\alpha\)
\(\because \alpha \neq 0,\ \therefore \lambda \neq 0, A\alpha = \dfrac{1}{\lambda}\alpha\),记\(\mu = \dfrac{1}{\lambda}\),则
\[A\begin{bmatrix}
    1 \\ 
    k \\ 
    1
\end{bmatrix} = \mu\begin{bmatrix}
    1 \\ 
    k \\ 
    1
\end{bmatrix}\]
则\(\begin{cases}
    3 + k = \mu \\ 
    2 + 2k = k\mu \\ 
    3 + k = \mu
\end{cases}\),得\(k = 1, -2\),\(\lambda = \dfrac{1}{\mu} = \dfrac{1}{3 + k}\)

\subsubsection{n阶矩阵}
设\(A = \begin{bmatrix}
    1 & -2 & 0 & 0 \\ 
    -1 & 0 & 0 & 0 \\ 
    0 & 0 & 2 & 1 \\ 
    0 & 0 & 0 & 2
\end{bmatrix}\),求\(A^n, n >= 2\)

\subparagraph{解}
记\(B = \begin{bmatrix}
    1 & -2 \\ 
    -1 & 0
\end{bmatrix}, C = \begin{bmatrix}
    2 & 1 \\ 
    0 & 2
\end{bmatrix}\),则\(A = \begin{bmatrix}
    B & O \\ 
    O & C
\end{bmatrix}, A^n = \begin{bmatrix}
    B^n & O \\ 
    O & C^n
\end{bmatrix}\),

对B,由\(|\lambda E - B| = (\lambda + 1)(\lambda - 2) = 0\),得特征值\(\lambda_1 = -1, \lambda_2 = 2,\ B^n = P^{-1}\begin{bmatrix}
    -1 & 0 \\ 
    0 & 2
\end{bmatrix}^nP,\ P = \begin{bmatrix}
    1 & -2 \\ 
    1 & 1
\end{bmatrix}\)

对C,当\(n >= 2\)时,\(\begin{bmatrix}
    0 & 1 \\ 
    0 & 0
\end{bmatrix}^n = O\),故\(C^n = (2E + \begin{bmatrix}
    0 & 1 \\ 
    0 & 0
\end{bmatrix})^n = 2^nE^n + n * 2^{n - 1}E^{n - 1}\begin{bmatrix}
    0 & 1 \\ 
    0 & 0
\end{bmatrix} = \begin{bmatrix}
    2^n & n * 2^{n - 1} \\ 
    0 & 2^n
\end{bmatrix}\)

\subsubsection{特征值和特征向量求矩阵}
已知1,1,-1是实对称矩阵A的特征值,向量\(\xi_1 = [1, 1, 1]^T,\ \xi_2 = [2, 2, 1]^T\)是对应于\(\lambda_1 = \lambda_2 = 1\)的特征向量,求A
\subparagraph{解}
设\(\lambda_3 = -1\)对应特征向量\(\xi_3 = [x_1, x_2, x_3]^T,\ \because A\)为实对称矩阵,有\[\begin{cases}
    (\xi_1, \xi_3) = x_1 + x_2 + x_3 = 0 \\ 
    (\xi_2, \xi_3) = 2x_1 + 2x_2 + x_3 = 0
\end{cases}\]
取\(\xi_3 = [-1, 1, 0]^T\),得\(P = [\xi_1, \xi_2, \xi_3],\ \therefore A = P\Lambda P^{-1}\)

\subsubsection{特征值和一个特征向量求矩阵}
已知二次型\(f = x^TAx\)在正交变换\(x = Qy\)下标准形为\(y_1^2 + y_2^2\)且Q第3列为\([\dfrac{1}{\sqrt{2}}, 0, \dfrac{1}{\sqrt{2}}]^T\)(1)求矩阵A(2)证明\(A + E\)为正定矩阵

\subparagraph{解1}
由题可知特征值1,1,0,设\([x_1, x_2, x_3]^T\)是对应特征值1的特征向量,由不同特征值的特征向量正交得\([\dfrac{1}{\sqrt{2}}, 0, \dfrac{1}{\sqrt{2}}]\begin{bmatrix}
    x_1 \\ 
    x_2 \\ 
    x_3
\end{bmatrix} = 0\),即\(x_1 + x_3 = 0\),得\(\xi_1 = []^T, \xi_2 = []^T\),故\(Q = [\xi_1, \xi_2, \xi_3], \therefore\ A = Q\begin{bmatrix}
    1 \\ 
    & 1 \\ 
    & & 0
\end{bmatrix}Q^T\)

\subparagraph{证明2}


\subsubsection{实对称矩阵求特征向量}
已知A三阶实对称矩阵,特征值为1,3,-2,其中\(\alpha_1 = (1, 2, -2)^T, \alpha_2 = (4, -1, a)^T\)分别属于\(\lambda = 1, \lambda = 3\)的特征向量,则属于\(\lambda = 2\)的特征向量为?
\subparagraph{解}
\(\because\)实对称矩阵,因此特征向量正交,\(\alpha_1^T\alpha_2 = 0\)得\(a = 1\),由\(\begin{cases}
    x_1 + 2x_2 - 2x_3 = 0 \\ 
    4x_1 - x_2 + x_3 = 0
\end{cases}\)得基础解系\((0, 1, 1)^T,\ \therefore\ \alpha_3 = (0, k, k)^T\)


\section{二次型}

\subsection{计算}

\subsubsection{正交变换化为标准形}
设二次型f在正交变换\(x = Py\)下的标准形为\(2y_1^2 + y_2^2 - y_3^2\),其中\(P = [e_1, e_2, e_3]\),若\(Q = [e_1, -e_3, e_2]\)则二次型f在\(x = Qy\)下的标准形为?(\(2y_1^2 - y_2^2 + y_3^2\))

\subparagraph{解}
\(e_1, e_2, e_3\)分别是A对应于特征值2, 1, -1的特征向量,于是\(-e_3\)对应特征值-1的特征向量,故


\subsubsection{正交变换化为二次型}
设二次型\(f(x_1, x_2) = x_1^2 - 4x_1x_2 + ax_2^2\)经过正交变换\(\begin{bmatrix}
    x_1 \\ 
    x_2
\end{bmatrix} = Q\begin{bmatrix}
    y_1 \\ 
    y_2
\end{bmatrix}\)化为二次型\(g(y_1, y_2) = 4y_1^2 + 4y_1y_2 + by_2^2\)
\begin{enumerate}
    \item 求a,b值
    \item 求正交矩阵Q
\end{enumerate}

\subparagraph{解1}
由题可知,二次型f,g的矩阵分别为\(A = \begin{bmatrix}
    1 & -2 \\ 
    -2 & a
\end{bmatrix}, B = \begin{bmatrix}
    4 & 2 \\ 
    2 & b
\end{bmatrix}\),且\(Q^TAQ = B\)由于Q为正交矩阵,于是有\(Q^{-1}AQ = B\),因此\(tr(A) = tr(B), |A| = |B|\),得\(a = 4, b = 1\)

\subparagraph{解2}
由\(|\lambda E - A| = |\lambda E - B| = \lambda(\lambda - 5)\)得特征值为\(0, 5\)

矩阵A对应特征值\(\lambda_1 = 0\)得特征向量\(a_1\),对应特征值\(\lambda_2 = 5\)得特征向量\(a_2\)。令\(Q_1 = [a_1, a_2]\),则\(Q_1\)为正交矩阵

矩阵B对应特征值\(\lambda_1 = 0\)得特征向量\(b_1\),对应特征值\(\lambda_2 = 5\)得特征向量\(b_2\)。令\(Q_2 = [b_1, b_2]\),则\(Q_2\)为正交矩阵

由\(Q_1^TAQ_1 = Q_2^TBQ_2, \therefore\ Q = Q_1Q_2^T\)


\subsubsection{惯性指数}
设二次型\(f(x_1, x_2, x_3) = x_1^2 - x_2^2 + 2ax_1x_3 + 4x_2x_3\)的负惯性指数为1,则a的取值范围为?

\subparagraph{解}
\(f = x_1^2 + 2ax_1x_3 + a^2x_3^2 - x_2^2 + 4x_2x_3 - 4x_3^2 + 4x_3^2 - a^2x_3^2 = (x_1 + ax_3)^2 - (x_2 - 2x_3)^2 + (4 - a^2)x_3^2\)

\(\because\ 4 - a^2 >= 0, \therefore\ -2 <= a <= 2\)


\subsubsection{二次型概念,规范形}
设实二次型\(f(x_1, x_2, x_3) = (x_1 - x_2 + x_3)^2 + (x_2 + x_3)^2 + (x_1 + ax_3)^2\),其中\(a\)是参数。
\begin{enumerate}
    \item 求\(f(x_1, x_2, x_3) = 0\)的解
    \item 求\(f(x_1, x_2, x_3) = 0\)的规范形
\end{enumerate}

\subparagraph{解1}
由\(f(x_1, x_2, x_3) = 0\)得\(\begin{cases}
    x_1 - x_2 + x_3 = 0 \\ 
    x_2 + x_3 = 0 \\
    x_1 + ax_3 = 0
\end{cases}\),其系数矩阵\(A = \begin{bmatrix}
    1 & -1 & 1 \\ 
    0 & 1 & 1 \\
    1 & 0 & a
\end{bmatrix} \rightarrow \begin{bmatrix}
    1 & 0 & 2 \\
    0 & 1 & 1 \\
    0 & 0 & a - 2
\end{bmatrix}\),得\begin{itemize}
    \item 当\(a \neq 2\)时,\(r(A) = 3\),只有零解
    \item 当\(a = 2\)时,\(r(A) = 2\),有无穷解,通解\(x = k(2, 1, -1)^T\)
\end{itemize}

\subparagraph{解2}
\begin{itemize}
    \item 当\(a \neq 2\)时,A可逆,规范形\(f = y_1^2 + y_2^2 + y_3^2\)
    \item 当\(a = 2\)时,\(r(A) = 2\),解\(|\lambda E - A|\),得特征值\(\lambda_1 = \lambda_2 = 2\),\(\lambda_3 = 0\)。故规范形\(f = y_1^2 + y_2^2\)
\end{itemize}


\subsection{正定}

\subsubsection{判别正定性}
判别\(f(x_1, x_2, x_3) = 2x_1^2 + 2x_2^2 + 2x_3^2 + 2x_1x_2 + 2x_1x_3 + 2x_2x_3\)的正定性

\paragraph{方法一}
判断各阶顺序主子式\[2 > 0, \begin{vmatrix}
    2 & 1 \\ 
    1 & 2
\end{vmatrix} = 3 > 0, |A| = 4 > 0\]

\paragraph{方法二}
判断特征值是否全部大于0

\paragraph{方法三}
配方法化为标准形,判断正惯性指数p是否等于n

\paragraph{方法四}
定义验证是否对任意\(x = [x_1, x_2, x_3]^T \neq 0\)有\(x^TAx > 0\)\[f = (x_1 + x_2)^2 + (x_1 + x_3)^2 + (x_2 + x_3)^2\]
故有\(f >= 0\),且\[f = 0 \Leftrightarrow \begin{cases}
    x_1 + x_2 = 0 \\ 
    x_1 + x_3 = 0 \\ 
    x_2 + x_3 = 0
\end{cases}\tag{*}\]
方程组(*)的系数行列式\( = 2 \neq 0\),故(*)只有零解,故\(x = [x_1, x_2, x_3]^T \neq 0\)时\(f > 0\)

\subparagraph{注}
对\(f = (...)^2 + (...)^2 + ...\)可直接计算系数行列式


\subsubsection{判断二次型矩阵}
设矩阵\[B = \begin{bmatrix}
    1 & 2 & 3 \\ 
    -2 & 0 & 1 \\ 
    0 & 4 & 5
\end{bmatrix}, x = \begin{bmatrix}
    x_1 \\ 
    x_2 \\ 
    x_3
\end{bmatrix}\]问\(f = x^TBx\)是否为关于\(x_1, x_2, x_3\)的二次型?B是否为f的矩阵?写出\(f\)的矩阵表达式

\(f\)是关于\(x_1, x_2, x_3\)的二次型,B不是f的矩阵
\paragraph{方法一}
由于
\begin{flalign}
    f & = \begin{bmatrix}
        x_1, & x_2, & x_3
    \end{bmatrix}B\begin{bmatrix}
        x_1 \\ 
        x_2 \\ 
        x_3
    \end{bmatrix} \nonumber \\ 
    & = \begin{bmatrix}
        x_1, & x_2, & x_3
    \end{bmatrix}\begin{bmatrix}
        x_1 + 2x_2 + 3x_3 \\ 
        -2x_1 + x_3 \\ 
        4x_2 + 5x_3
    \end{bmatrix} \nonumber \\ 
    & = x_1^2 + 5x_3^2 + 3x_1x_3 + 5x_2x_3 \nonumber
\end{flalign}
故矩阵为\[A = \begin{bmatrix}
    1 & 0 & \dfrac{3}{2} \\ 
    0 & 0 & \dfrac{5}{2} \\ 
    \dfrac{3}{2} & \dfrac{5}{2} & 5
\end{bmatrix}\]

\paragraph{方法二}
注意到\(x^TBx\)是\(1 * 1\)矩阵,故转置不变,故有
\begin{flalign}
    f & = x^TBx = (x^TBx)^T = \dfrac{1}{2}[x^TBx + (x^TBx)^T] \nonumber \\ 
    & = \dfrac{1}{2}(x^TBx + x^TB^Tx) = \dfrac{1}{2}x^T(B + B^T)x = x^T\dfrac{B + B^T}{2}x
\end{flalign}
故\[A = \dfrac{B + B^T}{2}\]


