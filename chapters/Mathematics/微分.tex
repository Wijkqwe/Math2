
\chapter{微分}


\section{一阶微分方程}

\begin{displaymath}
\dfrac{dy}{dx} = f(ax + by + c)
\end{displaymath}
令u = ax + by + c,\begin{flalign}
    \dfrac{du}{dx} & = a + b\dfrac{dy}{dx} \nonumber \\ 
    \dfrac{du}{dx} & = a + bf(u) \nonumber
\end{flalign}


\subsection{齐次微分方程}
\[\dfrac{dy}{dx} = f(\dfrac{y}{x})\]
令\(u = \dfrac{y}{x}\), \(\dfrac{dy}{dx} = u + x\dfrac{du}{dx}\), 即
\[x\dfrac{du}{dx} + u = f(u)\]


\subsection{可化为齐次方程}

\[\dfrac{dy}{dx} = \dfrac{ax + by + c}{a_1x + b_1y + c_1}\]

\begin{itemize}
    \item 若\(\dfrac{a_1}{a} \neq \dfrac{b_1}{b}\),令\[x = X + h, y = Y + k\]
    则\[dx = dX, dy = dY\]
    \[\dfrac{dY}{dX} = \dfrac{aX + bY + ah + bk + c}{a_1X + b_1Y + a_1h + b_1k + c_1}\]
    则可求出h,k使满足\(\begin{cases}
        ah + bk + c = 0 \\
        a_1h + b_1k + c_1 = 0
    \end{cases}\),上式化为
    \[\dfrac{dY}{dX} = \dfrac{aX + bY}{a_1X + b_1Y}\]
    将\(X = x - h, Y = y - k\)代入得到通解;
    
    \item 若\(\dfrac{a_1}{a} == \dfrac{b_1}{b}\),令\(\dfrac{a_1}{a} = \dfrac{b_1}{b} = \lambda\),则
    \[\dfrac{ax + by + c}{\lambda(ax + by) + c_1}\]
    令\(v = ax + by\),则\[\dfrac{1}{b}(\dfrac{dv}{dx} - a) = \dfrac{v + c}{\lambda v + c_1}\]
\end{itemize}


\subsection{一阶线性微分方程}
\[\dfrac{dy}{dx} + P(x)y = Q(x)\]
若\(Q(x) \equiv 0\), 则齐次, 若\(Q(x) \not\equiv 0\),则非齐次

\subsubsection{通解}
\begin{itemize}
    \item 齐次线性方程:\[y = Ce^{-\int_{}^{} P(x) \,dx}\]
    \item 非齐次线性方程:\[y = e^{-\int_{}^{} P(x) \,dx}(\int_{}^{}Q(x)e^{\int_{}^{}P(x) \,dx} \,dx + C)\]
\end{itemize}


\subsection{伯努利方程}
\[\dfrac{dy}{dx} + P(x)y = Q(x)y^{n}\space (n != 0, 1)\]
令\[z = y^{1 - n}\]
\[\dfrac{dz}{dx} = (1 - n)y^{-n}\dfrac{dy}{dx}\]
则上式化为\[\dfrac{dz}{dx} + (1 - n)P(x)z = (1 - n)Q(x)\]


\section{二阶可降解微分方程}

\subsection{\(y'' = f(x, y')\)}

令\(y' = p(x), y'' = p'(x)\),则原方程为\[\dfrac{dp}{dx} = f(x, p)\]
若通解为\(p = F(x, C_1)\),即\(y' = F(x, C_1)\),则积分。


\subsection{\(y'' = f(y, y')\)}

令\(y' = p(y), y'' = \dfrac{dp}{dx} = p(y)\dfrac{dp}{dy}\),则原方程为\[p\dfrac{dp}{dy} = f(y, p)\]
若通解为\(p = F(y, C_1)\),则\(\int\dfrac{dy}{F(y, C_1)} = x + C_2\)

\subsection{\(y'' = f(y')\)}

既不显含y,也不显含x,按\(y'' = f(x, y')\)处理


\section{高阶线性微分方程}

\subsection{特征方程}
对n阶常系数齐次线性微分方程\[a_ny^{(n)} + ... + a_1y' + a_0y = 0\]
将\(y^{(k)}\)替换为\(r^k\)得\textbf{特征方程}\[a_nr^n + ... + a_1r + a_0 = 0\]
解特征方程得根\(r_1, ..., r_n\);


\subsection{解的性质与结构}
\begin{itemize}
    \item 若解中含特解\(e^{rx}\),则r至少为单实根;
    \item 若解中含特解\(x^{k - 1}e^{rx}\),则r至少为k重实根;
    \item 若解中含特解\(e^{\alpha x}cos\beta x\)或\(e^{\alpha x}sin\beta x\),则\(\alpha \pm \beta i\)至少为单复根;
    \item 若解中含特解\(e^{\alpha x}xcos\beta x\)或\(e^{\alpha x}xsin\beta x\),则\(\alpha \pm \beta i\)至少为二重复根;
    
    \item 若\(y_1^*, y_2^*\)是非齐次方程的特解,则\(y_1^* - y_2^*\)是对应齐次方程的解;

    \item 若\(y_1^*\)是\(y'' + py' + qy = f_1(x)\)的解,\(y_2^*\)是\(y'' + py' + qy = f_2(x)\)的解,则\(y_1^* + y_2^*\)是\(y'' + py' + qy = f_1(x) + f_2(x)\)的解;

    \item 若\(y = C_1y_1(x) + C_2y_2(x)\)是\(y'' + py' + qy = 0\)的通解,\(y^*\)是\(y'' + py' + qy = f(x)\)的一个特解,则\(y + y^*\)是\(y'' + py' + qy = f(x)\)的特解;
\end{itemize}


\subsection{二阶常系数齐次线性微分方程}
\[y'' + py' + qy = 0\]

若\(y_1(x),y_2(x)\)是两个解,且\(\dfrac{y_1}{y_2} \neq C\)(C常数),则是线性无关解,通解为\[y(x) = C_1y_1(x) + C_2y_2(x)\]

\subsubsection{特征方程}
\(r^2 + pr + q = 0\)
\begin{itemize}
    \item \(p^2 - 4q > 0\),则\(r_1 != r_2\),通解\[y = C_1e^{r_1x} + C_2e^{r_2x}\]
    \item \(p^2 - 4q = 0\),则\(r_1 = r_2 = r\),通解\[y = (C_1 + C_2)e^{rx}\]
    \item \(p^2 - 4q < 0\),则共轭复根\(\alpha \pm \beta i\),通解\[y = (C_1cos\beta x + C_2sin\beta x)e^{\alpha x}\]
\end{itemize}


\subsection{二阶常系数非齐次线性微分方程}
\[y'' + py' + qy = f(x)\]


\paragraph{当\(f(x) = P_n(x)\)时}
特解为\[y^* = Q_n(x)x.;^k\]其中
\[\begin{cases}
    Q_n(x)\text{ 与P(x)同阶多项式,通常代入计算得到} \\
    k = \begin{cases}
        0, 0 \neq r_1,\ \ 0 \neq r_2 \\ 
        1, 0 = r_1\ \text{ or }\ 0 = r_2 \\ 
        2, 0 = r_1 = r_2
    \end{cases}
\end{cases}\]

\paragraph{当\(f(x) = P_n(x)e^{\alpha x}\)时}
特解设为
\[y^* = Q_n(x)x^ke^{\alpha x}\]其中
\[\begin{cases}
    e^{\alpha x}\text{ 照抄} \\
    Q_n(x)\text{ 与P(x)同阶多项式,通常代入计算得到} \\
    k = \begin{cases}
        0, \alpha \neq r_1,\ \ \alpha \neq r_2 \\ 
        1, \alpha = r_1\ \text{ or }\ \alpha = r_2 \\ 
        2, \alpha = r_1 = r_2
    \end{cases}
\end{cases}\]

\paragraph{当\(f(x) = [P_m(x)\cos\beta x + P_n(x)\sin\beta x]e^{\alpha x}\)时}
特解设为
\[y^* = [Q_l^1(x)cos\beta x + Q_l^2(x)sin\beta x]x^ke^{\alpha x}\]
其中
\[\begin{cases}
    e^{\alpha x}\text{ 照抄} \\ 
    l = max\{m, n\} \\ 
    k = \begin{cases}
        0, \alpha \pm \beta i \neq r_{1,2} \\ 
        1, \alpha \pm \beta i = r_{1,2}
    \end{cases}
\end{cases}\]


\section{n阶常系数齐次线性微分方程}
\[a_1r^n + a_2r^{n - 1} + ... + a_n = 0\]

根据特征方程得到特征根的值,根据特征根得到通解。


\section{欧拉方程(数一)}
\[x^2y'' + pxy' + qy = f(x)\]

当\(x > 0\)时,令\(x = e^t\),则\(t = lnx, \dfrac{dt}{dx} = \dfrac{1}{x}\),于是
\[\dfrac{dy}{dx} = \dfrac{dy}{dt} \dfrac{dt}{dx} = \dfrac{1}{x} \dfrac{dy}{dt}\]
\[\dfrac{d^2y}{dx^2} = \dfrac{d}{dx}(\dfrac{1}{x}\dfrac{dy}{dt}) = -\dfrac{1}{x^2}\dfrac{dy}{dt} + \dfrac{1}{x}\dfrac{d}{dx}(\dfrac{dy}{dt}) = -\dfrac{1}{x^2}\dfrac{dy}{dt} + \dfrac{1}{x^2}\dfrac{d^2y}{dt^2}\]
方程化为\[\dfrac{d^2y}{dt^2} + (p - 1)\dfrac{dy}{dt} + qy = f(e^t)\]

当\(x < 0\)时,令\(x = -e^t\),同理


\section{例}

\subsubsection{根据特解求方程}
已知\(y_1 = xe^x + e^{2x}, y_2 = xe^x + e^{-x}, y_3 = xe^x + e^{2x} - e^{-x}\)是某二阶线性非齐次微分方程的三个解,则此微分方程为:?
\paragraph{解}
\(y_1 - y_3 = e^{-x},\ y_1 - y_2 = e^{2x} - e^{-x}\)是对应齐次方程的解,\((y_1 - y_3) + (y_1 - y_2) = e^{2x}\)是对应齐次方程的解,\(e^{-x}, e^{2x}\)是对应齐次方程两个线性无关的特解,\(y_2 - e^{-x} = xe^x\)是非齐次方程的解。

\subparagraph{方法1}
由“\(e^{-x}, e^{2x}\)是对应齐次方程两个线性无关的特解”得\(\lambda_1 = -1, \lambda_2 = 2\)是特征方程的两个根,故特征方程\((\lambda + 1)(\lambda - 2) = 0\),对应齐次微分方程为\[y'' - y' - 2y = 0\]
设非齐次方程为\(y'' - y' - 2y = f(x)\),非齐次解\(xe^x\)代入得\(f(x) = (1 - 2x)e^x\)

\subparagraph{方法2}
由非齐次特解\(xe^x\)及对应齐次方程两个线性无关解得非齐次方程通解为\[y = C_1e^{-x} + C_2e^{2x} + xe^x\]
求得\[y' = -C_1e^{-x} + 2C_2e^{2x} + (x + 1)e^x\]
\[y'' = C_1e^{-x} + 4C_2e^{2x} + (x + 2)e^x\]
消去\(C_1, C_2\)\[y'' - y' = 2(C_1e^{-x} + C_2e^{2x} + xe^x) - 2xe^x + e^x = 2y + (1 - 2x)e^x\]
故方程为\(y'' - y' - 2y = (1 - 2x)e^x\)


\subsubsection{根据特解求方程}
已知\(y_1 = \cos2x - \dfrac{1}{4}x\cos2x,\ y_2 = \sin2x - \dfrac{1}{4}x\cos2x\)是某二阶线性常系数非齐次微分方程得两个解,\(y_3 = \cos2x\)是它对应得齐次方程得一个解,则该微分方程是?

\paragraph{解}
\(y_1 - y_2 = \cos2x - \sin2x\)是对应齐次方程的一个解,故\(\cos2x - (\cos2x - \sin2x) = \sin2x\)也是对应齐次方程的解。根据两个线性无关解,的特征根为\(\pm2i\),特征方程为\(\lambda^2 + 4 = 0\),原方程为\(y'' + 4y = f(x)\)。由叠加原理得非齐次解\(-\dfrac{x}{4}\cos2x\),代入得\(f(x) = y'' + 4y = \sin2x\)


\subsubsection{高阶求特解}
方程\(y''' - y' = 0\)满足条件\(y\bigg|_{x = 0} = 3, y'\bigg|_{x = 0} = -1, y''\bigg|_{x = 0} = 1\)的特解为?
\paragraph{解}
特征方程为\[r^3 - r = 0\]即\(r(r^2 - 1) = 0\),得\(r_1 = 0, r_2 = 1, r_3 = -1\),故通解为\(y = C_1 + C_2e^x + C_3e^{-x}\),条件代入得\(y = 2 + e^{-x}\)


\section{一元函数微分学}

\subsection{定义}

函数\(y = f(x)\)在点\(x_0\)的某个邻域内有定义,且\(x_0 + \Delta x\)在该邻域内,对于函数增量\(\Delta y = f(x_0 + \Delta x) - f(x_0)\),若存在与\(\Delta x\)无关的常数A使得\(\Delta y = A\Delta x + o(\Delta x)\),则\(f(x)\)在点\(x_0\)处可微,\(y' = A\),记\(dy|_{x = x_0} = A\Delta x\)或\(dy|_{x = x_0} = f'(x_0)dx\)

函数的微分是函数增量的线性主部,\(dy = y'dx = y'\Delta x\)


\subsection{判别}
\begin{enumerate}
    \item 写增量\(\Delta y = f(x_0 + \Delta x) - f(x_0)\)
    \item 写线性增量\(A\Delta x = f'(x_0)\Delta x\)
    \item 作极限\(\displaystyle \lim_{\Delta x \to 0}\dfrac{\Delta y - A\Delta x}{\Delta x}\)
    \item 若极限等于0,则可微,否则不可
\end{enumerate}
\[\text{可微} \Leftrightarrow \text{可导} \Rightarrow \text{连续}\]


\subsection{几何意义}

若\(f(x)\)在点\(x_0\)处可微,则在点\((x_0, y_0)\)附近可用切线段近似代替曲线段


\section{多元函数微分学}

\subsection{邻域的几何意义}
\(U(P_0, \delta)\)表示xOy平面上以点\(P_0(x_0, y_0)\)为中心,\(\delta > 0\)为半径的圆内部的点\(P(x, y)\)的全体。

\subsection{极限}
\[\lim_{(x, y) \to (x_0, y_0)}f(x, y) = A\ or\ \lim_{x\to x_0\ y \to y_0}f(x, y) = A\]

\begin{itemize}
    \item 二重极限有无穷多种方式
    \item 若有两条路径使极限\(\displaystyle\lim_{(x, y) \to (x_0, y_0)}f(x, y)\)的值不相等或某一路径使极限\(\displaystyle\lim_{(x, y) \to (x_0, y_0)}f(x, y)\)的值不存在,说明\(\displaystyle\lim_{(x, y) \to (x_0, y_0)}f(x, y)\)不存在
    \item \(\displaystyle\lim_{(x, y) \to (x_0, y_0)}f(x, y) = A \Leftrightarrow f(x, y) = A + \alpha\),\(\alpha\)是无穷小量;
\end{itemize}


\subsection{连续}


\subsection{偏导数}

\subsubsection{定义}
设函数\(z = f(x, y)\)在点\((x_0, y_0)\)的某邻域内有定义,若极限\[\lim_{\Delta x \to 0}\dfrac{f(x_0 + \Delta x, y_0) - f(x_0, y_0)}{\Delta x}\]
存在,则此极限为对x的偏导数。


\subsubsection{几何意义}


\subsection{高阶偏导数}
若函数的两个二阶混合偏导数都在区域D内连续,则在区域D内\(\dfrac{\vartheta^2z}{\vartheta x\vartheta y} = \dfrac{\vartheta^2z}{\vartheta y\vartheta x}\)


\section{可微}

\subsection{定义}
全增量\[\Delta z = f(x + \Delta x, y + \Delta y) - f(x, y)\]
\[\Delta z = A\Delta x + B\Delta y + o(\rho)\]
A,B仅与点\((x, y)\)有关,与\(\Delta x,\Delta y\)无关;\(\rho = \sqrt{(\Delta x)^2 + (\Delta y)^2}\),\(\Delta x \to 0,\Delta y \to 0\)时,\(o(\rho)\)是\(\rho\)的高阶无穷小;

全微分\[dz = A\Delta x + B\Delta y\]


\subsection{线性近似微分}
设\(f(x, y)\)在点\((a, b)\)上可微,则存在线性近似\[f(x, y) - f(a, b) = f_x'(x - a) + f_y'(y - b) + o(\rho)\]
其中\(\rho = \sqrt{(x - a)^2 + (y - b)^2}\),即\[\lim_{\rho \to 0}\dfrac{f(x, y) - f(a, b) - [f_x'(x - a) + f_y'(y - b)]}{\rho} = 0\]


\subsection{必要条件}
若在点\((x, y)\)可微,则在该点的偏导数必存在\[A = \dfrac{\vartheta z}{\vartheta x}, B = \dfrac{\vartheta z}{\vartheta y}\]
则全微分可记为\(dz = \dfrac{\vartheta z}{\vartheta x}dx + \dfrac{\vartheta z}{\vartheta y}dy\)


\subsection{充分条件}
偏导数存在且连续\(\Rightarrow\)在该点可微


\subsection{判别}
\begin{enumerate}
    \item 写出全增量\(\Delta z\)
    \item 写出线性增量\(A\Delta x + B\Delta y\)
    \item 作极限\(\displaystyle\lim_{\Delta x \to 0\ \Delta y \to 0}\dfrac{\Delta z - (A\Delta x + B\Delta y)}{\sqrt{(\Delta x)^2 + (\Delta y)^2}}\),若极限等于0,则可微;否则不可微。
\end{enumerate}



\section{计算}

\subsection{链式求导规则}
对函数\(z = f(u, v), u = u(x, y), v = v(x, y)\),则\[\dfrac{\vartheta z}{\vartheta x} = \dfrac{\vartheta f}{\vartheta u}\dfrac{\vartheta u}{\vartheta x} + \dfrac{\vartheta f}{\vartheta v}\dfrac{\vartheta v}{\vartheta x}\]
\[\dfrac{\vartheta^2z}{\vartheta x\vartheta y} = \]


\subsection{全微分形式不变性}
设\(z = f(u, v), u = u(x, y), v = v(x, y)\),若分别有连续偏导数,则复合函数\(z = f(u, v)\)在\((x, y)\)处的全微分仍可表示为\[dz = \dfrac{\vartheta z}{\vartheta u}du + \dfrac{\vartheta z}{\vartheta v}dv\]

\subsection{隐函数存在定理(公式法)}
\begin{enumerate}
    \item 对方程\(F(x, y) = 0\)确定的隐函数\(y = f(x)\),当\(F_y'(x, y) \neq 0\)时,有\[\dfrac{dy}{dx} = -\dfrac{F_x'(x, y)}{F_y'(x, y)}\]

    \item 对由方程\(F(x, y, z) = 0\)确定的隐函数\(z = f(x, y)\),当\(F_z'(x, y, z) \neq 0\)时,则有\[\dfrac{\vartheta z}{\vartheta x} = -\dfrac{F_x'(x, y, z)}{F_z'(x, y, z)}, \dfrac{\vartheta z}{\vartheta y} = -\dfrac{F_y'(x, y, z)}{F_z'(x, y, z)}\]
\end{enumerate}


\section{极值与最值}

\subsection{无条件极值}
\begin{enumerate}
    \item 必要条件:在点\((x_0, y_0)\)处\(\begin{cases}
        \text{一阶偏导数存在} \\ 
        \text{取极值}
    \end{cases}\),则\(f_x'(x_0, y_0) = 0, f_y'(x_0, y_0) = 0\)
    \item 充分条件:记\(\begin{cases}
        f_{xx}''(x_0, y_0) = A \\ 
        f_{xy}''(x_0, y_0) = B \\ 
        f_{yy}''(x_0, y_0) = C
    \end{cases}\),则\(\Delta = AC - B^2\begin{cases}
        > 0 \Rightarrow \text{极值}\begin{cases}
            A < 0 \Rightarrow \text{极大值} \\ 
            A > 0 \Rightarrow \text{极小值}
        \end{cases} \\ 
        < 0 \Rightarrow \text{非极值} \\ 
        = 0 \Rightarrow \text{方法失效,换方法}
    \end{cases}\)
\end{enumerate}
通过必要条件找出可疑点,使用充分条件判别。


\subsection{条件最值与拉格朗日乘数法}
求目标函数\(u = f(x, y, z)\)在约束条件\(\begin{cases}
    \varphi(x, y, z) = 0 \\ 
    \psi(x, y, z) = 0
\end{cases}\)下的最值,则\begin{enumerate}
    \item 构造辅助函数\(F(x, y, z, \lambda, \mu) = f(x, y, z) + \lambda\varphi(x, y, z) + \mu\psi(x, y, z)\)
    \item 令\[\begin{cases}
        F_x' = f_x' + \lambda\varphi_x' + \mu\psi_x' = 0 \\ 
        F_y' = f_y' + \lambda\varphi_y' + \mu\psi_y' = 0 \\ 
        F_z' = f_z' + \lambda\varphi_z' + \mu\psi_z' = 0 \\ 
        F_\lambda' = \varphi(x, y, z) = 0 \\ 
        F_\mu' = \psi(x, y, z) = 0
    \end{cases}\]
    \item 解方程组得备选点\(P_i\),求\(f(P_i)\)取最值
\end{enumerate}
若约束条件\(\begin{cases}
    \varphi(x, y, z) = 0 \\ 
    \psi(x, y, z) = 0
\end{cases}\)易得\(z = z(x, y)\),则代入\(f(x, y, z(x, y))\)转化为无条件最值问题。


\subsection{最远(近)点的垂线定理}
若\(\Gamma\)是光滑闭曲线,点Q是\(\Gamma\)外一个点,点\(P_1, P_2\)分别是\(\Gamma\)上与点Q的最远点,最近点,则直线\(P_1Q, P_2Q\)分别在点\(P_1\)处,\(P_2\)处与\(\Gamma\)垂直,即\(P_1Q, P_2Q\)分别与点\(P_1, P_2\)的切线垂直。

若光滑闭曲线\(\Gamma_1, \Gamma_2\)不相交,点\(P_1, P_2\)分别是它们之间的最远(近)点,则直线\(P_1P_2\)是\(\Gamma_1, \Gamma_2\)的公垂线,即\(P_1P_2\)同时垂直于\(\Gamma_1, \Gamma_2\)在这两点的切线。



\section{例}

\subsubsection{隐函数求偏导数全微分}
若函数\(z = z(x, y)\)由方程\(e^{x + 2y + 3z} + xyz = 1\)确定,则\(dz\bigg|_{(0, 0)} = \)?
\paragraph{解}
x = 0, y = 0代入得z = 0
\subparagraph{方法1}
原式两端微分得\[e^{x + 2y + 3z}(dx + 2dy + 3dz) + yzdx + xzdy + zydz = 0\]
代入得\(dx + 2dy + 3dz = 0\)
\subparagraph{方法2}
隐函数求导公式得\[\dfrac{\vartheta z}{\vartheta x} = -\dfrac{e^{x + 2y + 3z} + yz}{3e^{x + 2y + 3z} + xy}\]
\[\dfrac{\vartheta z}{\vartheta y} = -\dfrac{2e^{x + 2y + 3z} + xz}{3e^{x + 2y + 3z} + xy}\]
代入得
\subparagraph{方法3}
将\(y = 0\)代入原式得\(e^{x + 3z} = 1\),两端对x求导得\[e^{x + 3z}(1 + 3z_x') = 0\]
代入得


\subsubsection{二元最值}
二元函数\(f(x, y) = x^2(2 + y^2) + y\ln y\)的极小值为?
\subparagraph{解}
\(f_x' = 2x(2 + y^2),\ f_y' = 2x^2y + \ln y + 1\)
令\(\begin{cases}
    f_x' = 0 \\ 
    f_y' = 0
\end{cases}\),解得驻点\((0, \dfrac{1}{e})\)
\begin{flalign}
    A & = f_{xx}''(x_0, y_0) \nonumber \\ 
    B & = f_{xy}''(x_0, y_0) \nonumber \\ 
    C & = f_{yy}''(x_0, y_0) \nonumber
\end{flalign}
故\(AC - B^2 > 0,\ A > 0\),\(\therefore f(0, \dfrac{1}{e})\)是极小值为\(-\dfrac{1}{e}\)


\subsubsection{导数定义求偏导}
设\(z = (y^x + \dfrac{\sin x}{\sqrt{x^2 + 2y^2}})^{\sqrt{x^2 + y^2}}\),则\(\dfrac{\vartheta z}{\vartheta x}\bigg|_{(0, 1)} = \)?
\subparagraph{解}
\(y = 1\)代入得\(z(x, 1) = (1 + \dfrac{\sin x}{\sqrt{x^2 + 2}})^{\sqrt{x^2 + 1}}\),设\(z(x, 1) = \varphi(x)\),故\[\dfrac{\vartheta z}{\vartheta x}\bigg|_{(0, 1)} = \varphi'(0) = \lim_{x \to 0}\dfrac{\varphi(x) - \varphi(0)}{x}\]


\subsubsection{隐函数全微分}
设z是方程\(x + y + z = \displaystyle\int_0^{xyz}e^{-t^2}dt\)确定的隐函数,则\(dz = \)?
\subparagraph{解}
两端一阶全微分得\[dx + dy + dz = e^{-x^2y^2z^2}d(xyz) = e^{-x^2y^2z^2}(yzdx + xzdy + xydz)\]


\subsubsection{极限、线性近似微分}
设连续函数\(z = f(x, y)\)满足\(\displaystyle\lim_{x \to 0, y \to 1}\dfrac{f(x, y) - 2x + y - 2}{\sqrt{x^2 + (y - 1)^2}} = 0\),则\(dz\bigg|_{(0, 1)} = \)?
\subparagraph{解}
由题可知\[\displaystyle\lim_{x \to 0, y \to 1}[f(x, y) - 2x + y - 2] = 0\]
由\(f(x, y)\)连续,则\[f(0, 1) = 1\]
从而有\(\displaystyle\lim_{x \to 0, y \to 1}\dfrac{f(x, y) - f(0, 1) - 2x + (y - 1)}{\sqrt{x^2 + (y - 1)^2}} = 0\)
,即\[f(x, y) - f(0, 1) = 2x - (y - 1) + o(\rho)\]
故\[f_x'(0, 1) = 2,\ f_y'(0, 1) = -1\]





