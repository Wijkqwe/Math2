
\chapter{微分}

\section{一元函数微分学}

\subsection{定义}

函数\(y = f(x)\)在点\(x_0\)的某个邻域内有定义,且\(x_0 + \Delta x\)在该邻域内,对于函数增量\(\Delta y = f(x_0 + \Delta x) - f(x_0)\),若存在与\(\Delta x\)无关的常数A使得\(\Delta y = A\Delta x + o(\Delta x)\),则\(f(x)\)在点\(x_0\)处可微,\(y' = A\),记\(dy|_{x = x_0} = A\Delta x\)或\(dy|_{x = x_0} = f'(x_0)dx\)

函数的微分是函数增量的线性主部,\(dy = y'dx = y'\Delta x\)。


\subsection{判别}
\begin{enumerate}
    \item 写增量\(\Delta y = f(x_0 + \Delta x) - f(x_0)\)
    \item 写线性增量\(A\Delta x = f'(x_0)\Delta x\)
    \item 作极限\(\displaystyle \lim_{\Delta x \to 0}\dfrac{\Delta y - A\Delta x}{\Delta x}\)
    \item 若极限等于0,则可微,否则不可
\end{enumerate}
\[\text{可微} \Leftrightarrow \text{可导} \Rightarrow \text{连续}\]


\subsection{几何意义}

若\(f(x)\)在点\(x_0\)处可微,则在点\((x_0, y_0)\)附近可用切线段近似代替曲线段。


\section{多元函数微分学}

\subsection{邻域的几何意义}
\(U(P_0, \delta)\)表示xOy平面上以点\(P_0(x_0, y_0)\)为中心,\(\delta > 0\)为半径的圆内部的点\(P(x, y)\)的全体。

\subsection{极限}
\[\lim_{(x, y) \to (x_0, y_0)}f(x, y) = A\ or\ \lim_{x\to x_0\ y \to y_0}f(x, y) = A\]

\begin{itemize}
    \item 二重极限有无穷多种方式
    \item 若有两条路径使极限\(\displaystyle\lim_{(x, y) \to (x_0, y_0)}f(x, y)\)的值不相等或某一路径使极限\(\displaystyle\lim_{(x, y) \to (x_0, y_0)}f(x, y)\)的值不存在,说明\(\displaystyle\lim_{(x, y) \to (x_0, y_0)}f(x, y)\)不存在
    \item \(\displaystyle\lim_{(x, y) \to (x_0, y_0)}f(x, y) = A \Leftrightarrow f(x, y) = A + \alpha\),\(\alpha\)是无穷小量;
\end{itemize}


\subsection{连续}


\subsection{偏导数}

\subsubsection{定义}
设函数\(z = f(x, y)\)在点\((x_0, y_0)\)的某邻域内有定义,若极限\[\lim_{\Delta x \to 0}\dfrac{f(x_0 + \Delta x, y_0) - f(x_0, y_0)}{\Delta x}\]
存在,则此极限为对x的偏导数。


\subsubsection{几何意义}


\subsection{高阶偏导数}
若函数的两个二阶混合偏导数都在区域D内连续,则在区域D内\(\dfrac{\vartheta^2z}{\vartheta x\vartheta y} = \dfrac{\vartheta^2z}{\vartheta y\vartheta x}\)


\section{可微}

\subsection{定义}
全增量\[\Delta z = f(x + \Delta x, y + \Delta y) - f(x, y)\]
\[\Delta z = A\Delta x + B\Delta y + o(\rho)\]
A,B仅与点\((x, y)\)有关,与\(\Delta x,\Delta y\)无关;\(\rho = \sqrt{(\Delta x)^2 + (\Delta y)^2}\),\(\Delta x \to 0,\Delta y \to 0\)时,\(o(\rho)\)是\(\rho\)的高阶无穷小;

全微分\[dz = A\Delta x + B\Delta y\]


\subsection{线性近似微分}
设\(f(x, y)\)在点\((a, b)\)上可微,则存在线性近似\[f(x, y) - f(a, b) = f_x'(x - a) + f_y'(y - b) + o(\rho)\]
其中\(\rho = \sqrt{(x - a)^2 + (y - b)^2}\),即\[\lim_{\rho \to 0}\dfrac{f(x, y) - f(a, b) - [f_x'(x - a) + f_y'(y - b)]}{\rho} = 0\]


\subsection{必要条件}
若在点\((x, y)\)可微,则在该点的偏导数必存在\[A = \dfrac{\vartheta z}{\vartheta x}, B = \dfrac{\vartheta z}{\vartheta y}\]
则全微分可记为\(dz = \dfrac{\vartheta z}{\vartheta x}dx + \dfrac{\vartheta z}{\vartheta y}dy\)


\subsection{充分条件}
\begin{flalign}
    & \text{偏导数存在且连续} \nonumber \\ 
    \Leftrightarrow & \lim_{(x, y) \to (x_0, y_0)}f'_x(x, y) = f_x'(x_0, y_0) \And \lim_{(x, y) \to (x_0, y_0)}f'_y(x, y) = f_y'(x_0, y_0) \nonumber \\ 
    \Rightarrow & \text{在该点可微} \nonumber
\end{flalign}


\subsection{判别}
\begin{enumerate}
    \item 写出全增量\(\Delta z\)
    \item 写出线性增量\(A\Delta x + B\Delta y\)
    \item 作极限\(\displaystyle\lim_{\Delta x \to 0\ \Delta y \to 0}\dfrac{\Delta z - (A\Delta x + B\Delta y)}{\sqrt{(\Delta x)^2 + (\Delta y)^2}}\),若极限等于0,则可微;否则不可微。
\end{enumerate}


\section{计算}

\subsection{链式求导规则}
对函数\(z = f(u, v), u = u(x, y), v = v(x, y)\),则\[\dfrac{\vartheta z}{\vartheta x} = \dfrac{\vartheta f}{\vartheta u}\dfrac{\vartheta u}{\vartheta x} + \dfrac{\vartheta f}{\vartheta v}\dfrac{\vartheta v}{\vartheta x}\]
\[\dfrac{\vartheta^2z}{\vartheta x\vartheta y} = \]


\subsection{全微分形式不变性}
设\(z = f(u, v), u = u(x, y), v = v(x, y)\),若分别有连续偏导数,则复合函数\(z = f(u, v)\)在\((x, y)\)处的全微分仍可表示为\[dz = \dfrac{\vartheta z}{\vartheta u}du + \dfrac{\vartheta z}{\vartheta v}dv\]

\subsection{隐函数存在定理(公式法)}
\begin{enumerate}
    \item 对方程\(F(x, y) = 0\)确定的隐函数\(y = f(x)\),当\(F_y'(x, y) \neq 0\)时,有\[\dfrac{dy}{dx} = -\dfrac{F_x'(x, y)}{F_y'(x, y)}\]

    \item 对由方程\(F(x, y, z) = 0\)确定的隐函数\(z = f(x, y)\),当\(F_z'(x, y, z) \neq 0\)时,则有\[\dfrac{\vartheta z}{\vartheta x} = -\dfrac{F_x'(x, y, z)}{F_z'(x, y, z)}, \dfrac{\vartheta z}{\vartheta y} = -\dfrac{F_y'(x, y, z)}{F_z'(x, y, z)}\]
\end{enumerate}


\section{极值与最值}

\subsection{无条件极值}
\begin{enumerate}
    \item 必要条件:在点\((x_0, y_0)\)处\(\begin{cases}
        \text{一阶偏导数存在} \\ 
        \text{取极值}
    \end{cases}\),则\(f_x'(x_0, y_0) = 0, f_y'(x_0, y_0) = 0\)
    \item 充分条件:记\(\begin{cases}
        f_{xx}''(x_0, y_0) = A \\ 
        f_{xy}''(x_0, y_0) = B \\ 
        f_{yy}''(x_0, y_0) = C
    \end{cases}\),则\(\Delta = AC - B^2\begin{cases}
        > 0 \Rightarrow \text{极值}\begin{cases}
            A < 0 \Rightarrow \text{极大值} \\ 
            A > 0 \Rightarrow \text{极小值}
        \end{cases} \\ 
        < 0 \Rightarrow \text{非极值} \\ 
        = 0 \Rightarrow \text{方法失效,换方法}
    \end{cases}\)
\end{enumerate}
通过必要条件找出可疑点,使用充分条件判别。


\subsection{条件最值与拉格朗日乘数法}
求目标函数\(u = f(x, y, z)\)在约束条件\(\begin{cases}
    \varphi(x, y, z) = 0 \\ 
    \psi(x, y, z) = 0
\end{cases}\)下的最值,则\begin{enumerate}
    \item 构造辅助函数\(F(x, y, z, \lambda, \mu) = f(x, y, z) + \lambda\varphi(x, y, z) + \mu\psi(x, y, z)\)
    \item 令\[\begin{cases}
        F_x' = f_x' + \lambda\varphi_x' + \mu\psi_x' = 0 \\ 
        F_y' = f_y' + \lambda\varphi_y' + \mu\psi_y' = 0 \\ 
        F_z' = f_z' + \lambda\varphi_z' + \mu\psi_z' = 0 \\ 
        F_\lambda' = \varphi(x, y, z) = 0 \\ 
        F_\mu' = \psi(x, y, z) = 0
    \end{cases}\]
    \item 解方程组得备选点\(P_i\),求\(f(P_i)\)取最值
\end{enumerate}
若约束条件\(\begin{cases}
    \varphi(x, y, z) = 0 \\ 
    \psi(x, y, z) = 0
\end{cases}\)易得\(z = z(x, y)\),则代入\(f(x, y, z(x, y))\)转化为无条件最值问题。


\subsection{最远(近)点的垂线定理}
若\(\Gamma\)是光滑闭曲线,点Q是\(\Gamma\)外一个点,点\(P_1, P_2\)分别是\(\Gamma\)上与点Q的最远点,最近点,则直线\(P_1Q, P_2Q\)分别在点\(P_1\)处,\(P_2\)处与\(\Gamma\)垂直,即\(P_1Q, P_2Q\)分别与点\(P_1, P_2\)的切线垂直。

若光滑闭曲线\(\Gamma_1, \Gamma_2\)不相交,点\(P_1, P_2\)分别是它们之间的最远(近)点,则直线\(P_1P_2\)是\(\Gamma_1, \Gamma_2\)的公垂线,即\(P_1P_2\)同时垂直于\(\Gamma_1, \Gamma_2\)在这两点的切线。



