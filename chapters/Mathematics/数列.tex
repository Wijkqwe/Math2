
\chapter{数列}

\section{定义}

\section{前n项和}

\begin{flalign}
\sum_{k = 1}^nk & = \dfrac{n(n + 1)}{2} \nonumber \\ 
\sum_{k = 1}^nk^2 & = \dfrac{n(n + 1)(n + 2)}{6} \nonumber \\ 
\sum_{k = 1}^n\dfrac{1}{k(k + 1)} & = \dfrac{n}{n + 1} = 1 - \dfrac{1}{n + 1} \nonumber
\end{flalign}


\subsection{重要数列}

\paragraph{\((1 + \dfrac{1}{n})^n\)}
单调递增,\(\displaystyle\lim_{x \to \infty}(1 + \dfrac{1}{n})^n = e\)


\section{数列极限}

\subsection{定义}
设数列\(\{x_n\}\),若存在常数a,对于\(\forall \epsilon > 0, \exists N \in Z_+\),使得当n > N时,\(|x_n - a| < \epsilon\)恒成立,则称常数a是数列的极限,或数列收敛于a,记为\(\displaystyle\lim_{n \to \infty}x_n = a\);

若不存在a,则数列是发散的。

\(\displaystyle \lim_{n \to \infty}x_n = a\),当a = 0时,称\(x_n\)为\(n \to \infty\)时的无穷小量;

\(\displaystyle \lim_{n \to \infty}x_n = \infty\),称\(x_n\)为\(n \to \infty\)时的无穷大量;


\subsection{定理}
\begin{itemize}
    \item 若数列\(\{a_n\}\)收敛,则其任意子列\(\{a_{n_k}\}\)也收敛,且\(\displaystyle \lim_{k \to \infty}a_{n_k} = \lim_{n \to \infty}a_n\)。反之,若存在发散子列,则原数列发散;若至少两个收敛子列收敛到不同极限,则原数列发散。
    \item 对数列\(\{x_n\}\)若\(\displaystyle \lim_{n \to \infty}x_n = a\)存在,则a唯一;
    \item 对数列\(\{x_n\}\)若极限存在,则数列有界;
    \item 设\(\displaystyle \lim_{n \to \infty}x_n = a >(<) b\),则存在\(N > 0\),当n > N时,有\(x_n >(<) b\);若从某项起有\(x_n >=(<=) b\),且\(\lim_{n \to \infty}x_n = a\),则\(a >=(<=) b\),其中b为任意实数,如b = 0。
\end{itemize}


\subsection{四则运算}
设\(\displaystyle \lim_{n \to \infty}x_n = a, \lim_{n \to \infty}y_n = b\),则
\[\lim_{n \to \infty}(x_n \pm y_n) = a \pm b\]
\[\lim_{n \to \infty}x_ny_n = ab\]
\[\lim_{n \to \infty}\dfrac{x_n}{y_n} = \dfrac{a}{b},b \neq 0\]


\subsection{海涅定理/归结原则}
设\(f(x)\)在\(U(x_0, \delta)\)内有定义,则
\[\displaystyle \lim_{x \to x_0}f(x) = A\text{存在} \Leftrightarrow \text{对任意}U(x_0, \delta)\text{内以}x_0\text{为极限的数列,极限}\lim_{x \to \infty}f(x_n) = A\text{存在}\]

在极限存在的情况下,函数极限和数列极限可以相互转化;

当\(x \to 0\),取\(x_n = \dfrac{1}{n}\),即若\(\displaystyle \lim_{x \to 0}f(x) = A\),则\(\displaystyle \lim_{n \to \infty}f(\frac{1}{n}) = A\)

当\(x \to \infty\),取\(x_n = n\),即若\(\displaystyle \lim_{x \to \infty}f(x) = A\),则\(\displaystyle \lim_{n \to \infty}f(n) = A\)

当\(x \to a\),且\(x_n \neq a\)时,若\(\displaystyle \lim_{x \to a}f(x) = A\),则\(\displaystyle \lim_{n \to \infty}f(x_n) = A\)

\section{夹逼准则}

设数列{\(x_n\)},{\(y_n\)},{\(z_n\)}满足以下条件:

()从某项起,\(\exists n_0 \in N_+\)当n > \(n_0\)时,\(y_n <= x_n <= z_n\)

()\(\displaystyle \lim_{n \to \infty}y_n = a, \displaystyle \lim_{n \to \infty}z_n = a\)

则数列{\(x_n\)}极限存在,\(\displaystyle \lim_{n \to \infty}x_n = a\)

\section{放缩}
常用方法

\subsection{简单放大缩小}
\[
\begin{cases}
n * u_{min} <= u_1 + ... + u_n <= n * u_{max} \\ 
\text{当}u_i >= 0, 1 * u_{max} <= u_1 + ... + u_n <= n * u_{max}
\end{cases}
\]

\subsection{重要不等式}
\begin{enumerate}
    \item 设\(a, b\)实数,则\(|a \pm b| <= |a| + |b|, ||a| - |b|| <= |a - b|\),即\[|a_1 \pm a_2 \pm ... \pm a_n| <= |a_1| + ... + |a_n|\]
    \item \begin{itemize}
        \item \(\sqrt{ab} <= \dfrac{a + b}{2} <= \sqrt{\dfrac{a^2 + b^2}{2}}(a, b >= 0)\)
        \item \(\sqrt[3]{abc} <= \dfrac{a + b + c}{3} <= \sqrt{\dfrac{a^2 + b^2 + c^2}{3}}(a, b, c >= 0)\)
        \item \(|ab| <= \dfrac{a^2 + b^2}{2}\),若\(u_n > 0\),则\(\dfrac{u_n}{n} = u_n * \dfrac{1}{n} <= \dfrac{u_n^2 + \dfrac{1}{n^2}}{2}\)
    \end{itemize}
    \item 设\(a >= b >= 0\),则\(
    \begin{cases}
    m > 0, a^m >= b^m \\ 
    m < 0, a^m <= b^m
    \end{cases}\)
    \item 若\(0 < a < x < b, 0 < c < y < d\),则\(\dfrac{c}{b} < \dfrac{y}{x} < \dfrac{d}{a}\),如\(n\pi < x < (n + 1)\pi, 2n < S(x) < 2(n + 1), \dfrac{2n}{(n + 1)\pi} < \dfrac{S(x)}{x} < \dfrac{2(n + 1)}{n\pi}\)
    \item \(\sin x < x < \tan x(0 < x < \dfrac{\pi}{2})\)
    \item \(\sin x < x(x > 0)\),如当\(x_n > 0, x_{n + 1} = \sin x_n < x_n\),故{\(x_n\)}单调递减
    \item \(x < \tan x < \dfrac{4}{\pi}x, (0 < x < \dfrac{\pi}{4})\)
    \item \(\sin x > \dfrac{2}{\pi}x, (0 < x < \dfrac{\pi}{2})\)
    \item \(\arctan x <= x <= \arcsin x, (0 <= x <= 1\),如当\(x_n > 0\)时,\(x_{n + 1} = \arctan x_n < x_n\),故{\(x_n\)}单调递减
    \item \(e^x >= x + 1\),如当\(x_{n + 1} = e^x - 1\)时,\(e^x - 1 >= x_n, x_{n + 1} >= x_n\)
    \item \(x - 1 >= \ln x, (x > 0)\),如当\(x_n > 0\)时,\(x_{n + 1} = \ln x_n + 1, \ln x_n + 1 <= x_n, x_{n + 1} <= x_n\)
    \item \(\dfrac{1}{1 + x} < \ln(1 + \dfrac{1}{x}) < \dfrac{1}{x}, (x > 0), \dfrac{x}{1 + x} < \ln(1 + n) < x, (x > 0)\)
\end{enumerate}






\subsection{3}
利用闭区间上连续函数必有最大值与最小值;

\subsection{压缩映射原理}
需证明过程

(1)对数列{\(x_n\)},若存在常数k(0 < k< 1),使得\(|x_{n + 1} - a| <= k|x_n - a|, n = 1, 2, ...\),则{\(x_n\)}收敛于\(a\);

证明:\(0 <= |x_{n + 1} - a| <= k|x_n - a| <= k^2|x_{n - 1} - a| <= ... <= k^n|x_1 - a|\),由\(\displaystyle \lim_{n \to \infty}k^n = 0\),夹逼准则得\(\displaystyle \lim_{n \to \infty}|x_{n + 1} - a| = 0\),即{\(x_n\)}收敛于\(a\);

(2)对数列{\(x_n\)},若\(x_{n + 1} = f(x_n), n = 1, 2, ..., f(x)\)可导,\(a\)是\(f(x) = x\)的唯一解,且\(\forall x \in R, |f'(x)| <= k < 1\),则{\(x_n\)}收敛于\(a\);

证明:\(|x_{n + 1} - a| = |f(x_n) - f(a)| = |f'(\xi)||x_n - a| <= k|x_n - a|\),其中\(\xi\)介于\(a, x_n\)之间,由(1)得{\(x_n\)}收敛于\(a\);


\section{单调有界准则}
单调有界数列必有极限,即若数列{\(x_n\)}单调增加(减少)且有上界(下界),则\(\displaystyle \lim_{x \to \infty}x_n\)存在;

\paragraph{证明数列单调性}
:\begin{itemize}
    \item \(x_{n + 1} - x_n >(<) 0, \dfrac{x_{n + 1}}{x_n} >(<) 1\)
    \item 数学归纳法;
    \item 重要不等式(同夹逼准则);
    \item \(x_n - x_{n - 1}, x_{n - 1} - x_{n - 2}\)同号,则数列单调
    \item 对\(x_{n + 1} = f(x_n)\),\(x_n \in\)区间I\begin{itemize}
        \item 若\(f'(x) > 0\),则数列单调,且\(\begin{cases}
        \text{当}x_2 > x_1\text{时,数列}\{x_n\}\text{单调增加} \\ 
        \text{当}x_2 < x_1\text{时,数列}\{x_n\}\text{单调减少}
        \end{cases}\)
        \item 若\(f'(x) < 0\),则数列不单调
    \end{itemize}
\end{itemize}





\section{计算}

对于分奇偶数列\(x_n\),分别求\(\displaystyle \lim_{n \to \infty}x_{2n}\)与\(\displaystyle \lim_{n \to \infty}x_{2n - 1}\)并比较

\(\displaystyle \lim_{n \to \infty}\sqrt[n]{a_1^n + ... + a_n^n} = \max\{a_1, ..., a_n\}\)

当\(0 < a < b\)时,\(\displaystyle \lim_{n \to \infty}(a^{-n} + b^{-n})^{\frac{1}{n}} = \lim_{n \to \infty}\sqrt[n]{(\dfrac{1}{a})^n + (\dfrac{1}{b})^n} = \dfrac{1}{a}\)

当\(0 <= x <= \dfrac{\pi}{2}\)时,\(\displaystyle \lim_{n \to \infty}\sqrt[n]{\sin^nx + \cos^nx} = 
\begin{cases}
\cos x, 0 <= x <= \dfrac{\pi}{4} \\ 
\sin x, \dfrac{\pi}{4} < x <= \dfrac{\pi}{2}
\end{cases}
\)

\(\displaystyle \lim_{n \to \infty}\sqrt[n]{1 + |x|^{3n}} = 
\begin{cases}
1,\ \ \ \ |x| <= 1 \\ 
|x|^3, |x| > 1
\end{cases}
\)

对\(x_n = f(x_{n - 1})\),设\(\displaystyle\lim_{n \to +\infty}x_n = a\),解\(a = f(a)\)

