
\chapter{导数}

\section{定义}
\[f'(x_0) = \lim_{\Delta x \to 0}\dfrac{f(x_0 + \Delta x) - f(x_0)}{\Delta x} = 
\lim_{x \to x_0}\dfrac{f(x) - f(x_0)}{x - x_0}\]

\subsection{单侧导数}
\(f'_-(x_0) = \displaystyle \lim_{\Delta x \to 0^-}\dfrac{f(x_0 + \Delta x) - f(x_0)}{\Delta x}\)

\subsection{判别}

\(f(x)\)在\(x = a\)处可导,即\(f'_-(a) = f'_+(a)\)

函数不具备导数存在条件时,往往仅能用导数定义求导。

判断复合函数是否可导用\textbf{导数定义}。
\begin{itemize}
    \item 若\(f(x)\)在\(x = x_0\)处连续,则\(F(x) = |x - x_0|f(x)\)在点\(x_0\)处可导的\textbf{充要条件}是\(f(x_0) = 0\)。
    \item 设\(f(x), g(x)\)在\(x_0\)可导,\(F(x) = g(x)|f(x)|\),又\(f(x_0) = 0\),则\(F'(x_0)\)存在的\textbf{充要条件}是\(g(x_0)f'(x_0) = 0\)。
\end{itemize}

设\(f(x) = g(x) + h(x)\),若\(g(x), h(x)\)可导,则\(f(x)\)可导;

\[\text{可导} \Leftrightarrow \text{可微} \Rightarrow \text{连续}\]



\section{高阶导数}

\subsection{归纳法}
\begin{flalign}
(e^{ax + b})^{(n)} & = a^ne^{ax + b} \nonumber \\ 
[\sin(ax + b)]^{(n)} & = a^n\sin(ax + b + \dfrac{n\pi}{2}) \nonumber \\ 
[\cos(ax + b)]^{(n)} & = a^n\cos(ax + b + \dfrac{n\pi}{2}) \nonumber \\ 
[\ln(ax + b)]^{(n)} & = (-1)^{n - 1}a^n\dfrac{(n - 1)!}{(ax + b)^n} \nonumber \\ 
(\dfrac{1}{ax + b})^{(n)} & = (-1)^na^n\dfrac{n!}{(ax + b)^{n + 1}} \nonumber
\end{flalign}

\subsection{莱布尼茨公式}
设\(u(x), v(x)\)n阶可导,则
\[(u \pm v)^{(n)} = u^{(n)} \pm v^{(n)}\]
\[(uv)^{(n)} = u^{(n)}v + C_n^1u^{(n - 1)}v' + ... + C_n^{(n - 1)}u'v^{(n - 1)} + uv^{(n)} = \sum_{k = 0}^nC_n^ku^{(n - k)}v^{(k)}\]

\subsection{泰勒展开式}
根据展开式的唯一性,比较系数
\begin{flalign}
f(x) & = \sum_{n = 0}^\infty\dfrac{f^{(n)}(x_0)}{n!}(x - x_0)^n \nonumber \\ 
f(x) & = \sum_{n = 0}^\infty\dfrac{f^{(n)}(0)}{n!}x^n \nonumber \\ 
e^x & = \sum_{n = 0}^\infty\dfrac{x^n}{n!} = 1 + x + \dfrac{x^2}{2!} + ... + \dfrac{x^n}{n!} + ... \nonumber \\ 
sin\ x & = \sum_{n = 0}^\infty(-1)^n\dfrac{x^{2n + 1}}{(2n + 1)!} = x - \dfrac{x^3}{3!} + \dfrac{x^5}{5!} - ... + (-1)^{n - 1}\dfrac{x^{2n - 1}}{(2n - 1)!} + ... \nonumber \\ 
cos\ x & = \sum_{n = 0}^\infty(-1)^n\dfrac{x^{2n}}{(2n)!} = 1 - \dfrac{x^2}{2!} + \dfrac{x^4}{4!} - ... + (-1)^{n}\dfrac{x^{2n}}{(2n)!} + ... \nonumber \\ 
tan\ x & =x + \dfrac{x^3}{3} + ... \nonumber \\ 
arcsin\ x & = x + \dfrac{x^3}{3!} + ... \nonumber \\ 
arctan\ x & = x - \dfrac{x^3}{3} + ... \nonumber \\ 
\ln(1 + x) & = \sum_{n = 1}^\infty(-1)^{n - 1}\dfrac{x^n}{n} = x - \dfrac{x^2}{2} + \dfrac{x^3}{3} - ... + (-1)^{n - 1}\dfrac{x^n}{n} + ..., -1 < x <= 1 \nonumber \\ 
(1 + x)^\alpha & = 1 + \alpha x + \dfrac{\alpha(\alpha - 1)}{2!}x^2 + ... + \dfrac{\alpha(\alpha - 1)...(\alpha - n + 1)}{n!}x^n + ... \nonumber \\ 
\dfrac{1}{1 + x} & = \sum_{n = 0}^\infty(-1)^nx^n = 1 -x + x^2 - x^3 + ... + (-1)^nx^n + ..., -1 < x < 1 \nonumber \\ 
\dfrac{1}{1 - x} & = \sum_{n = 0}^\infty x^n = 1 + x + x^2 + x^3 + ... + x^n + ..., -1 < x < 1 \nonumber
\end{flalign}


\section{计算}

\subsection{复合函数导数}

\begin{flalign}
\{f[g(x)]\}' &= \dfrac{d\{f[g(x)]\}}{dx} \nonumber \\ 
f'[g(x)] &= \dfrac{d\{f[g(x)]\}}{d[g(x)]} \nonumber
\end{flalign}


\subsection{分段函数导数}

设\(f(x) = 
\begin{cases}
f_1(x), x >= x_0 \\ 
f_2(x), x < x_0
\end{cases}\),其中\(f_1(x), f_2(x)\)分别在\(x > x_0, x < x_0\)可导,则\begin{itemize}
    \item 分段点\(x_0\)处用导数定义求导:\(f'_+(x_0) = \displaystyle \lim_{x \to x^+_0}\dfrac{f_1(x) - f(x_0)}{x - x_0}, f'_-(x_0) = \displaystyle \lim_{x \to x^-_0}\dfrac{f_2(x) - f(x_0)}{x - x_0}\)
    \item 非分段点用导数公式
\end{itemize}


\subsection{反函数导数}

设\(y = f(x)\)为单调可导函数且\(f'(x) \neq 0\),则存在反函数\(x = \varphi(y)\),且\(\dfrac{dx}{dy} = \dfrac{1}{\dfrac{dy}{dx}}\),即\(\varphi'(y) = \dfrac{1}{f'(x)}\)

\subsubsection{二阶导数}
设\(y = f(x)\)为单调二阶可导函数,且\(f'(x) \neq 0\),则存在反函数\(x = \varphi(y)\),记\(f'(x) = y'_x, \varphi'(y) = x'_y\),则有\[y'_x = \dfrac{dy}{dx} = \dfrac{1}{\dfrac{dx}{dy}} = \dfrac{1}{x'_y}\]
\[y''_{xx} = \dfrac{d^2y}{dx^2} = \dfrac{d\dfrac{dy}{dx}}{dx} = \dfrac{d\dfrac{1}{x'_y}}{dx} = \dfrac{d\dfrac{1}{x'_y}}{dy} * \dfrac{1}{x'_y} = -\dfrac{1}{(x'_y)^2} * (x'_y)' * \dfrac{1}{x'_y} = -\dfrac{x''_{yy}}{(x'_y)^2} * \dfrac{1}{x'_y} = -\dfrac{x''_{yy}}{(x'_y)^3}\]
反之,有\(x'_y = \dfrac{1}{y'_x}, x''_{yy} = -\dfrac{y''_{xx}}{(y'_x)^3}\)


\subsection{隐函数导数}

设\(y = y(x)\)是由\(F(x, y) = 0\)确定的可导函数,则\begin{enumerate}
    \item 方程\(F(x, y) = 0\)两边对自变量x求导,将y看作中间变量,得到关于\(y'\)的方程;
    \item 解该方程求出\(y'\)
\end{enumerate}


\subsection{参数方程确定函数导数}

设函数\(y = y(x)\)由参数方程\(\begin{cases}
x = \varphi(t) \\ 
y = \psi(t)
\end{cases}\)确定,其中t是参数,且\(\varphi(t), \psi(t)\)均可导,\(\varphi'(t) \neq 0\),则\[\dfrac{dy}{dx} = \dfrac{dy/dt}{dx/dt} = \dfrac{\psi'(t)}{\varphi'(t)}\]

\subsubsection{二阶导数}

\[\dfrac{d^2y}{dx^2} = \dfrac{d\dfrac{dy}{dx}}{dx} = \dfrac{d\dfrac{dy}{dx} / dt}{dx / dt} = \dfrac{\psi''(t)\varphi'(t) - \psi'(t)\varphi''(t)}{[\varphi'(t)]^3}\]

\subsection{对数求导法}

对多项相乘、相除、开方、乘方的式子,一般先取对数在求导,设\(y = f(x), (f(x) > 0)\),则\begin{enumerate}
    \item 等式两边取对数,得\(\ln y = \ln f(x)\)
    \item 两边对自变量x求导,得\(\dfrac{1}{y}y' = [\ln f(x)]' \Rightarrow y' = \dfrac{yf'(x)}{f(x)}\)
\end{enumerate}


\subsection{幂指函数求导法}

对\(u(x)^{v(x)}, u(x) > 0\)且\(u(x) \neq 1\),可化为指数函数\[u(x)^{v(x)} = e^{v(x)\ln u(x)}\]后求导\[[u(x)^{v(x)}]' = [e^{v(x)\ln u(x)}]' = u(x)^{v(x)}[v'(x)\ln u(x) + v(x)\dfrac{u'(x)}{u(x)}]\]




\subsection{例}

\paragraph{1}
设\(f(x)\)在\(x = x_0\)处二阶可导,且\(f'(x_0) = 0, f''(x_0) \neq 0\)证明:

1)\(f''(x_0) < 0\)时,f(x)在\(x_0\)处取极大值;

2)\(f''(x_0) > 0\)时,f(x)在\(x_0\)处取极小值;

1):\(f''(x_0) = \displaystyle \lim_{x \to x_0}\dfrac{f'(x) - f'(x_0)}{x - x_0} < 0\),根据函数极限的局部保号性,存在\(x_0\)的去心邻域\(U(x_0, \delta)\),当\(x \in U(x_0, \delta)\),有\(\dfrac{f'(x) - f'(x_0)}{x - x_0} < 0\),因为\(f'(x_0) = 0\),\(f'(x)\)与\(x - x_0\)符号相反,根据判别极值第一充分条件,\(f(x)\)在点\(x_0\)处极大值。

\paragraph{2}
已知\(g(x)\)在\(x = 0\)处二阶可导,且\(g(0) = g'(0) = 0\),设\(f(x) = \begin{cases}
\dfrac{g(x)}{x}, x\neq 0, \\ 
0, x = 0,
\end{cases}\),证明:\(f(x)\)导函数在x = 0处连续

\(f'(0) = \displaystyle \lim_{x \to 0}\dfrac{\dfrac{g(x)}{x} - 0}{x - 0} = \lim_{x \to 0}\dfrac{g(x)}{x^2} = \lim_{x \to 0}\dfrac{g'(x)}{2x} = \dfrac{1}{2}\lim_{x \to 0}\dfrac{g'(x) - g'(0)}{x - 0} = \dfrac{1}{2}g''(0)\)

当\(x \neq 0\)时,\(f'(x) = \dfrac{xg'(x) - g(x)}{x^2}\),则\(\displaystyle\lim_{x \to 0}f'(x) = \lim_{x \to 0}\dfrac{xg'(x) - g(x)}{x^2} = \lim_{x \to 0}\dfrac{g'(x)}{x} - \lim_{x \to 0}\dfrac{g(x)}{x^2} = g''(0) - \dfrac{1}{2}g''(0) = f'(0)\),因此连续。


