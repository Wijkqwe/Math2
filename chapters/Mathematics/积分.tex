
\chapter{积分}

\section{概念性质}

\subsection{不定积分}
\(\int f(x)dx = F(x) + C, F(x)\)是\(f(x)\)在区间\(I\)上的一个原函数;

\paragraph{原函数存在定理}
连续函数\(f(x)\)必有原函数\(F(x)\)

含第一类间断点和无穷间断点的函数在包含间断点的区间内必无原函数


\subsection{定积分}

\subsubsection{定义}
\(\displaystyle \int_a^b f(x)dx = \lim_{\lambda\to 0}\sum_{k = 1}^nf(\xi_k)\Delta x_k\)

\paragraph{几何意义}
在\([a,\ b]\)上,
\begin{enumerate}
    \item 若\(f(x) >= 0\)定积分\(\displaystyle\int_a^bf(x)dx\)表示由曲线\(y = f(x)\),直线\(x = a,\ x = b\)与x轴围成的曲边梯形的面积
    \item 若\(f(x) <= 0\)定积分\(\displaystyle\int_a^bf(x)dx\)表示由曲线\(y = f(x)\),直线\(x = a,\ x = b\)与x轴围成的曲边梯形的面积的负值
    \item 若\(f(x)\)既有正值也有负值,则表示x轴上方面积减去x轴下方面积
\end{enumerate}

\paragraph{精确定义}
\[\int_a^bf(x)dx = \lim_{x \to \infty}\sum_{i = 1}^nf(a + \dfrac{b - a}{n}i)\dfrac{b - a}{n}\]
\[\int_0^1f(x)dx = \lim_{x \to \infty}\sum_{i = 1}^nf(\dfrac{i}{n})\dfrac{1}{n}\]


\subsubsection{存在定理}
一元函数的(常义)可积性,即“黎曼”可积性,“常义”指区间有限,函数有界;

\paragraph{充分条件}
\begin{enumerate}
    \item 若\(f(x)\)在[a, b]上连续,则\(\displaystyle\int_a^bf(x)dx\)存在
    \item 若\(f(x)\)在[a, b]上单调,则\(\displaystyle\int_a^bf(x)dx\)存在
    \item 若\(f(x)\)在[a, b]上有界,且只有有限个间断点,则\(\displaystyle\int_a^bf(x)dx\)存在
\end{enumerate}

\paragraph{必要条件}
可积函数必有界,若定积分\(\displaystyle\int_a^bf(x)dx\)存在,则\(f(x)\)在[a, b]上必有界

\subsubsection{性质}
假设积分均存在
\begin{itemize}
    \item 当\(a = b\)时,\(\displaystyle\int_a^bf(x)dx = 0\)
    \item 当\(a > b\)时,\(\displaystyle\int_a^bf(x)dx = -\int_b^af(x)dx\)
\end{itemize}

\paragraph{求区间长度}
设\(a < b\),则\(\displaystyle\int_a^bdx = b - a = L\),L为[a, b]的长度

\paragraph{积分的线性性质}
设\(k_1, k_2\)常数,则\(\displaystyle\int_a^b[k_1f(x) \pm k_2g(x)]dx = k_1\int_a^bf(x)dx \pm k_2\int_a^bg(x)dx\)

\paragraph{可加(拆)性}
总有\(\displaystyle \int_a^bf(x)dx = \int_a^cf(x)dx + \int_c^bf(x)dx\)

\paragraph{保号性}
若区间[a, b]上\(f(x) <= g(x)\),则\(\displaystyle\int_a^bf(x)dx <= \int_a^bg(x)dx\)
\[\displaystyle|\int_a^bf(x)dx| <= \int_a^b|f(x)|dx\]

\paragraph{估值定理}
设M,m为\(f(x)\)在区间[a, b]上最大值与最小值,L为[a, b]长度,则有\[mL <= \displaystyle\int_a^bf(x)dx <= ML\]

\paragraph{中值定理}
设\(f(x)\)在区间[a, b]上连续,则[a, b]上至少存在一点\(\xi\),使得\[\displaystyle\int_a^bf(x)dx = f(\xi)(b - a)\]

\subparagraph{证明}
\(\because\ f(x)\)连续,\(\therefore\ \)有最大值M最小值m
\[m(b - a) <= \int_a^bf(x)dx <= M(b - a)\]
\[m <= \dfrac{1}{b - a}\int_a^bf(x)dx <= M\]
由介值定理得\(\exists \xi \in [a, b]\)使得\(f(\xi) = \dfrac{1}{b - a}\int_a^bf(x)dx\)


\subsection{变限积分}

\subsubsection{定义}
\[F(x) = \int_a^xf(t)dt,\ a <= x <= b\]
变上限定积分

\subsubsection{性质}
\begin{itemize}
    \item 函数\(f(x)\)在\(I\)上可积,则函数\(F(x) = \displaystyle \int_a^xf(t)dt\)在\(I\)上连续
    \item 函数\(f(x)\)在\(I\)上连续,则函数\(F(x) = \displaystyle \int_a^xf(t)dt\)在\(I\)上可导,且\(F'(x) = f(x)\)
    \item \begin{itemize}
        \item 若\(x = x_0\in I\)是\(f(x)\)的唯一跳跃间断点,则\(F(x) = \int_a^xf(t)dt\)在\(x_0\)处不可导,且\(\begin{cases}
            F'_-(x_0) = \lim_{x \to x_0^-}f(x) \\ 
            F'_+(x_0) = \lim_{x \to x_0^+}f(x)
        \end{cases}\)
        \item 若\(x = x_0\in I\)是\(f(x)\)的唯一可去间断点,则\(F(x) = \int_a^xf(t)dt\)在\(x_0\)处可导,且\(F'(x_0) = \lim_{x \to x_0}f(x)\)
    \end{itemize}
    \item 变限积分存在则必连续
\end{itemize}


\subsection{反常积分}
定积分的两个必要条件:积分区间有限,被积函数有界;破坏积分区间的有限性,引出无穷区间上的反常积分;破坏被积函数的有界性,引出无界函数的反常积分。
\subsubsection{定义}

\paragraph{无穷区间上}
设\(F(x)\)是\(f(x)\)在对应区间上的原函数
\begin{itemize}
    \item \(\displaystyle\int_a^{+\infty}f(x)dx = \lim_{x \to +\infty}F(x) - F(a)\)若极限存在,则反常积分收敛,否则发散
    \item \(\displaystyle\int_{-\infty}^bf(x)dx = F(b) - \lim_{x \to -\infty}F(x)\)若极限存在,则反常积分收敛,否则发散
    \item \(\displaystyle\int_{-\infty}^{+\infty}f(x)dx = \int_{-\infty}^{x_0}f(x)dx + \int_{x_0}^{+\infty}f(x)dx\)若右端两积分都收敛,则反常积分收敛,否则发散
    \item \(\displaystyle\lim_{x \to +\infty}f(x) = 0\)是\(\displaystyle\int_a^{+\infty}f(x)dx\)收敛的必要条件
\end{itemize}

\paragraph{无界函数的}
设\(F(x)\)是\(f(x)\)在对应区间上的原函数,\(x_0\)为\(f(x)\)的瑕点\footnote{使\(f(x)\)在\(x_0\)的邻域内无界的点为瑕点}
\begin{itemize}
    \item 若\(x = b\)是唯一瑕点,则\(\displaystyle\int_a^{b}f(x)dx = \lim_{x \to b^-}F(x) - F(a)\)若极限存在,则反常积分收敛,否则发散
    \item 若\(x = a\)是唯一瑕点,则\(\displaystyle\int_{a}^bf(x)dx = F(b) - \lim_{x \to a^+}F(x)\)若极限存在,则反常积分收敛,否则发散
    \item 若\(x = c \in (a, b)\)是唯一瑕点,则\(\displaystyle\int_{a}^{b}f(x)dx = \int_{a}^{c}f(x)dx + \int_{c}^{b}f(x)dx\)若右端两积分都收敛,则反常积分收敛,否则发散
\end{itemize}


\subsubsection{敛散性判别}

反常积分中,通常将\(\infty\)与瑕点统称为奇点,在判别积分敛散性时,一个积分只能有一个奇点;若出现两个及以上,需拆分

\paragraph{无穷区间}

\subparagraph{比较判别法}
设函数\(f(x), g(x)\)在区间\([a, +\infty)\)上连续,且\(0 <= f(x) <= g(x), a <= x < +\infty\)则
\begin{itemize}
    \item 当\(\displaystyle\int_a^{+\infty}g(x)dx\)收敛时,\(\displaystyle\int_a^{+\infty}f(x)dx\)收敛
    \item 当\(\displaystyle\int_a^{+\infty}f(x)dx\)发散时,\(\displaystyle\int_a^{+\infty}g(x)dx\)发散
\end{itemize}

\subparagraph{比较判别法的极限形式}
设函数\(f(x), g(x)\)在区间\([a, +\infty)\)上连续,且\(f(x) >= 0, g(x) > 0, \displaystyle\lim_{x \to +\infty}\dfrac{f(x)}{g(x)} = \lambda\)(有限或\(\infty\)),则
\begin{itemize}
    \item 当\(\lambda \neq 0\ \&\&\ \lambda \neq \infty\)时,\(\displaystyle\int_a^{+\infty}f(x)dx\)与\(\displaystyle\int_a^{+\infty}g(x)dx\)有相同的敛散性
    \item 当\(\lambda = 0\)时,若\(\displaystyle\int_a^{+\infty}g(x)dx\)收敛,则\(\displaystyle\int_a^{+\infty}f(x)dx\)也收敛
    \item 当\(\lambda = \infty\)时,若\(\displaystyle\int_a^{+\infty}g(x)dx\)发散,则\(\displaystyle\int_a^{+\infty}f(x)dx\)也发散
\end{itemize}


\paragraph{无界函数}

\subparagraph{比较判别法}
设函数\(f(x), g(x)\)在区间\((a, b]\)上连续,瑕点同为\(x = a\),且\(0 <= f(x) <= g(x), a < x <= b \)则
\begin{itemize}
    \item 当\(\displaystyle\int_a^{b}g(x)dx\)收敛时,\(\displaystyle\int_a^{b}f(x)dx\)收敛
    \item 当\(\displaystyle\int_a^{b}f(x)dx\)发散时,\(\displaystyle\int_a^{b}g(x)dx\)发散
\end{itemize}

\subparagraph{比较判别法的极限形式}
设函数\(f(x), g(x)\)在区间\((a, b]\)上连续,瑕点同为\(x = a\),且\(f(x) >= 0, g(x) > 0, \displaystyle\lim_{x \to a^+}\dfrac{f(x)}{g(x)} = \lambda\)(有限或\(\infty\)),则
\begin{itemize}
    \item 当\(\lambda \neq 0\ \ \&\&\ \ \lambda \neq \infty\)时,\(\displaystyle\int_a^{b}f(x)dx\)与\(\displaystyle\int_a^{b}g(x)dx\)有相同的敛散性
    \item 当\(\lambda = 0\)时,若\(\displaystyle\int_a^{b}g(x)dx\)收敛,则\(\displaystyle\int_a^{b}f(x)dx\)也收敛
    \item 当\(\lambda = \infty\)时,若\(\displaystyle\int_a^{b}g(x)dx\)发散,则\(\displaystyle\int_a^{b}f(x)dx\)也发散
\end{itemize}


\paragraph{重要结论}

\begin{enumerate}
    \item \(\displaystyle\int_0^1\dfrac{1}{x^p}dx
    \begin{cases}
        \text{收敛,}0 < p < 1 \\ 
        \text{发散,}p >= 1
    \end{cases}\)对与x趋于0的“速度”相同的函数\(f(x)\)均适用,即\(x \to 0,\ f(x) \sim x\)
    \item \(\displaystyle\int_1^{+\infty}\dfrac{1}{x^p}dx
    \begin{cases}
        \text{收敛},\ p > 1 \\ 
        \text{发散},\ p <= 1
    \end{cases}\)
    \item \begin{itemize}
        \item 当\(f(x)\)偶函数且\(\int_0^{+\infty}f(x)dx\)收敛时,\(\displaystyle\int_{-\infty}^{+\infty}f(x)dx = 2\int_0^{+\infty}f(x)dx\)
        \item 当\(f(x)\)奇函数且\(\int_0^{+\infty}f(x)dx\)收敛时,\(\displaystyle\int_{-\infty}^{+\infty}f(x)dx = 0\)
    \end{itemize}
\end{enumerate}


\section{计算}

\subsection{分部积分}

\begin{displaymath}
\int_{}^{} u \,dv
= uv -
\int_{}^{} v \,du
\end{displaymath}


\subsection{表格积分法}

\begin{center}
\begin{tabular}{ c|c|c|c|c|c }
\(u\) & \(u'\) & \(u''\) & ... & \(u^{(n-1)}\) & \(u^{(n)}\) \\ 
\hline
\(v^{(n)}\) & \(v^{(n-1)}\) & \(v^{(n-2)}\) & ... & \(v'\) & \(v\)
\end{tabular}
\end{center}
u多次求导结果为0,u导n次,v积n次,交错相乘,正负交替

\begin{displaymath}
\int_{}^{} uv^{(n)} \,dx
= uv^{(n-1)} - u'v^{(n - 2)} + u''v^{(n - 3)} - ...
+(-1)^{n-1}u^{(n-1)}v + (-1)^{n}
\int_{}^{} u^{(n)}v \,dx
\end{displaymath}


\subsection{积分化为行列式}

\begin{displaymath}
\int_{}^{} xe^{ax} \,dx = 
\frac{1}{a^2}
\begin{vmatrix}
  x & e^{ax}\\ 
  (x)' & (e^{ax})'
\end{vmatrix}
+ C
\end{displaymath}

\begin{displaymath}
\int_{}^{} x\sin ax \,dx 
= \frac{1}{a^2}
\begin{vmatrix}
  (x)' & (\sin ax)' \\ 
  x & \sin ax
\end{vmatrix}
+ C
\end{displaymath}

\begin{displaymath}
\int_{}^{} e^{ax}\sin bx \,dx 
= \frac{1}{a^2 + b^2}
\begin{vmatrix}
  (e^{ax})' & (\sin bx)' \\ 
  e^{ax} & \sin bx
\end{vmatrix}
+ C
\end{displaymath}

\begin{displaymath}
\int_{}^{} e^{ax}\cos bx \,dx 
= \frac{1}{a^2 + b^2}
\begin{vmatrix}
  (e^{ax})' & (\cos bx)' \\ 
  e^{ax} & \cos bx
\end{vmatrix}
+ C
\end{displaymath}


\subsection{三角函数积分}
\begin{center}
\begin{tabular}{ c c c }
\hline
\(\displaystyle \int_{}^{} f(x) \,dx\) & \(f(x)\) & \(f'(x)\) \\
\hline
\(-\cos x + C\) & \(\sin x\) & \(\cos x\) \\
\(\sin x + C\) & \(\cos x\) & \(-\sin x\) \\
\(ln|\csc x - \cot x| + C\) & \(\csc x\) & \(-\csc\,x\,\cot\,x\) \\
\(ln|\sec x + \tan x| + C\) & \(\sec x\) & \(\sec\,x\,\tan\,x\) \\
\(ln|\sec x| + C = -ln|\cos x| + C\) & \(\tan x\) & \(\sec^{2}x\) \\
\(-ln|\csc x| + C = ln|\sin x| + C\) & \(\cot x\) & \(-\csc^{2}x\) \\
\(x\arcsin x + \sqrt{1 - x^{2}} + C\) & \(\arcsin x\) & \(\dfrac{1}{\sqrt{1 - x^{2}}}\) \\
\(x\arccos x - \sqrt{1 - x^{2}} + C\) & \(\arccos x\) & \(\dfrac{-1}{\sqrt{1 - x^{2}}}\) \\
\(x\arctan x - \dfrac{1}{2}ln|x^{2} + 1| + C\) & \(\arctan x\) & \(\dfrac{1}{1 + x^{2}}\) \\
null & \(arccsc x\) & \(\dfrac{-1}{|x|\sqrt{x^{2} - 1}}\) \\
null & \(arcsec x\) & \(\dfrac{1}{|x|\sqrt{x^{2} - 1}}\) \\
null & \(arccot x\) & \(\dfrac{-1}{1 + x^{2}}\) \\
\(\cosh x + C\) & \(\sinh x\) & \(\cosh x\) \\
\(\sinh\,x + C\) & \(\cosh\,x\) & \(\sinh\,x\) \\
\(x\,arcsinh\,x - \sqrt{x^{2} + 1} + C\) & \(arcsinh\,x\) & \(\dfrac{1}{\sqrt{x^{2} + 1}}\) \\
\(x\,arccosh\,x - \sqrt{x^{2} - 1} + C\) & \(arccosh\,x\) & \(\dfrac{1}{\sqrt{x^{2} - 1}}\) \\
\(\dfrac{-\sin\,2x + 2x}{4} + C\) & \(\sin^{2}x\) & \(\sin2x\) \\
\(\dfrac{\sin\,2x + 2x}{4} + C\) & \(\cos^{2}x\) & \(-\sin2x\) \\
\(\tan\,x - x + C\) & \(\tan^{2}x\) & null \\
\(-\cot\,x - x + C\) & \(\cot^{2}x\) & null \\
\(\dfrac{\sinh\,2x - 2x}{4} + C\) & \(\sinh^{2}x\) & \(\sinh2x\) \\
\(\dfrac{\sinh\,2x + 2x}{4} + C\) & \(\cosh^{2}x\) & \(\sinh2x\) \\
\hline
\end{tabular}
\end{center}

\[\int\sin^nxdx = -\dfrac{1}{n}\sin^{n - 1}x\cos x + \dfrac{n - 1}{n}\int\sin^{n - 2}xdx\]
\[\int\cos^nxdx = \dfrac{1}{n}\cos^{n - 1}x\sin x + \dfrac{n - 1}{n}\int\cos^{n - 2}xdx\]


\subsection{根号积分}
\begin{flalign}
    \int_{}^{} \dfrac{1}{x^{2} - a^{2}} \,dx & =
\dfrac{1}{2a}\ln|\dfrac{x - a}{x + a}| + C \nonumber \\ 
    \int_{}^{} \dfrac{1}{a^{2} - x^{2}} \,dx & =
\dfrac{1}{2a}\ln|\dfrac{x + a}{x - a}| + C \nonumber \\ 
    \int_{}^{} \dfrac{1}{x^{2} + a^{2}} \,dx & =
\dfrac{1}{a}\arctan\dfrac{x}{a} + C \nonumber \\ 
    \int_{}^{} \dfrac{1}{\sqrt{a^{2} - x^{2}}} \,dx & =
\arcsin\dfrac{x}{a} + C \nonumber \\ 
    \int_{}^{} \dfrac{1}{\sqrt{x^{2} + a^{2}}} \,dx & = 
\ln(\sqrt{x^{2} + a^{2}} + x) + C \nonumber \\ 
    \int_{}^{} \dfrac{1}{\sqrt{x^{2} - a^{2}}} \,dx & = 
\ln|\sqrt{x^{2} - a^{2}} + x| + C \nonumber \\ 
    \int_{}^{} \sqrt{x^{2} + a^{2}} \,dx & =
\frac{1}{2}(x\sqrt{x^{2} + a^{2}} + a^{2}\arcsin h\frac{x}{a}) + C \nonumber \\ 
    \int_{}^{} \sqrt{x^{2} - a^{2}} \,dx & =
\frac{1}{2}(x\sqrt{x^{2} - a^{2}} - a^{2}\arccos h\frac{x}{a}) + C \nonumber \\ 
    \int_{}^{} \sqrt{a^{2} - x^{2}} \,dx & =
\frac{1}{2}(x\sqrt{a^{2} - x^{2}} + a^{2}\arcsin\frac{x}{a}) + C \nonumber 
\end{flalign}

被积函数中含有形如\(\sqrt[n]{\dfrac{ax + b}{cx + d}}\)的根式的积分\begin{enumerate}
    \item 令\(t = \sqrt[n]{\dfrac{ax + b}{cx + d}}\)
    \item \(x = \dfrac{b - t^nd}{t^nc - a}\)
    \item \(dx = \dfrac{n(ad - bc)t^{n - 1}}{(t^nc - a)^2}dt\)
\end{enumerate}



\subsection{凑积分}
\[\int f[g(x)]g'(x)dx = \int f[g(x)]d[g(x)] = \int f(u)du\]

\begin{flalign}
    \int xf(x^2)dx & = \dfrac{1}{2}\int f(x^2)d(x^2) \nonumber \\ 
    \int\sqrt{x}f(x^{\frac{3}{2}})dx & = \dfrac{2}{3}\int f(x^{\frac{3}{2}})d(x^{\frac{3}{2}}) \nonumber \\ 
    \int\dfrac{f(\sqrt{x})}{\sqrt{x}}dx & = 2\int f(\sqrt{x})d(\sqrt{x}) \nonumber \\ 
    \int\dfrac{f(-\dfrac{1}{x})}{x^2}dx & = \int f(-\dfrac{1}{x})d(-\dfrac{1}{x}) \nonumber \\ 
    \int\dfrac{f(\ln x)}{x}dx & = \int f(\ln x)d(\ln x) \nonumber \\ 
    \int e^xf(e^x)dx & = \int f(e^x)d(e^x) \nonumber \\ 
    \int a^xf(a^x)dx & = \dfrac{1}{\ln a}\int f(a^x)d(a^x) \nonumber \\ 
    \int \sin xf(-\cos x)dx & = \int f(-\cos x)d(-\cos x) \nonumber \\ 
    \int\cos xf(\sin x)dx & = \int f(\sin x)d(\sin x) \nonumber \\ 
    \int\dfrac{f(\tan x)}{\cos^2 x}dx & = \int f(\tan x)d(\tan x) \nonumber \\ 
    \int \dfrac{f(-\cot x)}{\sin^2x}dx & = \int f(-\cot x)d(-\cot x) \nonumber \\ 
    \int\dfrac{f(\arctan x)}{1 + x^2}dx & = \int f(\arctan x)d(\arctan x) \nonumber \\ 
    \int\dfrac{f(\arcsin x)}{\sqrt{1 - x^2}}dx & = \int f(\arcsin x)d(\arcsin x) \nonumber
\end{flalign}


\subsection{换元}
\begin{enumerate}
    \item 三角函数代换,\(a > 0\)\[\begin{cases}
        \sqrt{a^2 - x^2}\rightarrow x = a\sin t,\ |t| < \dfrac{\pi}{2} \\ 
        \sqrt{a^2 + x^2}\rightarrow x = a\tan t,\ |t| < \dfrac{\pi}{2} \\ 
        \sqrt{x^2 - a^2}\rightarrow x = a\sec t,\ \begin{cases}
            x > 0,\ 0 < t < \dfrac{\pi}{2} \\ 
            x < 0,\ \dfrac{\pi}{2} < t < \pi
        \end{cases}
    \end{cases}\]
    \item 恒等变形后三角函数代换
    \item 根式代换
    \item 倒代换
    \item 复杂函数直接代换
\end{enumerate}


\subsection{分段函数积分}
\(f(x)\)连续,则必存在原函数\(F(x) = \displaystyle\int f(x)dx\)连续

\subsubsection{分段函数的变限积分}
对每一分段求变限积分。


\subsection{有理函数积分}

\subsubsection{定义}
形如\(\displaystyle\int\dfrac{P_n(x)}{Q_m(x)}dx,\ (n < m)\)的积分称为有理函数积分,其中\(P_n(x), Q_m(x)\)分别为x的n次多项式和m次多项式

\subsubsection{思想}
若\(Q_m(x)\)在实数域内可因式分解,则因式分解后再把\(\dfrac{P_n(x)}{Q_m(x)}\)拆成若干项最简有理分式之和(每个分式为真分式)

\subsubsection{方法}
\begin{enumerate}
    \item \(Q_m(x)\)的一次单因式\(ax + b\)产生一项\(\dfrac{A}{ax + b}\)
    \item \(Q_m(x)\)的k重一次因式\((ax + b)^k\)产生k项,分别为\(\dfrac{A_1}{ax + b}, ..., \dfrac{A_k}{(ax + b)^k}\)
    \item \(Q_m(x)\)的二次单因式\(px^2 + qx + r\)产生一项\(\dfrac{Ax + B}{px^2 + qx + r}\)\ ,(\(q^2 - 4pr < 0\))
    \item \(Q_m(x)\)的k重二次因式\((px^2 + qx + r)^k\)产生k项\(\dfrac{A_ix + B_i}{(px^2 + qx + r)^i}\)
    \item 求出对应的系数
\end{enumerate}

\subsubsection{求待定系数}
\begin{itemize}
    \item 通分后展开,左右x的同次幂的系数相等
    \item 在恒等式中赋予x适当的值,得到简单条件
    \item 留数法:左右同乘某个因子,再令该因子为0,代入对应x
\end{itemize}


\subsection{定积分计算}
\[\int_a^bf(x)dx = F(x)\bigg|_a^b = F(b) - F(a)\]

若\(f(x)\)在\([a, b]\)上分段有函数,如\([a, c)\)上有原函数\(F_1(x)\),\((c, b]\)上有原函数\(F_2(x)\),则\[\int_a^bf(x)dx = \int_a^cf(x)dx + \int_c^bf(x)dx = F_1(c - 0) - F_1(a) + F_2(b) - F_2(c + 0)\]
\begin{itemize}
    \item 若\(F_1(c - 0), F_2(c + 0)\)均存在,则\(\displaystyle\int_a^bf(x)dx\)收敛
    \item 若\(F_1(c - 0), F_2(c + 0)\)至少一个不存在,则\(\displaystyle\int_a^bf(x)dx\)发散
\end{itemize}

\subsubsection{结论}
\begin{itemize}
    \item 设\(f(x)\)连续偶函数,则\[\int_{-a}^af(x)dx = 2\int_0^af(x)dx\]
    \item 设\(f(x)\)连续奇函数,则\[\int_{-a}^af(x)dx = 0\]
    \item 设\(f(x)\)为以T为周期的连续函数,则对任意实数\(a\)有\[\int_a^{a + T}f(x)dx = \int_0^Tf(x)dx\]
    \item 设\(f(x)\)连续函数,则\[\int_a^bf(x)dx = \int_a^bf(a + b - x)dx\]
\end{itemize}


\subsubsection{Wallis公式}
\begin{itemize}
    \item \(\displaystyle\int_0^{\frac{\pi}{2}}\sin^nxdx = \int_0^{\frac{\pi}{2}}\cos^nxdx = \begin{cases}
        \dfrac{n - 1}{n} * \dfrac{n - 3}{n - 2} * ... * \dfrac{2}{3} * 1,\ \ n\text{为大于1的奇数} \\ 
        \dfrac{n - 1}{n} * \dfrac{n - 3}{n - 2} * ... * \dfrac{1}{2} * \dfrac{\pi}{2},\ \ n\text{为正偶数}
    \end{cases}\)
    \item \(\displaystyle\int_0^\pi\sin^nxdx = \begin{cases}
        2 * \dfrac{n - 1}{n} * \dfrac{n - 3}{n - 2} * ... * \dfrac{2}{3} * 1,\ \ n\text{为大于1的奇数} \\ 
        2 * \dfrac{n - 1}{n} * \dfrac{n - 3}{n - 2} * ... * \dfrac{1}{2} * \dfrac{\pi}{2},\ \ n\text{为正偶数}
    \end{cases}\)
    \item \(\displaystyle\int_0^\pi\cos^nxdx = \begin{cases}
        0, \ \ n\text{为正奇数} \\ 
        2 * \dfrac{n - 1}{n} * \dfrac{n - 3}{n - 2} * ... * \dfrac{1}{2} * \dfrac{\pi}{2},\ \ n\text{为正偶数}
    \end{cases}\)
    \item \(\displaystyle\int_0^{2\pi}\cos^nxdx = \int_0^{2\pi}\sin^nxdx = \begin{cases}
        0,\ \  n\text{为正奇数} \\ 
        4 * \dfrac{n - 1}{n} * \dfrac{n - 3}{n - 2} * ... * \dfrac{1}{2} * \dfrac{\pi}{2},\ \ n\text{为正偶数}
    \end{cases}\)
    \item \(\displaystyle\int_0^{\frac{\pi}{2}}\sin^mx\cos^nxdx = \begin{cases}
        \dfrac{(m - 1)!!(n - 1)!!}{(m + n)!!} * \dfrac{\pi}{2},\ \ \ \text{m,n都是偶数} \\ 
        \dfrac{(m - 1)!!(n - 1)!!}{(m + n)!!},\ \ \ \ \ \text{else}
    \end{cases}\)
\end{itemize}


\subsection{变限积分计算}
\subsubsection{求导公式}
设\(F(x) = \displaystyle\int_{\varphi_1(x)}^{\varphi_2(x)}f(t)dt\),其中\(f(t)\)在\([a, b]\)上连续,可导函数\(\varphi_1(x), \varphi_2(x)\)的值域在\([a, b]\)上,则在函数\(\varphi_1(x), \varphi_2(x)\)的公共定义域上,有\[F'(x) = \dfrac{d}{dx}[\int_{\varphi_1(x)}^{\varphi_2(x)}f(t)dt] = f[\varphi_2(x)]\varphi_2'(x) - f[\varphi_1(x)]\varphi_1'(x)\]

\subsubsection{结论}
\begin{itemize}
    \item 只有被积函数可积,就可有变限积分相关性质
    \item 只有被积函数连续,才能谈原函数相关性质
\end{itemize}
\begin{enumerate}
    \item \(f(x)\)为可积奇函数\(\Rightarrow \begin{cases}
        \int_0^xf(t)dt\text{为偶函数} \\ 
        \int_a^xf(t)dt\text{为偶函数}
    \end{cases}\)
    \item \(f(x)\)为可积偶函数\(\Rightarrow \begin{cases}
        \int_0^xf(t)dt\text{为奇函数} \\ 
        \int_a^xf(t)dt\ (a \neq 0) \begin{cases}
            \text{若}\int_a^xf(t)dt = \int_0^xf(t)dt\text{,奇函数} \\ 
            \text{若}\int_a^xf(t)dt \neq \int_0^xf(t)dt\text{,非奇非偶函数}
        \end{cases}
    \end{cases}\)
    \item \(f(x)\)可积且以T为周期,则\(\int_0^xf(t)dt\)是以T为周期的周期函数\(\Leftrightarrow \int_0^Tf(x)dx = 0\)
\end{enumerate}


\subsection{反常积分计算}
注意识别奇点(端点、内部)


\subsection{\(\Gamma\)函数}
\subsubsection{定义}
\[\Gamma(\alpha) = \int_0^{+\infty}x^{\alpha - 1}e^{-x}dx \xrightarrow{x = t^2}2\int_0^{+\infty}t^{2\alpha - 1}e^{-t^2}dt\ (x, t > 0)\]

\subsubsection{递推}
\begin{flalign}
    \Gamma(\alpha + 1) & = \int_0^{+\infty}x^{\alpha}e^{-x}dx = -\int_0^{+\infty}x^{\alpha}d(e^{-x}) \nonumber \\ 
    & = -x^{\alpha}e^{-x}\bigg|_0^{+\infty} + \int_0^{+\infty}e^{-x}\alpha x^{\alpha - 1}dx = \alpha\Gamma(\alpha) \nonumber
\end{flalign}

其中\(\Gamma(1) = 1, \Gamma(\dfrac{1}{2}) = \sqrt{\pi}\),
\(\therefore\ \Gamma(n + 1) = n!,\ \Gamma(2) = 1,\ \Gamma(\dfrac{5}{2}) = \dfrac{3}{2} * \dfrac{1}{2} * \Gamma(\dfrac{1}{2}) = \dfrac{3}{4}\sqrt{\pi}\)


\subsection{高斯积分}
\[\displaystyle\int_0^{+\infty}e^{-x^2}dx = \dfrac{\sqrt{\pi}}{2}\]






\section{二重积分}

\subsection{概念}
\[\iint_Df(x, y)d\sigma = \lim_{\lambda \to 0}\sum_{i = 1}^nf(\xi_i, \eta_i)\Delta\sigma_i\]
\(f(x, y)\)被积函数,\(f(x, y)d\sigma\)被积表达式,\(d\sigma( > 0)\)面积元素,D积分区域,\(\displaystyle\sum_{i = 1}^nf(\xi_i, \eta_i)\Delta\sigma_i\)积分和。

若\(f(x, y)\)在有界闭区间D上连续,则二重积分\(\displaystyle\iint_Df(x, y)d\sigma\)一定存在。


\subsection{性质}

\paragraph{求区域面积}
\(\displaystyle\iint_D1d\sigma = \iint_Dd\sigma = A\),A为D的面积;

\paragraph{可积函数必有界}
当\(f(x, y)\)在有界闭区间D上可积时,\(f(x, y)\)在D上必有界;

\paragraph{积分的线性性质}
设\(k_1, k_2\)常数,则\[\iint_D[k_1f(x, y) \pm k_2g(x, y)]d\sigma = k_1\iint_Dfd\sigma \pm k_2\iint_Dgd\sigma\]

\paragraph{积分的可加性}
设\(f(x, y)\)在有界闭区间D上可积,且\(D_1\cup D_2 = D,\ D_1 \cap D_2 = \emptyset\),则
\[\iint_Dfd\sigma = \iint_{D_1}fd\sigma + \iint_{D_2}fd\sigma\]

\paragraph{积分的保号性}
当\(f, g\)在有界闭区间D上可积时,若D上有\(f(x, y) <= g(x, y)\),则有\[\iint_Dfd\sigma <= \iint_Dgd\sigma\]
\[|\iint_Df(x, y)d\sigma| <= \iint_D|f(x, y)|d\sigma\]

\paragraph{二重积分估值定理}
设\(M, m\)分别是\(f\)在有界闭区间D上的最大值和最小值,A为D的面积,则有
\[mA <= \iint_Dfd\sigma <= MA\]

\paragraph{二重积分中值定理}
设\(f(x, y)\)在有界闭区间D上连续,A为D的面积,则D上至少存在一点\((\xi, \eta)\),使得
\[\iint_Df(x, y)d\sigma = f(\xi, \eta)A\]


\subsection{普通对称性}
\begin{itemize}
    \item 若D关于y轴对称,则\[\iint_Df(x, y)d\sigma = \begin{cases}
        2\iint_{D_1}f(x, y)d\sigma,\ f(x, y) = f(-x, y) \\
        0,\ \ f(x, y) = -f(-x, y)
    \end{cases}\]

    \item 若D关于\(x = a\)对称,则\[\iint_Df(x, y)d\sigma = \begin{cases}
        2\iint_{D_1}f(x, y)d\sigma,\ f(x, y) = f(2a - x, y) \\ 
        0,\ \ f(x, y) = -f(2a - x, y)
    \end{cases}\]

    \item 若D关于x轴对称,则\[\iint_Df(x, y)d\sigma = \begin{cases}
        2\iint_{D_1}f(x, y)d\sigma,\ f(x, y) = f(x, -y) \\ 
        0,\ \ f(x, y) = -f(x, -y)
    \end{cases}\]

    \item 若D关于\(y = a\)对称,则\[\iint_Df(x, y)d\sigma = \begin{cases}
        2\iint_{D_1}f(x, y)d\sigma,\ f(x, y) = f(x, 2a - y) \\ 
        0,\ \ f(x, y) = -f(x, 2a - y)
    \end{cases}\]

    \item 若D关于原点对称,则\[\iint_Df(x, y)d\sigma = \begin{cases}
        2\iint_{D_1}f(x, y)d\sigma,\ f(x, y) = f(-x, -y) \\ 
        0,\ \ f(x, y) = -f(-x, -y)
    \end{cases}\]

    \item 若D关于\(y = x\)对称,则\[\iint_Df(x, y)d\sigma = \begin{cases}
        2\iint_{D_1}f(x, y)d\sigma,\ f(x, y) = f(y, x) \\ 
        0,\ \ f(x, y) = -f(y, x)
    \end{cases}\]
\end{itemize}


\subsection{轮换对称性}
在直角坐标系下,把x与y对调后,区域D不变(或关于\(y = x\)对称),则
\[\iint_Df(x, y)d\sigma = \iint_Df(y, x)d\sigma\]

在直角坐标系下,若\(f(x, y) + f(y, x) =(>)\ a\),则
\[I = \dfrac{1}{2}\iint_D[f(x, y) + f(y, x)]dxdy =(>)\ \dfrac{1}{2}\iint_Dadxdy = \dfrac{a}{2}S_D\]


\subsection{计算}






