
\chapter{几何应用}
\[\begin{cases}
\text{极值定义} \\ 
\text{单调性与极值的判别}\begin{cases}
\text{单调性判别} \\ 
\text{一阶可导点} \\ 
\text{判断极值的充分条件}\end{cases} \\ 
\text{凹凸点与拐点}\begin{cases}
\text{凹凸点定义} \\ 
\text{拐点定义}\end{cases} \\ 
\text{凹凸点与拐点的判别}\begin{cases}
\text{判断凹凸性} \\ 
\text{二阶可导点} \\ 
\text{判断拐点的充分条件}\end{cases} \\ 
\text{极值点与拐点的重要结论} \\ 
\text{渐近线}\begin{cases}
\text{铅直渐近线} \\ 
\text{水平渐近线} \\ 
\text{斜渐近线}\end{cases}
\end{cases}\]

\section{极值}

间断点可以是极值点

\subsection{必要条件}
一阶可导点

设\(f(x)\)在\(x = x_0\)处可导,且在点\(x_0\)处取得极值,则必有\(f'(x_0) = 0\)

\subsection{第一充分条件}
设\(f(x)\)在\(x = x_0\)处连续,且在\(x_0\)的某去心邻域\(U(x_0, \delta)\)内可导,\begin{itemize}
    \item 若\(x \in (x_0 - \delta, x_0)\)时,\(f'(x) < 0\),而\(x \in (x_0, x_0 + \delta)\)时,\(f'(x) > 0\),则\(f(x)\)在\(x = x_0\)处取得极小值;
    \item 若\(x \in (x_0 - \delta, x_0)\)时,\(f'(x) > 0\),而\(x \in (x_0, x_0 + \delta)\)时,\(f'(x) < 0\),则\(f(x)\)在\(x = x_0\)处取得极大值;
\end{itemize}


\subsection{第二充分条件}
设\(f(x)\)在\(x = x_0\)处二阶可导,且\(f'(x_0) = 0, f''(x_0) \neq 0\)\begin{itemize}
    \item 若\(f''(x_0) < 0\),则\(f(x)\)在\(x_0\)处取得极大值;
    \item 若\(f''(x_0) > 0\),则\(f(x)\)在\(x_0\)处取得极小值;
\end{itemize}


\subsection{第三充分条件}
设\(f(x)\)在\(x = x_0\)处n阶可导,且\(f^{(m)}(x_0) = 0, (m = n - 1), f^{(n)}(x_0) \neq 0, (n >= 2)\)则\begin{itemize}
    \item 当n为偶数且\(f^{(n)}(x_0) < 0\)时,\(f(x)\)在\(x_0\)处取极大值;
    \item 当n为偶数且\(f^{(n)}(x_0) > 0\)时,\(f(x)\)在\(x_0\)处取极小值;
\end{itemize}


\section{单调性}


\section{驻点}
一阶导数为零的点;


\section{凹凸性}

\subsection{定义}
设\(f(x)\)在区间\(I\)上连续,若任意\(x_1, x_2 \in I\),恒有\[f(\dfrac{x_1 + x_2}{2}) < \dfrac{f(x_1) + f(x_2)}{2}\]则称\(f(x)\)在区间\(I\)上的图形是凹的;若恒有
\[f(\dfrac{x_1 + x_2}{2}) > \dfrac{f(x_1) + f(x_2)}{2}\]
则称\(f(x)\)在区间\(I\)上的图形是凸的。


\subsection{定义2}
设\(f(x)\)在\([a, b]\)上连续,在\((a, b)\)内可导,若任意\(x, x_0 \in (a, b), x \neq x_0\)恒有\[f(x_0) + f'(x_0)(x - x_0) <(>) f(x)\]则称\(f(x)\)在\([a, b]\)上的图形是凹(凸)的;


\subsection{几何意义}
\(y = f(x_0) + f'(x_0)(x - x_0)\)是\(f(x)\)在点\((x_0, f(x_0)\)处的切线方程


\subsection{判别}
设\(f(x)\)在区间\(I\)上二阶可导\begin{itemize}
    \item 若在\(I\)上\(f''(x) > 0\),则f(x)在\(I\)上图形是凹的;
    \item 若在\(I\)上\(f''(x) < 0\),则f(x)在\(I\)上图形是凸的;
\end{itemize}


\section{拐点}
\subsection{定义}
连续曲线的凹弧与凸弧的分界点为拐点;

\subsection{必要条件}
设\(f''(x_0)\)存在,且点\((x_0, f(x_0))\)为曲线的拐点,则\(f''(x_0) = 0\)


\subsection{第一充分条件}
设\(f(x)\)在点\(x = x_0\)连续,在点\(x = x_0\)的某去心邻域\(U(x_0, \delta)\)内二阶导数存在,且该点的左右邻域内\(f''(x)\)变号,则点\(x = x_0\)为曲线的拐点。


\subsection{第二充分条件}
设\(f(x)\)在点\(x = x_0\)的某邻域内三阶可导,且\(f''(x_0) = 0, f'''(x_0) \neq 0\),则点\((x_0, f(x_0))\)为曲线的拐点;


\subsection{第三充分条件}
设\(f(x)\)在\(x_0\)处n阶可导且\(f^{(m)}(x_0) = 0, (m = 2, ... n - 1), f^{(n)}(x_0) \neq 0(n >= 3)\),则当n为奇数时,点\((x_0, f(x_0))\)为曲线的拐点;


\subsection{极值点与拐点重要结论}
\begin{itemize}
    \item 曲线的可导点不可同时为极值点和拐点;不可导点可同时为极值点和拐点;
    \item 设多项式函数\(f(x) = (x - a)^ng(x), (n > 1), g(a) \neq 0\),当n为偶数时,x = a是\(f(x)\)的极值点;当n为奇数时,点\((a, 0)\)是曲线\(f(x)\)的拐点;
    \item 设多项式函数\(f(x) = (x - a_1)^{n_1}(x - a_2)^{n_2}...(x - a_k)^{n_k}\),其中\(n_i\)正整数,\(a_i\)实数且两两不等;记\(k_1\)为\(n_i = 1\)的个数,\(k_2\)为\(n_i > 1, n_i\)为偶数的个数,\(k_3\)为\(n_i > 1, n_i\)为奇数的个数,则\(f(x)\)的极值点个数为\(k_1 + 2k_2 + k_3 - 1\),拐点数为\(k_1 + 2k_2 + 3k_3 - 2\);
\end{itemize}


\section{渐近线}

\subsection{铅直渐近线}
若\(\displaystyle\lim_{x \to x_0^+}f(x) = \infty(\lim_{x \to x_0^-}f(x) = \infty)\),则\(x = x_0\)为铅直渐近线;

\subsection{水平渐近线}
若\(\displaystyle\lim_{x \to +\infty}f(x) = y_1\),则\(y = y_1\)为一条水平渐近线;若\(\displaystyle\lim_{x \to -\infty}f(x) = y_2\),则\(y = y_2\)为一条水平渐近线;

若\(\displaystyle\lim_{x \to +\infty}f(x) = \lim_{x \to -\infty}f(x) = y_0\),则\(y = y_0\)为一条水平渐近线;

\subsection{斜渐近线}
若\(\displaystyle\lim_{x \to +(-)\infty}\dfrac{f(x)}{x} = a_1, \lim_{x \to +(-)\infty}[f(x) - a_1x] = b_1\),则\(y = a_1x + b_1\)是曲线\(y = f(x)\)的一条斜渐近线;

若\(\displaystyle\lim_{x \to +\infty}\dfrac{f(x)}{x} = \lim_{x \to -\infty}\dfrac{f(x)}{x} = a, \lim_{x \to +\infty}[f(x) - a_1x] = \lim_{x \to -\infty}[f(x) - a_1x] = b\),则\(y = ax + b\)是一条斜渐近线;

\subsection{顺序}
\begin{enumerate}
    \item 铅直,水平,斜
    \item 找函数无定义点,区间端点,函数分段点;
    \item 判断\(\displaystyle\lim_{x \to x_0^+}f(x) = \infty, \text{判断}\lim_{x \to \infty}f(x)\)是否常数,若\(\infty\),判断\(\displaystyle\lim_{x \to \infty}\dfrac{f(x)}{x}\)是否为非零常数;
\end{enumerate}


\section{最值}
若\(f(x)\)在区间\(I\)上有最值点\(x_0\),\(x_0\)不是端点,则是极值点;

\paragraph{闭区间上最值}
函数驻点与不可导点,端点;

\paragraph{开区间上最值}
函数驻点与不可导点,两端单侧极限,


\section{函数图像}

\subsection{斜率}
f(x)在点\(x_0\)处的导数值\(f'(x_0)\)是曲线\(y = f(x)\)在点\((x_0, y_0)\)处切线的斜率k,即\(k = f'(x_0)\),切线方程为\(y - y_0 = f'(x_0)(x - x_0)\),法线方程为\(y - y_0 = -\dfrac{1}{f'(x_0)}(x - x_0), (f'(x_0) \neq 0)\)

\subsection{倾角}
设\(y(x)\)具有二阶导数,记\(\alpha\)为\(y(x)\)在点(x, y)处切线的倾角,则\[y' = tan\,\alpha\]


\section{曲率}
设\(y(x)\)二阶可导,则曲线\(y = y(x)\)在点\((x, y(x))\)处曲率为\(k = \dfrac{|y''|}{[1 + (y')^2]^{\frac{3}{2}}}\),曲率半径的计算公式为\(R = \dfrac{1}{k} = \dfrac{[1 + (y')^2]^{\frac{3}{2}}}{|y''|}\)


\subsection{曲率}
y(x)在任意点的曲率\(K = \dfrac{|y''(x)|}{(1 + y'^2(x))^{\frac{3}{2}}}\)

\subsection{曲率半径}
\[\rho = \dfrac{1}{K}\]

\subsection{曲率中心坐标}
\[\alpha = x - \dfrac{y'(1 + y'^2)}{y''}\]
\[\beta = y + \dfrac{1 + y'^2}{y''}\]

\subsection{曲率圆方程}
\[(x - \alpha)^2 + (y - \beta)^2 = \rho^2\]



\section{积分}

\subsection{定积分表示计算旋转体体积}

曲线\(y=f(x)\)与\(x=a,x=b(a < b)\)及x轴所围成的曲边梯形绕x轴旋转一周所得到的旋转体的体积
\begin{displaymath}
V_{x} = \pi \int_{a}^{b} f^{2}(x) \,dx
\end{displaymath}

曲线\(y = f(x)\)与直线\(y = h\)围成的区域绕直线\(y = h\)旋转一周所得旋转体的体积为\[V = \pi\int_a^b[f(x) - h]^2dx\]
\[V = 2\pi\int_q^p(y - h)x(y)dy\]

曲线\(y=f(x)\)与\(x=a,x=b(0 \leq a \leq b)\)及x轴所围成的曲边梯形绕y轴旋转一周所得到的旋转体的体积
\begin{displaymath}
V_{y} = 2\pi \int_{a}^{b} x \lvert f(x) \rvert \,dx
\end{displaymath}

曲线\(y = f(x)\)绕直线\(x = h\)旋转一周所得旋转体的体积为\[V = 2\pi\int_a^b|x - h||f(x)|dx\]

设平面曲线\(L_1 : y = f(x), a \leq x \leq b, f(x)\)可导, \\
定直线\(L : Ax + By + C = 0\),且过\(L\)的任一条垂线与\(L_1\)至多有一个交点,则\(L_1\)绕L旋转一周所得旋转体体积为
\begin{displaymath}
V = \frac{\pi}{(A^2 + B^2)^{\frac{3}{2}}}
\int_{a}^{b} (Ax + Bf(x) + C)^2 \lvert Af'(x) - B \rvert \,dx
\end{displaymath}
其中a,b为过\(L\)的垂线与\(L_1\)的端点的交点的横坐标.

\paragraph{极坐标}
平面图形\(D = \{(\rho, \theta) | 0 <= \rho <= \rho(\theta), \theta \in [a, b] \subset [0, \pi]\}\)绕极轴一周所得旋转体体积为\[V = \dfrac{2\pi}{3}\int_a^b\rho^3(\theta)\sin\theta d\theta\]


\subsection{定积分表示计算函数平均值}

设\(x \in [a, b]\),函数\(y = f(x)\)在[a,b]上平均值为\(\overline{y} = \frac{1}{b- a} \int_{a}^{b} f(x) \,dx\)


\subsection{平面形心坐标公式}

设平面区域\(D = \{ (x, y) | 0 \leq y \leq f(x), a \leq x \leq b \}, y = f(x)\)在[a, b]上连续,\(S_D\)为D的面积,则D的形心坐标\((\overline{x}, \overline{y})\)公式为

\begin{displaymath}
\overline{x} = \frac{\iint_{D} x \,d\sigma}{S_D} = 
\frac{\int_{a}^{b} \,dx \int_{0}^{f(x)} x \,dy}{S_D} =
\frac{\int_{a}^{b} xf(x) \,dx}{S_D}
\end{displaymath}

\begin{displaymath}
\overline{y} = \frac{\iint_{D} y \,d\sigma}{S_D} = 
\frac{\int_{a}^{b} \,dx \int_{0}^{f(x)} y \,dy}{S_D} =
\frac{\frac{1}{2}\int_{a}^{b} f^2(x) \,dx}{S_D}
\end{displaymath}
\[S_D = \iint_{D}d\sigma = \int_{a}^{b}dx \int_{0}^{f(x)}dy = \int_{a}^{b} f(x)\,dx\]


\subsection{平面曲线弧长}

\begin{displaymath}
由直角坐标方程y = f(x)(a \leq x \leq b)给出,
则s = \int_{a}^{b} \sqrt{1 + (f'(x))^2} \,dx
\end{displaymath}

\begin{displaymath}
由参数方程
\begin{cases}
x = x(t) \\
y = y(t)
\end{cases}
(\alpha \leq t \leq \beta)给出,
则s = \int_{\alpha}^{\beta} \sqrt{[x'(t)]^2 + [y'(t)]^2} \,dt
\end{displaymath}

\begin{displaymath}
由极坐标方程\rho = \rho(\theta)(\alpha \leq \theta \leq \beta)给出,
则s = \int_{\alpha}^{\beta} \sqrt{[\rho(\theta)]^2 + [\rho'(\theta)]^2} \,d\theta
\end{displaymath}


\subsection{旋转曲面侧面积}

曲线\(L : y = f(x), a \leq x \leq b\)绕x轴一周所得旋转曲面面积
\begin{displaymath}
S = 2\pi \int_{a}^{b}|y|\sqrt{1 + (y_x')^2} \,dx
\end{displaymath}

曲线
\(
L : 
\begin{cases}
x = x(t) \\
y = y(t)
\end{cases},
\alpha \leq t \leq \beta, x'(t) \neq 0
\),
绕x轴一周所得旋转曲面面积
\begin{displaymath}
S = 2\pi \int_{\alpha}^{\beta}|y(t)|\sqrt{(x_t')^2 + (y_t')^2} \,dt
\end{displaymath}

曲线\(L : \rho = \rho(\theta), \alpha \leq \theta \leq \beta\)绕x轴一周所得旋转曲面面积
\begin{displaymath}
S = 2\pi \int_{\alpha}^{\beta}|\rho(\theta)sin\theta|
\sqrt{[\rho(\theta)]^2 + [\rho'(\theta])^2} \,d\theta
\end{displaymath}


\subsection{平面截面面积为已知的立体体积}

在区间[a, b]上, 垂直于x轴的平面截立体\(\Omega\)所得到的截面面积为x的连续函数A(x), 则\(\Omega\)的体积为
\begin{displaymath}
V = \int_{a}^{b} A(x) \,dx
\end{displaymath}


\subsection{平面图形面积}

极坐标下\(r(\theta), (a <= \theta <= b)\)所围成的面积为\[\dfrac{1}{2}\int_a^br^2d\theta\]







