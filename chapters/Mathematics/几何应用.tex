
\chapter{几何应用}
\[\begin{cases}
\text{极值定义} \\ 
\text{单调性与极值的判别}\begin{cases}
\text{单调性判别} \\ 
\text{一阶可导点} \\ 
\text{判断极值的充分条件}\end{cases} \\ 
\text{凹凸点与拐点}\begin{cases}
\text{凹凸点定义} \\ 
\text{拐点定义}\end{cases} \\ 
\text{凹凸点与拐点的判别}\begin{cases}
\text{判断凹凸性} \\ 
\text{二阶可导点} \\ 
\text{判断拐点的充分条件}\end{cases} \\ 
\text{极值点与拐点的重要结论} \\ 
\text{渐近线}\begin{cases}
\text{铅直渐近线} \\ 
\text{水平渐近线} \\ 
\text{斜渐近线}\end{cases}
\end{cases}\]

\section{极值}

间断点可以是极值点

\paragraph{必要条件}
一阶可导点

设\(f(x)\)在\(x = x_0\)处可导,且在点\(x_0\)处取得极值,则必有\(f'(x_0) = 0\)

\paragraph{第一充分条件}
设\(f(x)\)在\(x = x_0\)处连续,且在\(x_0\)的某去心邻域\(U(x_0, \delta)\)内可导,

()若\(x \in (x_0 - \delta, x_0)\)时,\(f'(x) < 0\),而\(x \in (x_0, x_0 + \delta)\)时,\(f'(x) > 0\),则\(f(x)\)在\(x = x_0\)处取得极小值;

()若\(x \in (x_0 - \delta, x_0)\)时,\(f'(x) > 0\),而\(x \in (x_0, x_0 + \delta)\)时,\(f'(x) < 0\),则\(f(x)\)在\(x = x_0\)处取得极大值;

()

\paragraph{第二充分条件}
设\(f(x)\)在\(x = x_0\)处二阶可导,且\(f'(x_0) = 0, f''(x_0) \neq 0\)

()若\(f''(x_0) < 0\),则\(f(x)\)在\(x_0\)处取得极大值;

()若\(f''(x_0) > 0\),则\(f(x)\)在\(x_0\)处取得极小值;

\paragraph{第三充分条件}
设\(f(x)\)在\(x = x_0\)处n阶可导,且\(f^{(m)}(x_0) = 0, (m = n - 1), f^{(n)}(x_0) \neq 0, (n >= 2)\)则

()当n为偶数且\(f^{(n)}(x_0) < 0\)时,\(f(x)\)在\(x_0\)处取极大值;

()当n为偶数且\(f^{(n)}(x_0) > 0\)时,\(f(x)\)在\(x_0\)处取极小值;


\section{单调性}

\section{驻点}
一阶导数为零的点;


\section{凹凸性}
\paragraph{定义}
设\(f(x)\)在区间\(I\)上连续,若任意\(x_1, x_2 \in I\),恒有\[f(\dfrac{x_1 + x_2}{2}) < \dfrac{f(x_1) + f(x_2)}{2}\]则称\(f(x)\)在区间\(I\)上的图形是凹的;若恒有
\[f(\dfrac{x_1 + x_2}{2}) > \dfrac{f(x_1) + f(x_2)}{2}\]
则称\(f(x)\)在区间\(I\)上的图形是凸的。

\paragraph{定义2}
设\(f(x)\)在\([a, b]\)上连续,在\((a, b)\)内可导,若任意\(x, x_0 \in (a, b), x \neq x_0\)恒有\[f(x_0) + f'(x_0)(x - x_0) <(>) f(x)\]则称\(f(x)\)在\([a, b]\)上的图形是凹(凸)的;

\paragraph{几何意义}
\(y = f(x_0) + f'(x_0)(x - x_0)\)是\(f(x)\)在点\((x_0, f(x_0)\)处的切线方程

\paragraph{判别}
设\(f(x)\)在区间\(I\)上二阶可导

()若在\(I\)上\(f''(x) > 0\),则f(x)在\(I\)上图形是凹的;

()若在\(I\)上\(f''(x) < 0\),则f(x)在\(I\)上图形是凸的;


\section{拐点}
\paragraph{定义}
连续曲线的凹弧与凸弧的分界点为拐点;

\paragraph{必要条件}
设\(f''(x_0)\)存在,且点\((x_0, f(x_0))\)为曲线的拐点,则\(f''(x_0) = 0\)

\paragraph{第一充分条件}
设\(f(x)\)在点\(x = x_0\)连续,在点\(x = x_0\)的某去心邻域\(U(x_0, \delta)\)内二阶导数存在,且该点的左右邻域内\(f''(x)\)变号,则点\(x = x_0\)为曲线的拐点。

\paragraph{第二充分条件}
设\(f(x)\)在点\(x = x_0\)的某邻域内三阶可导,且\(f''(x_0) = 0, f'''(x_0) \neq 0\),则点\((x_0, f(x_0))\)为曲线的拐点;

\paragraph{第三充分条件}
设\(f(x)\)在\(x_0\)处n阶可导且\(f^{(m)}(x_0) = 0, (m = 2, ... n - 1), f^{(n)}(x_0) \neq 0(n >= 3)\),则当n为奇数时,点\((x_0, f(x_0))\)为曲线的拐点;

\subsubsection{极值点与拐点重要结论}

()曲线的可导点不可同时为极值点和拐点;不可导点可同时为极值点和拐点;

()设多项式函数\(f(x) = (x - a)^ng(x), (n > 1), g(a) \neq 0\),当n为偶数时,x = a是\(f(x)\)的极值点;当n为奇数时,点\((a, 0)\)是曲线\(f(x)\)的拐点;

()设多项式函数\(f(x) = (x - a_1)^{n_1}(x - a_2)^{n_2}...(x - a_k)^{n_k}\),其中\(n_i\)正整数,\(a_i\)实数且两两不等;记\(k_1\)为\(n_i = 1\)的个数,\(k_2\)为\(n_i > 1, n_i\)为偶数的个数,\(k_3\)为\(n_i > 1, n_i\)为奇数的个数,则\(f(x)\)的极值点个数为\(k_1 + 2k_2 + k_3 - 1\),拐点数为\(k_1 + 2k_2 + 3k_3 - 2\);

\section{渐近线}

\paragraph{铅直渐近线}
若\(\displaystyle\lim_{x \to x_0^+}f(x) = \infty(\lim_{x \to x_0^-}f(x) = \infty)\),则\(x = x_0\)为铅直渐近线;

\paragraph{水平渐近线}
若\(\displaystyle\lim_{x \to +\infty}f(x) = y_1\),则\(y = y_1\)为一条水平渐近线;若\(\displaystyle\lim_{x \to -\infty}f(x) = y_2\),则\(y = y_2\)为一条水平渐近线;

若\(\displaystyle\lim_{x \to +\infty}f(x) = \lim_{x \to -\infty}f(x) = y_0\),则\(y = y_0\)为一条水平渐近线;

\paragraph{斜渐近线}
若\(\displaystyle\lim_{x \to +(-)\infty}\dfrac{f(x)}{x} = a_1, \lim_{x \to +(-)\infty}[f(x) - a_1x] = b_1\),则\(y = a_1x + b_1\)是曲线\(y = f(x)\)的一条斜渐近线;

若\(\displaystyle\lim_{x \to +\infty}\dfrac{f(x)}{x} = \lim_{x \to -\infty}\dfrac{f(x)}{x} = a, \lim_{x \to +\infty}[f(x) - a_1x] = \lim_{x \to -\infty}[f(x) - a_1x] = b\),则\(y = ax + b\)是一条斜渐近线;

\paragraph{顺序}
铅直,水平,斜

找函数无定义点,区间端点,函数分段点;

判断\(\displaystyle\lim_{x \to x_0^+}f(x) = \infty, \text{判断}\lim_{x \to \infty}f(x)\)是否常数,若\(\infty\),判断\(\displaystyle\lim_{x \to \infty}\dfrac{f(x)}{x}\)是否为非零常数;


\section{最值}
若\(f(x)\)在区间\(I\)上有最值点\(x_0\),\(x_0\)不是端点,则是极值点;

\paragraph{闭区间上最值}
函数驻点与不可导点,端点;

\paragraph{开区间上最值}
函数驻点与不可导点,两端单侧极限,


\section{函数图像}


\section{曲率}
设\(y(x)\)二阶可导,则曲线\(y = y(x)\)在点\((x, y(x))\)处曲率为\(k = \dfrac{|y''|}{[1 + (y')^2]^{\frac{3}{2}}}\),曲率半径的计算公式为\(R = \dfrac{1}{k} = \dfrac{[1 + (y')^2]^{\frac{3}{2}}}{|y''|}\)


