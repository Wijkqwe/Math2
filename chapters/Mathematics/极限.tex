
\chapter{极限}

\section{邻域}

\section{定义}

设f(x)在点\(x_0\)的某一去心邻域内有定义,若存在常数A,对于任意给定的\(\epsilon > 0\),总存在正数\(\delta\),使得当\(0 < |x - x_0| < \delta\)时,对应f(x)都满足不等式\(|f(x) - A| < \epsilon\),则A称为f(x)当\(x \to x_0\)时的极限,记作\(\lim_{x \to x_0}f(x) = A\)或\(f(x) \to A(x \to x_0)\),即
\begin{flalign}
    \lim_{x \to x_0}f(x) = A & \Leftrightarrow \forall \epsilon > 0, \exists \delta > 0\text{当}0 < |x - x_0| < \delta\text{时,有}|f(x) - A| < \epsilon \nonumber \\ 
    & \Leftrightarrow \text{在}x = x_0\text{的某去心邻域内}f(x)\text{存在} \nonumber
\end{flalign}
\[\]


\section{性质}

\subsection{唯一性}
若极限存在那么极限唯一
\begin{itemize}
    \item 对于\(x \to \infty\),意味着\(x \to +\infty\)且\(x \to -\infty\)
    \item 对于\(x \to x_0\), 意味着\(x \to x_0^+\)且\(x \to x_0^-\)
\end{itemize}


极限存在\textbf{充要条件}:
\[\lim_{x \to x_0}f(x) = A \Leftrightarrow \lim_{x \to x_0^+}f(x) = A \text{且} \lim_{x \to x_0^-}f(x) = A\]
\[\lim_{x \to x_0}f(x) = A \Leftrightarrow f(x) = A + \alpha(x), \lim_{x \to x_0}\alpha(x) = 0\]


\subsection{局部有界性}
若\(\lim_{x \to x_0}f(x) = A\),则存在正常数M,\(\delta\)使得当\(0 < |x - x_0| < \delta\)时,有\(|f(x)| <= M\)

若f(x)在[a, b]上连续函数,则f(X)在[a, b]上有界;

若f(x)在(a, b)上连续函数,且\(\displaystyle \lim_{x \to a^+}f(x)\lim_{x \to b^-}f(x)\)存在 ,则f(X)在(a, b)上有界;

极限存在是函数局部有界的\textbf{充分条件};

有界函数与有界函数的和、差、积仍为\textbf{有界函数}


\subsection{局部保号性}
若\(f(x) \to A(x \to x_0)\)且A > 0(或A < 0),那么存在常数\(\delta > 0\)使得当\(0 < |x - x_0| < \delta\)时,有\(f(x) > 0\)(或f(x) < 0);若在\(x_0\)某去心邻域内f(x) >= 0(或f(x) <= 0),且\(f(x) \to A(x \to x_0)\),则A >= 0(或A <= 0)


\section{无穷小}
f(x)为当\(x \to x_0\)时的无穷小,\(\displaystyle \lim_{x \to x_0}f(x) = 0\)
\[\lim_{x \to x_0}f(x) = A \Leftrightarrow f(x) = A + \alpha,\lim_{x \to x_0}\alpha = 0,\]

\subsection{性质}
有限个无穷小的和是无穷小;

有界函数与无穷小乘积是无穷小;

有限个无穷小乘积是无穷小;


\subsection{比阶}
自变量同一变化过程中,\(lim\alpha(x) = 0, \lim\beta(x) = 0\),且\(\beta(x) \neq 0\),则

若\(\lim\dfrac{\alpha(x)}{\beta(x)} = 0\),则\(\alpha(x)\)是\(\beta(x)\)\textbf{高阶无穷小},记作\(\alpha(x) = o(\beta(x))\);

若\(\lim\dfrac{\alpha(x)}{\beta(x)} = \infty\),则\(\alpha(x)\)是\(\beta(x)\)\textbf{低阶无穷小};

若\(\lim\dfrac{\alpha(x)}{\beta(x)} = c \neq 0\),则\(\alpha(x)\)是\(\beta(x)\)\textbf{同阶无穷小};

若\(\lim\dfrac{\alpha(x)}{\beta(x)} = 1\),则\(\alpha(x)\)是\(\beta(x)\)\textbf{等价无穷小},记作\(\alpha(x) \sim \beta(x)\);

若\(\lim\dfrac{\alpha(x)}{[\beta(x)]^k} = c \neq 0\),则\(\alpha(x)\)是\(\beta(x)\)\textbf{k阶无穷小};

\subsection{等价无穷小}

\subsubsection{x -> 0}
\[x \sim ln(1 + x) \sim ln(x + \sqrt{x^2 + 1}) \sim sin\ x \sim tan\ x \sim \arcsin\ x \sim arctan\ x \sim e^x - 1\]
\[ax \sim e^{ax} - 1\]
\[x\ln\alpha \sim \alpha^x - 1\]
\[1 - (cos\ x)^a \sim \frac{1}{2}ax^2\]
\[1 - \cos ax \sim \dfrac{1}{2}(ax)^2\]
\[x - ln(1 + x) \sim \frac{1}{2}x^2\]
\[a^x - 1 \sim xln\ a\]
\[(1 + x)^a - 1 \sim ax\]
\[\sqrt[a]{1 + x} - 1 \sim \frac{x}{a}\]
\[(1 + ax)^{\frac{1}{bx + d}} \sim e^{ab}\]
\[\sqrt{1 + \tan x} - 1 \sim \dfrac{1}{2}\tan x\]
\[\sqrt{1 - \sin x} - 1 \sim -\dfrac{1}{2}\sin x\]
\[x - sin\ x \sim arcsin\ x - x \sim \dfrac{1}{6}x^3\]
\[ax - \sin ax \sim \dfrac{1}{6}(ax)^3\]
\[tan\ x - x \sim x - arctan\ x \sim \dfrac{1}{3}x^3\]
\[\tan x - \sin x \sim \dfrac{1}{2}x^3\]
\[\tan(\tan x) - \sin(\sin x) \sim x^3\]
\[(x \to 0^+)(1 + x)^{\frac{1}{x}} - e \sim -\frac{e}{2}x\]

\subsubsection{x -> 1}
\begin{displaymath}
lnx = ln(x + 1 - 1) \sim x - 1 (x->1)
\end{displaymath}

\section{无穷大}
f(x)为当\(x \to x_0\)时的无穷大,\(\displaystyle \lim_{x \to x_0}f(x) = \infty\)


\paragraph{等价无穷大}




\section{计算}

\subsection{四则运算}
若\(\lim f(x) = A \neq 0, \lim g(x) = B\),则\(\lim f(x)g(x) = A\lim g(x)\)

若\(\lim f(x)\)存在,\(\lim g(x)\)不存在,则\(\lim[f(x) \pm g(x)]\)必不存在;

若\(\lim f(x)\)不存在,\(\lim g(x)\)不存在,则\(\lim[f(x) \pm g(x)]\)不一定不存在;


\subsection{定积分定义}
\[\int_a^bf(x)dx = \lim_{x \to \infty}\sum_{i = 1}^nf(a + \dfrac{b - a}{n}i)\dfrac{b - a}{n}\]
\[\int_0^1f(x)dx = \lim_{x \to \infty}\sum_{i = 1}^nf(\dfrac{i}{n})\dfrac{1}{n}\]


\subsection{洛必达}
\begin{itemize}
    \item 当\(x \to a || x \to \infty\)时,\begin{enumerate}
        \item f(x)和F(x)趋于0;
        \item \(f'(x),F'(x)\)在点a的某去心邻域内存在且\(F'(x) \neq 0\);
        \item \(\displaystyle \lim_{x \to a}\dfrac{f'(x)}{F'(x)}\)存在或为无穷大;
    \end{enumerate}
    则\[\displaystyle \lim_{x \to a}\dfrac{f(x)}{F(x)} = \lim_{x \to a}\dfrac{f'(x)}{F'(x)}\]
    \item 当\(x \to a || x \to \infty\)时,\begin{enumerate}
        \item f(x)和F(x)趋于无穷大;
        \item \(f'(x),F'(x)\)在点a的某去心邻域内存在且\(F'(x) \neq 0\);
        \item \(\displaystyle \lim_{x \to a}\dfrac{f'(x)}{F'(x)}\)存在或为无穷大;
    \end{enumerate}
    则\[\displaystyle \lim_{x \to a}\dfrac{f(x)}{F(x)} = \lim_{x \to a}\dfrac{f'(x)}{F'(x)}\]
\end{itemize}




对于\(\displaystyle \lim_{x \to a}\dfrac{f(x)}{F(x)} = \lim_{x \to a}\dfrac{f'(x)}{F'(x)}\),右存在则左存在,左存在而右不一定存在。


\subsection{泰勒公式}
设f(x)在点x = 0处n阶可导,则存在x = 0的一个邻域,对该邻域内任何一点x有\[f(x) = f(0) + f'(0)x + \dfrac{f''(0)}{2!}x^2 + ... + \dfrac{f^{(n)}(0)}{n!}x^n + 0(x^n)\]

带皮亚诺(Peano)余项形式的泰勒公式\(\displaystyle f(x) = \sum_{k = 0}^n\dfrac{f^{(k)}(x_0)}{k!}(x - x_0)^k + o((x - x_0)^n)\)给出函数在\(x_0\)点的局部表达式;
\begin{flalign}
e^x & = 1 + x + \dfrac{x^2}{2!} + ... + \dfrac{x^n}{n!} + o(x^n) \nonumber \\ 
sin\ x & = x - \dfrac{x^3}{3!} + \dfrac{x^5}{5!} - ... + (-1)^{n - 1}\dfrac{x^{2n - 1}}{(2n - 1)!} + o(x^{2n + 1}) \nonumber \\ 
cos\ x & = 1 - \dfrac{x^2}{2!} + \dfrac{x^4}{4!} - ... + (-1)^{n}\dfrac{x^{2n}}{(2n)!} + o(x^{2n + 1}) \nonumber \\ 
tan\ x & =x + \dfrac{x^3}{3} + o(x^3) \nonumber \\ 
arcsin\ x & = x + \dfrac{x^3}{3!} + o(x^3) \nonumber \\ 
arctan\ x & = x - \dfrac{x^3}{3} + o(x^3) \nonumber \\ 
ln(1 + x) & = x - \dfrac{x^2}{2} + \dfrac{x^3}{3} - ... + (-1)^{n - 1}\dfrac{x^n}{n} + o(x^n) \nonumber \\ 
(1 + x)^\alpha & = 1 + \alpha x + \dfrac{\alpha(\alpha - 1)}{2!}x^2 + ... + \dfrac{\alpha(\alpha - 1)...(\alpha - n + 1)}{n!}x^n + o(x^n) \nonumber
\end{flalign}


\subsection{泰勒公式应用原则}
\begin{itemize}
    \item \(\dfrac{A}{B}\)型,应用上下同阶原则,若分子(母)是x的k次幂,则将分母(子)展开到x的k次幂;
    \item \(A - B\)型,应用幂次最低原则,将A,B分别展开到系数不相等的x的最低次幂;
\end{itemize}


\subsection{无穷小运算}
设m,n为正整数,则
\[o(x^m) \pm o(x^n) = o(x^l), l = min\{m, n\}\]
\[o(x^m) * o(x^n) = o(x^{m + n}), x^m * o(x^n) = o(x^{m + n})\]
\[o(x^m) = o(kx^m) = k*o(x^m)\]


\subsection{夹逼准则}
\paragraph{1}
\[\lim_{n \to \infty}\sqrt[n]{a^n + b^n + c^n} = \max\{a, b, c\},\ (a, b, c > 0)\]

\paragraph{2}
\[\lim_{n \to \infty}\dfrac{(2n - 1)!!}{(2n)!!} = 0\]
\subparagraph{证明}
\(4n^2 > 4n^2 - 1 = (2n + 1)(2n - 1) \Rightarrow 2n > \sqrt{(2n + 1)(2n - 1)}\),故
\[\dfrac{(2n - 1)!!}{(2n)!!} < \dfrac{(2n - 1)!!}{(1 * \sqrt{3})(\sqrt{3} * \sqrt{5})...(\sqrt{(2n - 1)(2n + 1)}} = \dfrac{1}{\sqrt{2n + 1}}\]

\paragraph{3}
\[\lim_{n \to \infty}\sqrt{n}\dfrac{(2n - 1)!!}{(2n)!!} = \dfrac{1}{\sqrt{\pi}}\]


\subsection{七种极限}
\(\dfrac{0}{0}, \dfrac{\infty}{\infty}, 0 * \infty, \infty - \infty, \infty^0, 0^{\infty}, 1^{\infty}\)

\begin{itemize}
    \item \(\dfrac{0}{0}, \dfrac{\infty}{\infty}, 0 * \infty\)\begin{enumerate}
        \item 化简:\begin{enumerate}
            \item 提出极限不为0的因式;
            \item 等价无穷小代换;
            \item 恒等变形(提公因式、拆项、合并、分子分母同除变量的最高次幂、变量代换(换元法)等);
        \end{enumerate}
        \item 判断类型;
        \item 选择方法(洛必达、泰勒、夹逼);
    \end{enumerate}
    \item \(\infty - \infty\)有分母则通分,无分母则提取公因式或作倒代换或分子有理化后通分;
    \item \(\infty^0, 0^{\infty}\)恒等变形\(\lim\ u^v = e^{\lim\ v\ \ln\ u} \text{记作}exp\{\lim\ v\ \ln\ u\}\)
    \item \(1^{\infty}\)恒等变形\(\lim\ u^v = e^{\lim(u - 1)v}\)
    \[\lim\ u^v = \lim\{[1 + (u - 1)]^{\frac{1}{u - 1}}\}^{(u - 1)v} = e^{\lim(u - 1)v}\]
\end{itemize}


\subsection{重要极限}
\[\lim_{x \to \infty}(1 + \dfrac{a}{x})^{bx + d} = e^{ab}\]
\[(f(x) \to 0\ \text{or}\ g(x) \to \infty)\ (1 + \dfrac{f(x)}{g(x)})^{\frac{g(x)}{f(x)}h(x)} = e^{h(x)}\]
\begin{displaymath}
(x \to 0)
e^{tan\ x} - e^{sin\ x} = 
e^{sin\ x}(e^{tan\ x - sin\ x} - 1) \sim
tan\ x - sin\ x = tan\ x(1 - cos\ x) \sim \dfrac{1}{2}x^3
\end{displaymath}
\begin{displaymath}
当a > 0时,
\lim_{x \to 0^{+}} x^{a}lnx = 
\lim_{x \to 0^{+}} \frac{lnx}{x^{-a}} =
\lim_{x \to 0^{+}} \frac{x^{-1}}{-ax^{-a-1}} = 
-\frac{1}{a} \lim_{x \to 0^{+}} x^{a} = 0
\end{displaymath}
\[a > 0, b\text{任意实数}, \lim_{x \to +\infty}\dfrac{\ln^bx}{x^a} = 0\]

当\(x \to +\infty\)时,有\(ln^ax << x^b << c^x\),其中a, b > 0, c > 1,符号<<代表远小于;

当\(x \to \infty\)时,有\(ln^ax << x^b << c^x << x! << x^x\),其中a, b > 0, c > 1;

\[\lim_{x \to +\infty}x(a^{\frac{1}{x}} - b^{\frac{1}{x}}) = \lim_{t \to 0^+}\dfrac{a^t - b^t}{t} = \lim_{t \to 0^+}(a^tlna - b^tlnb) = ln\frac{a}{b}\]
\begin{flalign}
    \lim_{x \to \infty}\dfrac{x^{99}}{x^k - (x - 1)^k} & = \lim_{x \to \infty} \dfrac{x^{99}}{x^k[1 - (1 - \dfrac{1}{x})^k]} \nonumber \\ 
    & = -\lim_{x \to \infty}\dfrac{x^{99 - k}}{(1 - \dfrac{1}{x})^k - 1} = -\lim_{x \to \infty}\dfrac{x^{99 - k}}{k(-\dfrac{1}{x})} \nonumber \\ 
    & = \dfrac{1}{k}\lim_{x \to \infty}x^{99 - k + 1} \nonumber
\end{flalign}
\[(x \to 0^+)x^x = e^{x\ln x} = e^0 = 1\]
\[\text{Wallis乘积}\lim_{n \to \infty}\dfrac{(2n)!!^2}{(2n - 1)!!(2n + 1)!!} = \dfrac{\pi}{2}\]

\subsubsection{三角函数}
\[\lim_{x \to 0}\sin\dfrac{1}{x}\text{有界振荡}(\in[-1, 1])\text{极限不存在}\]
\[\lim_{x \to 0}\cos\dfrac{1}{x}\text{有界振荡}(\in[-1, 1])\text{极限不存在}\]
\[\lim_{x \to 0}\tan\dfrac{1}{x}\text{无界振荡,极限不存在}\]
\[\begin{cases}\displaystyle
    \lim_{x \to 0^+}\arctan\dfrac{1}{x} = \dfrac{\pi}{2} \\ 
    \displaystyle\lim_{x \to 0^-}\arctan\dfrac{1}{x} = -\dfrac{\pi}{2}
\end{cases}\]





