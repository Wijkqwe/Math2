
\chapter{中值定理}

\section{涉及函数}
设\(f(x)\)在\([a, b]\)上连续,则

\subsection{有界与最值定理}
\(m <= f(x) <= M\),其中\(m, M\)为\(f(x)\)在\([a, b]\)上最小值与最大值;

\subsection{介值定理}
当\(m <= \mu <= M\),存在\(\xi \in [a, b]\),使得\(f(\xi) = \mu\);

\subsection{平均值定理}
当\(a < x_1 < ... < x_n < b\)时,在\([x_1, x_n]\)内至少存在一点\(\xi\)使得\(f(\xi) = \dfrac{f(x_1) + f(x_2) + ... + f(x_n)}{n}\);

\subsection{零点定理}
当\(f(a) * f(b) < 0\)时,存在\(\xi \in (a, b)\),使得\(f(\xi) = 0\);

\paragraph{判断f(x)零点个数}
\begin{enumerate}
    \item 求f(x)单调区间
    \item 考察极值符号
    \item 求边界极限值
\end{enumerate}


\section{涉及导数(微分)}

\subsection{费马定理}
设\(f(x)\)在点\(x_0\)处满足\(\begin{cases}
\text{可导} \\ 
\text{取极值}
\end{cases}\),则\(f'(x) = 0\);


\subsection{导数零点定理}
设\(f(x)\)在\([a, b]\)上可导,证明当\(f'_+(a) * f'_-(b) < 0\)时,存在\(\xi \in (a, b)\),使得\(f'(\xi) = 0\)


\subsection{罗尔定理}
设\(f(x)\),满足\(\begin{cases}
\text{在}[a, b]\text{上连续} \\ 
\text{在}(a, b)\text{内可导} \\ 
f(a) = f(b)
\end{cases}\),则存在\(\xi \in (a, b)\)使得\(f'(\xi) = 0\);

设\(f(x)\)在\((a, b)\)内可导,\(\displaystyle\lim_{x \to a^+}f(x) = \lim_{x \to b^-}f(x) = A\),则在\((a, b)\)内至少存在一点\(\xi\),使得\(f'(\xi) = 0\);A可是有限数也可是无穷,区间\((a, b)\)可有限区间也可无穷区间;\textbf{}

\subsubsection{构造辅助函数}

\subparagraph{乘积求导公式的逆用}
\[[f(x)f(x)]' = [f^2(x)]' = 2f(x) * f'(x)\]
对于\(f(x)f'(x)\),令\(F(x) = f^2(x)\);
\[[f(x) * f'(x)]' = [f'(x)]^2 + f(x)f''(x)\]
对于\([f'(x)]^2 + f(x)f''(x)\),令\(F(x) = f(x) * f'(x)\);
\[[f(x)e^{\varphi(x)}]' = f'(x)e^{\varphi(x)} + f(x)e^{\varphi(x)} * \varphi'(x) = [f'(x) + f(x)\varphi'(x)]e^{\varphi(x)}\]
对于\(f'(x) + f(x)\varphi'(x)\),令\(F(x) = f(x)e^{\varphi(x)}\),其中\begin{itemize}
    \item \(\varphi(x) = x \Rightarrow f'(x) + f(x), F(x) = f(x)e^x\)
    \item \(\varphi(x) = -x \Rightarrow f'(x) - f(x), F(x) = f(x)e^{-x}\)
    \item \(\varphi(x) = kx \Rightarrow f'(x) + kf(x), F(x) = f(x)e^{kx}\)
\end{itemize}


\subparagraph{商的求导公式\((\dfrac{u}{v})' = \dfrac{u'v - uv'}{v^2}\)的逆用}

\[[\dfrac{f(x)}{x}]' = \dfrac{xf'(x) - f(x)}{x^2}\]
对于\(xf'(x) - f(x), x \neq 0\),令\(F(x) = \dfrac{f(x)}{x}\);
\[[\dfrac{f'(x)}{f(x)}]' = \dfrac{f''(x)f(x) - [f'(x)]^2}{f^2(x)}\]
对于\(f''(x)f(x) - [f'(x)]^2, f(x) \neq 0\),令\(F(x) = \dfrac{f'(x)}{f(x)}\);
\[[\ln f(x)]' = \dfrac{f'(x)}{f(x)}, [\ln f(x)]'' = \dfrac{f''(x)f(x) - [f'(x)]^2}{f^2(x)}\]
对于\(f''(x)f(x) - [f'(x)]^2, f(x) > 0\),令\(F(x) = \ln f(x)\);


\subsection{拉格朗日中值定理}
设\(f(x)\)满足\(\begin{cases}
\text{在}[a, b]\text{上连续} \\ 
\text{在}(a, b)\text{内可导}
\end{cases}\),则存在\(\xi \in (a, b)\),使得\(f(b) - f(a) = f'(\xi)(b - a)\),\(f'(\xi) = \dfrac{f(b) - f(a)}{b - a}\)


\subsection{柯西中值定理}
设\(f(x), g(x)\)满足\(\begin{cases}
[a, b]\text{上连续} \\ 
(a, b)\text{内可导} \\ 
g'(x) \neq 0
\end{cases}\),则存在\(\xi \in (a, b)\),使得\(\dfrac{f(b) - f(a)}{g(b) - g(a)} = \dfrac{f'(\xi)}{g'(\xi)}\)


\subsection{泰勒公式}

\subsubsection{带拉格朗日余项的n阶泰勒公式}
设\(f(x)\)在点\(x_0\)的某个邻域内n + 1阶导数存在,则对该邻域内有任意点x,有\[f(x) = f(x_0) + f'(x_0)(x - x_0) + ... + \dfrac{1}{n!}f^{(n)}(x_0)(x - x_0)^n + \dfrac{f^{(n + 1)}(\xi)}{(n + 1)!}(x - x_0)^{n + 1}\]

\subsubsection{带佩亚诺余项的n阶泰勒公式}
设\(f(x)\)在点\(x_0\)处\(n\)阶可导,则存在\(x_0\)的一个邻域,对于该邻域内的任意点x,有\[f(x) = f(x_0) + f'(x)(x - x_0) + ... + \dfrac{1}{n!}f^{(n)}(x_0)(x - x_0)^n + 0((x - x_0)^n)\]

\subsubsection{麦克劳林公式}
当\(x_0 = 0\)时的泰勒公式为麦克劳林公式;


\section{微分等式}

\subsection{零点定理(证明根的存在性)}

\subsection{单调性(证明根的唯一性)}

\subsection{罗尔定理及推论}

\subsection{实系数奇次方程至少一个实根}


\section{微分不等式}

\subsection{用函数性态证明不等式}
包括单调性,凹凸性,最值等\begin{itemize}
    \item 若有\(f'(x) >= 0, a < x < b\),则有\(f(a) <= f(x) <= f(b)\)
    \item 若有\(f''(x) >= 0, a < x < b\),则有\(f'(a) <= f'(x) <= f'(b)\)\begin{itemize}
        \item 当\(f'(a) > 0\)时,\(f'(x) > 0 \Rightarrow f(x)\)单调增加
        \item 当\(f'(b) < 0\)时,\(f'(x) < 0 \Rightarrow f(x)\)单调减少
    \end{itemize}
    \item 设\(f(x)\)在\(I\)内连续,且有唯一极值点\(x_0\),则\[\begin{cases}
        x_0\text{为极大值时,即为I内的最大值点,有}f(x_0) >= f(x) \\ 
        x_0\text{为极小值时,即为I内的最小值点,有}f(x_0) <= f(x)
    \end{cases}\]
    \item 若有\(f''(x) > 0, a < x < b, f(a) = f(b) = 0\),则有\(f(x) < 0\)
\end{itemize}


\subsection{用常数变量化证明不等式}
若不等式中都是常数,则可将其中一个或几个常数变量化;


\subsection{用中值定理证明不等式}





