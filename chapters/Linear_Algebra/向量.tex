
\chapter{向量}


\section{概念}

\section{运算}

\paragraph{相等}
设\(\alpha,\beta\)为n维向量,当且仅当\(a_i = b_i(i = 1,2,...,n)\)时,\(\alpha = \beta\)


\paragraph{加}
\(\alpha + \beta = [a_1 + b_1, ..., a_n + b_n]\)


\paragraph{乘}
\(k\alpha = [ka_1, ..., ka_n]\)


\subsection{内积}
设\(\alpha = [a_1, ..., a_n]^T, \beta = [b_1, ..., b_n]^T\),则称\[\alpha^T\beta = \sum_{i = 1}^{n} a_ib_i = a_1b_1 + ... + a_nb_n\]为\(\alpha, \beta\)的内积,记作\((\alpha, \beta) = \alpha^T\beta\)。


\subsection{正交}
当\(\alpha^T\beta = 0\)时,\(\alpha, \beta\)是正交向量。

\paragraph{施密特正交化公式}
设向量\(\alpha_1, \alpha_2, \alpha_3\),令
\[\beta_1 = \alpha_1,\ \beta_2 = \alpha_2 - \dfrac{(\beta_1, \alpha_2)}{(\beta_1, \beta_1)}\beta_1,\ \beta_3 = \alpha_3 - \dfrac{(\beta_1, \alpha_3)}{(\beta_1, \beta_1)}\beta_1 - \dfrac{(\beta_2, \alpha_3)}{(\beta_2, \beta_2)}\beta_2\]
则\(\beta_1, \beta_2, \beta_3\)相互正交。


\subsection{模}
\(||\alpha|| = \displaystyle \sqrt{\sum_{i = 1}^{n}a_i^2}\)称为\(\alpha\)的模(长度)。

当\(||\alpha|| = 1\)时,\(\alpha\)为单位向量。


\subsection{标准正交向量组}
若列向量组\(\alpha_1, ..., \alpha_n\)满足\[\alpha_i^T\alpha_j = 
\begin{cases}
0, i \neq j \\ 
1, i = j
\end{cases}
\]则称为标准或单位正交向量组,也叫\textbf{规范正交基}。


\section{线性相关}

\subsection{线性组合}
设n维向量组\(\alpha_1, ..., \alpha_m\)及m个数\(k_1, ..., k_m\),则向量\[k_1\alpha_1 + ... + k_m\alpha_m\]
称为向量组的线性组合。


\subsection{线性表示}
若\(\beta\)能表示成向量组\(\alpha_1, ..., \alpha_m\)的线性组合,即存在m个数\(k_1, ..., k_m\)使得\[\beta = k_1\alpha_1 + k_2\alpha_2 + ... + k_m\alpha_m\]则称向量\(\beta\)能被向量组线性表示。


\subsection{线性相关}
对于m个n维向量\(\alpha_1, ..., \alpha_m\),若存在一组不全为0的数\(k_1, ..., k_m\)使得\[k_1\alpha_1 + ... + k_m\alpha_m = 0\]则称向量组线性相关;

含零向量或存在成比例的向量的向量组必线性相关。


\subsection{线性无关}
若不存在不全为0的数\(k_1, ..., k_m\)使得\(k_1\alpha_1 + ... + k_m\alpha_m = 0\)成立;只有当\(k_1 = ... =k_m = 0\)时,才有\(k_1\alpha_1 + ... + k_m\alpha_m = 0\)成立,则称向量组\(\alpha_1, ..., \alpha_m\)线性无关。

单个非零向量或两个不成比例的向量均线性无关。


\subsection{线性相关七大定理}
\begin{enumerate}
    \item 向量组线性相关\textbf{充要条件}是向量组中至少一个向量可由其余n - 1个向量线性表示;

    向量组线性无关\textbf{充要条件}是向量组中任一个向量都不可由其余n - 1个向量线性表示;
    
    \item 若向量组\(a_1, ..., a_n\)线性无关,而\(\beta, a_1, ..., a_n\)线性相关,则\(\beta\)可由\(a_1, ..., a_n\)线性表示且表示法唯一;
    
    \item 若向量组\(b_1, ..., b_m\)可由向量组\(a_1, ..., a_n\)线性表示,且\(m > n\),则\(b_1, ..., b_m\)线性相关;(以少表多,多的相关)

    若向量组\(b_1, ..., b_m\)可由向量组\(a_1, ..., a_n\)线性表示,且\(b_1, ..., b_m\)线性无关,则\(m <= n\);
    
    \item 设m个n维向量\(a_1, ..., a_n\),其中\(a_1 = [a_{11}, ..., a_{n1}]^T, ..., a_m = [a_{1m}, ..., a_{nm}]\),那么\begin{flalign}
        & \text{向量组}a_1, ..., a_m\text{线性相关} \nonumber \\ 
        \Leftrightarrow & r(a_1, ..., a_m) < m \nonumber \\ 
        \Leftrightarrow & \text{齐次线性方程组}[a_1, ..., a_m]\begin{bmatrix}
            x_1 \\ 
            ... \\
            x_m
        \end{bmatrix} = x_1a_1 + x_2a_2 + ... + x_ma_m = 0 \text{有非零解} \nonumber
    \end{flalign}
    \mymatrix
    向量组线性无关的\textbf{充要条件}是齐次线性方程组只有零解;

    若n < m,即方程个数小于未知数个数,则齐次线性方程组求解时必有自由未知量,即必有非零解,因此任意n + 1个n维向量都是线性相关的,即向量个数超过向量维数时,向量组必线性相关;
    \begin{flalign}
        & \text{n个n维向量线性相关} \nonumber \\ 
        \Leftrightarrow & |a_1, ... a_n| = 0 \nonumber \\ 
        \Leftrightarrow & [a_1, ... a_n]\text{不可逆} \nonumber \\ 
        \Leftrightarrow & r([a_1, ... a_n]) < n \nonumber \\ 
        \Leftrightarrow & [a_1, ... a_n]x = 0\text{有非零解} \nonumber
    \end{flalign}
    
    \item 向量\(\beta\)可由向量组\(a_1, ..., a_n\)线性表示
    \begin{flalign}
        & \Leftrightarrow \text{非齐次线性方程组}
        \begin{bmatrix}
            a_1, ..., a_n
        \end{bmatrix}
        \begin{bmatrix}
            x_1 \\ 
            ... \\ 
            x_n
        \end{bmatrix} = x_1a_1 + ... + x_na_n = \beta\text{有解} \nonumber \\ 
        & \Leftrightarrow r([a_1, ..., a_n]) = r([a_1, ..., a_n, \beta]) \nonumber
    \end{flalign}
    
    \item 若向量组中一部分线性相关,则向量组线性相关;

    若向量组线性无关,则其任一部分向量组线性无关;
    
    \item 若一组n维向量线性无关,则这些向量各任意添加m个分量得到的n + m维向量组线性无关;

    \item 设n个向量\(a_1, ..., a_n\)线性无关,若n阶方阵B可逆
    \footnote[1]{\(A \text{可逆} \Leftrightarrow |A| \neq 0 \Leftrightarrow r(A_{n * n}) = n \Leftrightarrow (A)x = 0\text{无非零解} \Leftrightarrow A\text{列向量线性无关}\)},则\([a_1, ..., a_n]B\)线性无关;

    \item n个n维向量线性无关\begin{flalign}
        & \text{n个n维向量线性无关} \nonumber \\ 
        \Leftrightarrow & |a_1, ... a_n| \neq 0 \nonumber \\ 
        \Leftrightarrow & [a_1, ... a_n]\text{可逆} \nonumber \\ 
        \Leftrightarrow & r([a_1, ... a_n]) = n \nonumber \\ 
        \Leftrightarrow & [a_1, ... a_n]x = 0\text{无非零解} \nonumber \\ 
        \Leftrightarrow & \text{可以表示任一n维向量} \nonumber \\ 
        \Leftrightarrow & \text{与单位向量}\varepsilon_1 = (1, 0, ...)^T, ..., \varepsilon_n = (0, ..., 1)^T\text{等价} \nonumber
    \end{flalign}
\end{enumerate}


\subsection{极大线性无关组}

在向量组中,若存在部分组\(a_{i_1},..., a_{i_r}\)满足:\begin{enumerate}
    \item \(a_{i_1},..., a_{i_r}\)线性无关;
    \item 向量组中任一向量均可由部分组\(a_{i_1},..., a_{i_r}\)线性表示;
\end{enumerate}
则称部分组\(a_{i_1},..., a_{i_r}\)是原向量组的\textbf{极大线性无关组}。
\begin{itemize}
    \item 在一个向量组中,能代表该组所有成员的一组向量称为原向量组的极大线性无关组;
    \item 一个线性无关向量组的极大线性无关组是该向量组本身;
    \item 向量组的极大线性无关组一般不唯一;
    \item 只由一个零向量构成的向量组不存在极大线性无关组;
\end{itemize}


\section{等价向量组}

设向量组\((1)a_1, ..., a_s, (2)b_1, ..., b_t\)若(1)中每个向量均可由(2)中向量线性表示,则称(1)可由(2)线性表示;若(1)(2)可相互线性表示,则(1)(2)\textbf{等价向量组},记作\((1) \cong (2)\)。

\paragraph{反身性}
\((1) \cong (1)\)

\paragraph{对称性}
若\((1) \cong (2)\),则\((2) \cong (1)\)

\paragraph{传递性}
若\((1) \cong (2), (2) \cong (3)\),则\((1) \cong (3)\)

\begin{itemize}
    \item 向量组与其极大线性无关组是等价向量组;
\end{itemize}


\section{向量组的秩}
向量组的极大线性无关组所含向量的个数r为向量组的秩;

等价向量组的秩相等;

\paragraph{(1)三秩相等}
\(r(A)\)(矩阵的秩) = A的行秩(行向量的秩) = A的列秩(列向量的秩)

\paragraph{(2)}
若\(A \xrightarrow{\text{初等行变换}} B\)

则A的行向量与B的列向量是等价向量组;

A与B的任何相应的部分列向量具有相同的线性相关性;

\paragraph{(3)}
设向量组(I),(II),若任一(II)中向量可由(I)线性表示,则
\[r(II) <= r(I)\]


\section{等价矩阵等价向量组}

\subsection{定义}
矩阵等价要同型,行数列数相等;向量等价要同维,但向量个数可以不等。

\subsection{矩阵}
A,B同型时:\(A \cong B \Leftrightarrow r(A) = r(B) \Leftrightarrow PAQ = B\)

\subsection{向量}
\(a_i, b_i\)同维,则:
\begin{flalign}
& \{a_i\} \cong \{b_i\} \nonumber \\ 
\Leftrightarrow & \{a_i\}, \{b_i\}\text{可以相互表示} \nonumber \\ 
\Leftrightarrow & r(\{a_i\}) = r(\{b_i\}) = r(\{a_i, b_i\}) \nonumber \\ 
\Leftrightarrow & r(\{a_i\}) = r(\{b_i\}), \text{且可单方向线性表示} \nonumber
\end{flalign}


\section{向量空间(数一)}


