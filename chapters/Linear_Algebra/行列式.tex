
\chapter{行列式}

\subsection{性质}
\begin{itemize}
    \item 若某行(列)元素全零,则行列式为零;
    \item 两行(列)互换,行列式变号;
    \item 两行(列)元素相等或成比例,则行列式为零;
    \item 某行(列)的k倍加到另一行(列),行列式不变;
    \item 行列互换值不变\(|A| = |A^T|\)
    \item 若|A|!=0,则矩阵A可逆;
\end{itemize}
\[|AB| = |A||B|\]
\[|kA| = k^n|A|\]
\[|A^{-1}| = \dfrac{1}{|A|}\]
\[\begin{vmatrix}
a_{11} & ... & a_{1n} \\
... & ... & ... \\
ka_{i1} & ... & ka_{in} \\
... & ... & ... \\
a_{n1} & ... & a_{nn}
\end{vmatrix} = k
\begin{vmatrix}
a_{11} & ... & a_{1n} \\
... & ... & ... \\
a_{i1} & ... & a_{in} \\
... & ... & ... \\
a_{n1} & ... & a_{nn}
\end{vmatrix}\]
\[\begin{vmatrix}
a_{11} & ... & a_{1n} \\
... & ... & ... \\
b_{i1} + a_{i1} & ... & b_{in} + a_{in} \\
... & ... & ... \\
a_{n1} & ... & a_{nn}
\end{vmatrix} = 
\begin{vmatrix}
a_{11} & ... & a_{1n} \\
... & ... & ... \\
a_{i1} & ... & a_{in} \\
... & ... & ... \\
a_{n1} & ... & a_{nn}
\end{vmatrix} + 
\begin{vmatrix}
a_{11} & ... & a_{1n} \\
... & ... & ... \\
b_{i1} & ... & b_{in} \\
... & ... & ... \\
a_{n1} & ... & a_{nn}
\end{vmatrix}\]


\section{定义}

\subsection{逆序}

\paragraph{排列}
1,2,...,n组成的有序数组成为n级排列;

\paragraph{逆序}
n级排列中,\(i_a > i_b\)且\(a < b\),则这两数构成一个逆序;

\paragraph{逆序数}
一个排列中,逆序的总数,记作\(r(i_1...i_n)\);

\paragraph{奇偶排列}
逆序数为奇则奇排列,逆序数为偶则偶排列;


\subsection{n阶定义}
n >= 2时
\[\begin{vmatrix}
a_{11} & ... & a_{1n} \\
... & ... & ... \\
a_{n1} & ... & a_{nn}
\end{vmatrix} = 
\sum_{j_1...j_n} (-1)^{r(j1...jn)}a_{1j_1}...a_{nj_n}\]
\(\displaystyle \sum_{j_1...j_n}^{}\)表示对所有n个列下标排列求和,即n!项之和。


\section{展开定理}

\subsection{余子式}

n阶行列式中,去除元素\(a_{ij}\)所在第i行第j列元素,剩下的n-1阶行列式称为元素\(a_{ij}\)的\textbf{余子式},记作\(M_{ij}\),即
\[M_{ij} = 
\begin{vmatrix}
a_{11} & ... & a_{1,j-1} & a_{1, j+1} & ... & a_{1n} \\
... & ... & ... & ... & ... & ... \\
a_{i-1,1} & ... & a_{i-1,j-1} & a_{i-1, j+1} & ... & a_{i-1,n} \\
a_{i+1,1} & ... & a_{i+1,j-1} & a_{i+1, j+1} & ... & a_{i+1,n} \\
... & ... & ... & ... & ... & ... \\
a_{n1} & ... & a_{n,j-1} & a_{n, j+1} & ... & a_{nn}
\end{vmatrix}\]


\subsection{代数余子式}

余子式\(M_{ij}\)乘\((-1)^{i + j}\)后称为\(a_{ij}\)的\textbf{代数余子式},记作\(A_{ij}\),即
\[A_{ij} = (-1)^{i+j}M_{ij}\]


\subsection{展开公式}

行列式等于其某行(列)元素分别乘其对应代数余子式后的和,即
\[|A| = 
\begin{cases}
a_{i1}A_{i1} +...+ a_{in}A_{in} = \displaystyle \sum_{j=1}^{n} a_{ij}A_{ij} \\ 
a_{1j}A_{1j} +...+ a_{nj}A_{nj} = \displaystyle \sum_{i=1}^{n} a_{ij}A_{ij}
\end{cases}\]

某行(列)元素分别乘另一行(列)元素的代数余子式后\textbf{求和结果为零}。


\section{重要行列式}

\subsection{主对角线(上下三角)}

\[\begin{vmatrix}
a_{11} & ... & 0 \\
... & ... & ... \\
a_{n1} & ... & a_{nn}
\end{vmatrix} = 
\begin{vmatrix}
a_{11} & ... & a_{1n} \\
... & ... & ... \\
0 & ... & a_{nn}
\end{vmatrix} = \prod_{i = 1}^{n}a_{ii}\]


\subsection{副对角线}

\[\begin{vmatrix}
a_{11} & ... & a_{1n} \\
... & ... & ... \\
a_{n1} & ... & 0
\end{vmatrix} = 
\begin{vmatrix}
0 & ... & a_{1n} \\
... & ... & ... \\
a_{n1} & ... & a_{nn}
\end{vmatrix} = (-1)^{\frac{n(n - 1)}{2}}a_{1n}...a_{n1}\]


\subsection{拉普拉斯展开式}

设A为m阶矩阵,B为n阶矩阵,则
\[\begin{vmatrix}
A & 0 \\
0 & B
\end{vmatrix} = \begin{vmatrix}
A & C \\
0 & B
\end{vmatrix} = \begin{vmatrix}
A & 0 \\
C & B
\end{vmatrix} = |A||B|\]
\[\begin{vmatrix}
0 & A \\
B & 0
\end{vmatrix} = \begin{vmatrix}
C & A \\
B & 0
\end{vmatrix} = \begin{vmatrix}
0 & A \\
B & C
\end{vmatrix} = (-1)^{mn}|A||B|\]


\subsection{范德蒙德行列式}

\[\begin{vmatrix}
1 & ... & 1 \\
x_1 & ... & x_n \\ 
x_1^2 & ... & x_n^2 \\ 
... & ... & ... \\ 
x_1^{n-1} & ... & x_n^{n-1}
\end{vmatrix} = \prod_{1<=i<j<=n}(x_j-x_i)\]


\section{计算}

\subsection{对角a其它b}

\[\begin{vmatrix}
a & b & b & ... & b \\
b & a & b & ... & b \\ 
b & b & a & ... & b \\ 
... & ... & ... & ... & ... \\ 
b & b & b & ... & a
\end{vmatrix} = 
\begin{vmatrix}
a + (n-1)b & b & b & ... & b \\
a + (n-1)b & a & b & ... & b \\ 
a + (n-1)b & b & a & ... & b \\ 
... & ... & ... & ... & ... \\ 
a + (n-1)b & b & b & ... & a
\end{vmatrix} = (a+(n-1)b)
\begin{vmatrix}
1 & b & b & ... & b \\
1 & a & b & ... & b \\ 
1 & b & a & ... & b \\ 
... & ... & ... & ... & ... \\ 
1 & b & b & ... & a
\end{vmatrix}\]
\[= (a+(n-1)b)
\begin{vmatrix}
1 & b & b & ... & b \\
0 & a-b & 0 & ... & 0 \\ 
0 & 0 & a-b & ... & 0 \\ 
... & ... & ... & ... & ... \\ 
0 & 0 & 0 & ... & a-b
\end{vmatrix} = (a+(n-1)b)(a-b)^{n-1}\]

当a = 0,b = 1时
\[\begin{vmatrix}
0 & 1 & 1 & ... & 1 \\
1 & 0 & 1 & ... & 1 \\ 
1 & 1 & 0 & ... & 1 \\ 
... & ... & ... & ... & ... \\ 
1 & 1 & 1 & ... & 0
\end{vmatrix} = (-1)^{n-1}(n - 1)\]

当a = 2,b = 1时
\[\begin{vmatrix}
a & b & b & ... & b \\
b & a & b & ... & b \\ 
b & b & a & ... & b \\ 
... & ... & ... & ... & ... \\ 
b & b & b & ... & a
\end{vmatrix} = n + 1\]

当a副对角线上时
\[G_n = (-1)^{\frac{n(n - 1)}{2}}(a + (n - 1)b)(a - b)^{n - 1}\]


\subsection{向量}

设\(x_n\)三维向量,
\[\begin{vmatrix}
a x_1 + b x_2 + c x_3
\end{vmatrix} = 
\begin{vmatrix}
x_1, x_2, x_3
\end{vmatrix}
\begin{vmatrix}
a \\ 
b \\ 
c
\end{vmatrix}\]
\[\begin{vmatrix}
a_1 x_1 + a_2 x_2 + a_3 x_3, b_1 x_1 + b_2 x_2 + b_3 x_3, c_1 x_1 + c_2 x_2 + c_3 x_3
\end{vmatrix} = 
\begin{vmatrix}
x_1, x_2, x_3
\end{vmatrix}
\begin{vmatrix}
a_1 & b_1 & c_1 \\ 
a_2 & b_2 & c_2 \\ 
a_3 & b_3 & c_3
\end{vmatrix}\]


\subsection{余子式线性组合}

由\(a_{i1}A_{i1} + ... + a_{in}A_{in} = \begin{vmatrix}
     & * & \\ 
    a_{i1} & ... & a_{in} \\ 
     & * & 
\end{vmatrix}\),则\(k_{1}A_{i1} + ... + k_{n}A_{in} = 
\begin{vmatrix}
     & * & \\ 
    k_{1} & ... & k_{n} \\ 
     & * & 
\end{vmatrix}\),其中“*”表示其中元素不变。

即设\(|A| = \begin{vmatrix}
    a_1 & b_1 & c_1 \\ 
    a_2 & b_2 & c_2 \\ 
    a_3 & b_3 & c_3
\end{vmatrix}\),则\[xA_{21} + yA_{22} + zA_{23} = \begin{vmatrix}
    a_1 & b_1 & c_1 \\ 
    x & y & z \\ 
    a_3 & b_3 & c_3
\end{vmatrix}\]


\subsection{克拉默法则}

对n个方程n个未知数的非齐次线性方程组
\[\begin{cases}
a_{11}x_{1} +...+ a_{1n}x_{n} = b_1 \\ 
... \\ 
a_{n1}x_{1} +...+ a_{nn}x_{n} = b_n
\end{cases}\]
若系数行列式\(D = 
\begin{vmatrix}
a_{11} & ... & a_{1n} \\ 
... & ... & ... \\ 
a_{n1} & ... & a_{nn}
\end{vmatrix} != 0
\),则方程组有唯一解,解为\(x_i = \frac{D_i}{D}\),其中\(D_i\)是常数项\(b_1, ..., b_n\)替换D中第i列元素得到的行列式;若D==0,则无解或无穷多解;

对n个方程n个未知数的齐次线性方程组
\[\begin{cases}
a_{11}x_{1} +...+ a_{1n}x_{n} = 0 \\ 
... \\ 
a_{n1}x_{1} +...+ a_{nn}x_{n} = 0
\end{cases}\]
若D!=0,则只有零解;若D==0,则有非零解;


