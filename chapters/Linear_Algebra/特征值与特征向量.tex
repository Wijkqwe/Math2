
\chapter{特征值与特征向量}

\section{概述}

\subsection{定义}
设n阶矩阵A,\(\lambda\)是一个数,若存在n维非零列向量\(\xi\),使得\(A\xi = \lambda\xi\),则称\(\lambda\)是A的\textbf{特征值},\(\xi\)是A的对应特征值\(\lambda\)的\textbf{特征向量};

\subsection{求法}
\((\lambda E - A)\xi = 0\),因\(\xi \neq 0\),故齐次方程组\((\lambda E - A)x = 0\)有非零解\footnote[1]{\mymatrix},故特征方程\[|\lambda E - A| = \begin{vmatrix}
\lambda - a_{11} & -a_{12} & ... & -a_{1n} \\ 
-a_{21} & \lambda - a_{22} & ... & -a_{2n} \\
... \\ 
-a_{n1} & ... & ... & \lambda - a_{nn}
\end{vmatrix} = 0\]是未知量\(\lambda\)的n次方程,有n个根,\(\lambda E - A\)为\textbf{特征矩阵},\(|\lambda E - A|\)为\textbf{特性多项式};

先根据特征方程求出特征值,再解齐次线性方程组的特征向量;

\subsubsection{技巧}
\begin{enumerate}
    \item 对于特征根单根,若A是n阶实对称矩阵,则必可相似对角化,故\(r(\lambda E - A) = n - 1\),则必有一个方程是多余方程,若\(\lambda E - A\)中n行不成比例,则任一方程均是另外n - 1个方程的线性组合,故可去除任一方程,剩下的仍是同解方程组;
    \item 设k次多项式方程\(f(\lambda) = a_k\lambda^k + ... + a_1\lambda + a_0 = 0\)\begin{itemize}
        \item 若\(a_0 = 0\)则\(f(0) = 0\),则0是\(f(\lambda) = 0\)的根;
        \item 若\(a_k + ... + a_1 + a_0 = 0\),则\(f(1) = 0\);
        \item 若\(f(\lambda)\)偶次项系数包括\(a_0\)之和等于奇次项系数之和,则\(f(-1) = 0\)
    \end{itemize}
    \item 多项式带余除法
    \item 设\(f(x) = 1 * x^k + a_{k - 1}x^{k - 1} + ... + a_1x + a_0\)是\(a_i\)均为整数的多项式,则\(f(x) = 0\)的有理根均为整数,且均是\(a_0\)的因子;
\end{enumerate}


\section{特征值性质}
\begin{itemize}
    \item \(\lambda_0\)是A的特征值\(\Leftrightarrow |\lambda_0E - A| = 0\)
    \item \(\lambda_0\)不是A的特征值\(\Leftrightarrow |\lambda_0E - A| \neq 0 \Leftrightarrow\)矩阵可逆,满秩
    \item 若\(|aA + bE| = 0, a \neq 0\),则\(-\dfrac{b}{a}\)是A的特征值
    \item 若0是A的特征值,则\(|A| = 0 \Leftrightarrow A\text{不可逆}\)
    \item 若\(\lambda_1, ..., \lambda_n\)是A的n个特征值,则\(\begin{cases}
        |A| = \lambda_1\lambda_2...\lambda_n \\ 
        tr(A) = \lambda_1 + ... + \lambda_n
    \end{cases}\)
\end{itemize}

以3阶矩阵为例,设\(A = \begin{bmatrix}
a_{11} & a_{12} & a_{13} \\ 
a_{21} & a_{22} & a_{23} \\ 
a_{31} & a_{32} & a_{33}
\end{bmatrix}\),定义3阶矩阵的k阶主子式为\(\begin{vmatrix}
    a_{i_1i_1} & ... & a_{i_1i_k} \\ 
    ... & & ... \\ 
    a_{i_ki_1} & ... & a_{i_ki_k}
\end{vmatrix}\),则\(|\lambda E - A| = \begin{vmatrix}
    \lambda - a_{11} & ... & ... \\ 
    ... & \lambda - a_{22} & ... \\ 
    ... & ... & \lambda - a_{33}
\end{vmatrix}\)是\(\lambda\)的一元三次多项式,且1阶主子式为\(a_{11}, a_{22}, a_{33}\);2阶主子式为\(A_{11} = \begin{vmatrix}
    a_22 & a_23 \\ 
    a_32 & a_33
\end{vmatrix}, A_{22} = \begin{vmatrix}
    a_{11} & a_{13} \\ 
    a_{31} & a_{33}
\end{vmatrix}, A_{33} = \begin{vmatrix}
    a_{11} & a_{12} \\ 
    a_{21} & a_{22}
\end{vmatrix}\),3阶主子式为\(|A|\),则
\[|\lambda E - A| = 
\begin{vmatrix}
    \lambda & 0 & 0 \\ 
    0 & \lambda & 0 \\ 
    0 & 0 & \lambda
\end{vmatrix} + \begin{vmatrix}
    \lambda & 0 & -a_{13} \\ 
    0 & \lambda & -a_{23} \\ 
    0 & 0 & -a_{33}
\end{vmatrix} + \begin{vmatrix}
    \lambda & -a_{12} & 0 \\ 
    0 & -a_{22} & 0 \\ 
    0 & -a_{32} & \lambda
\end{vmatrix} + \begin{vmatrix}
    \lambda & -a_{12} & -a_{13} \\ 
    0 & -a_{22} & -a_{23} \\ 
    0 & -a_{32} & -a_{33}
\end{vmatrix}\]
\[ + \begin{vmatrix}
    -a_{11} & 0 & 0 \\ 
    -a_{21} & \lambda & 0 \\ 
    -a_{31} & 0 & \lambda
\end{vmatrix} + \begin{vmatrix}
    -a_{11} & 0 & -a_{13} \\ 
    -a_{21} & \lambda & -a_{23} \\ 
    -a_{31} & 0 & -a_{33}
\end{vmatrix} + \begin{vmatrix}
    -a_{11} & -a_{12} & 0 \\ 
    -a_{21} & -a_{22} & 0 \\ 
    -a_{31} & -a_{32} & \lambda
\end{vmatrix} + \begin{vmatrix}
    -a_{11} & -a_{12} & -a_{13} \\ 
    -a_{21} & -a_{22} & -a_{23} \\ 
    -a_{31} & -a_{32} & -a_{33}
\end{vmatrix}\]
\[ = \lambda^3 - (a_{11} + a_{22} + a_{33})\lambda^2 + (A_{11} + A_{22} + A_{33})\lambda - |A|\]

设\(|\lambda E - A| = (\lambda - \lambda_1)(\lambda - \lambda_2)(\lambda - \lambda_3) = \lambda^3 - (\lambda_1 + \lambda_2 + \lambda_3)\lambda^2 + (\lambda_1\lambda_2 + \lambda_1\lambda_3 + \lambda_2\lambda_3)\lambda - \lambda_1\lambda_2\lambda_3\)

比较得\(\begin{cases}
    a_{11} + a_{22} + a_{33} = \lambda_1 + \lambda_2 + \lambda_3 \\ 
    A_{11} + A_{22} + A_{33} = \lambda_2\lambda_3 + \lambda_1\lambda_3 + \lambda_1\lambda_2 \\ 
    |A| = \lambda_1\lambda_2\lambda_3
\end{cases}\)

设\(f(x)\)为多项式,若矩阵A满足\(f(A) = O\),\(\lambda\)是A的任一特征值,则\(\lambda\)满足\(f(\lambda) = 0\)

\section{特征向量性质}
\begin{itemize}
    \item \(\xi\)是A的属于\(\lambda_0\)的特征向量\(\Leftrightarrow \xi\)是\((\lambda_0E - A)x = 0\)的非零解\(\Leftrightarrow k\xi\)仍是A的属于\(\lambda_0\)的特征向量
    \item k重特征根\(\lambda\)至多只有k个线性无关的特征向量;
    \item 若\(\xi_1, \xi_2\)是A的属于不同特征值\(\lambda_1, \lambda_2\)的特征向量,则\(\xi_1, \xi_2\)\textbf{线性无关};
    \item 若\(\xi_1, \xi_2\)是A的属于同一特征值\(\lambda\)的特征向量,则非零向量\(k_1\xi_1 + k_2\xi_2\)\textbf{仍是}A的属于特征值\(\lambda\)的特征向量
    \item 若\(\xi_1, \xi_2\)是A的属于不同特征值\(\lambda_1, \lambda_2\)的特征向量,则当\(k_1 \neq 0, k_2 \neq 0\)时,\(k_1\xi_1 + k_2\xi_2\)\textbf{不是}A的任何特征值的特征向量;
    \item 若\(\lambda_1, \lambda_2\)是A的两个不同的特征值,\(\xi\)是对应于\(\lambda_1\)的特征向量,则\(\xi\)\textbf{不是}对应于\(\lambda_2\)的特征向量;
\end{itemize}


\section{常用矩阵的特征值与特征向量}

\begin{center}
\begin{tabular}{ c c c }
\hline
\text{矩阵} & \text{特征值} & \text{特征向量} \\ 
\hline
\(A\) & \(\lambda\) & \(\xi\) \\ 
\(kA\) & \(k\lambda\) & \(\xi\) \\ 
\(A^k\) & \(\lambda^k\) & \(\xi\) \\ 
\(f(A)\) & \(f(\lambda)\) & \(\xi\) \\ 
\(A^{-1}\) & \(\dfrac{1}{\lambda}\) & \(\xi\) \\ 
\(A^*\) & \(\dfrac{|A|}{\lambda}\) & \(\xi\) \\ 
\(P^{-1}AP\) & \(\lambda\) & \(P^{-1}\xi\) \\ 
\hline
\end{tabular}
\end{center}

