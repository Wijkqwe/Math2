
\chapter{矩阵}

\section{矩阵的秩}

\subsection{定义}
设A为m*n阶矩阵,最高阶非零子式的阶数为A的秩,记作r(A);

存在k阶子式不为零,任意k+1阶子式全为零,则r(A)=k;

组成该矩阵的线性无关的向量的个数;
\mymatrix


\subsection{重要式子}
设Am*n矩阵,B是满足相关运算矩阵,则
\[0 <= r(A) <= min\{m, n\}\]
\[r(kA) = r(A)(k \neq 0)\]
\[r(A + B) <= r(A) + r(B)\]
\[r(AB) <= min\{r(A), r(B)\}\]
\[\text{若}A_{m*n}B_{n*s} = O\text{则}r(A) + r(B) <= n\]
\[\text{若}r(A_{m*n}) = n, \text{则}r(A_{m*n}B_{n*s}) = r(B_{n*s})\]
\[\text{若B可逆,则}r(AB) = r(A)\]
\[\text{P,Q分别m阶n阶可逆矩阵,则}r(A) = r(PA) = r(AQ) = r(PAQ)\]
\[r(A) = r(A^T) = r(A^TA) = r(AA^T)\]
\[\text{对}A_{m * n}, B_{m * s}, r([A, B]) = r(\begin{bmatrix}
A^T \\ 
B^T
\end{bmatrix})\]

A为n阶方阵(n >= 2),则\[r(A^*) = 
\begin{cases}
n,\, r(A) = n \\ 
1,\, r(A) = n - 1 \\ 
0,\, r(A) < n - 1
\end{cases}
\]


\subsection{秩一矩阵}
\begin{flalign}
    \text{r(A}_{m * n}\text{) = 1} & \Leftrightarrow A = \alpha\beta^T\text{,其中}\alpha, \beta\text{分别为m,n维非零列向量。} \nonumber \\ 
    & \Leftrightarrow \text{A的所有行(列)成比例} \nonumber
\end{flalign}

\(A_{n * n} = \alpha\beta^T,\ (\alpha, \beta \neq 0)\):\begin{itemize}
    \item A的特征值为:\(\beta^T\alpha, 0, ..., 0\)
    \item tr(A) = \(\beta^T\alpha\)
    \item tr(A) \(\neq 0 \Rightarrow\)A可相似对角化
    \item \(A^2 = tr(A)A\),\(A^n = (tr(A))^{n - 1}A\)
    \item \(\alpha\)是A的特征值tr(A)的特征向量,\(A\alpha = \alpha\beta^T\alpha = \alpha tr(A)\)
\end{itemize}



\section{转置矩阵}

m*n矩阵A行列交换得到n*m矩阵\(A^T\)

\((A + B)^T = A^T + B^T\),\((AB)^T = B^TA^T\)


\section{方阵行列式}

\[|kA| = k^n|A|\]
\[|A + B| = |A| + |B|\]
\[A \neq O \nRightarrow |A| \neq 0\]
\[A != B \nRightarrow |A| != |B|\]
\[|A^T| = |A|\]
\[\text{AB同阶则}|AB| = |A||B|\]
\[|A^{-1}| = \dfrac{1}{|A|}\]
\mymatrix

\section{重要矩阵}

\paragraph{零矩阵}
每个元素为0,记作\(O\);


\paragraph{单位矩阵}
主对角线元素为1,其余元素为0,记作\(E\);


\paragraph{数量矩阵}
数k与单位矩阵的乘积;


\paragraph{对角矩阵}
主对角线以外元素为0;


\paragraph{行矩阵}
行向量


\paragraph{列矩阵}
列向量


\section{分块矩阵}

用横线纵线将一个矩阵分成若干小块,每个小块称为子块,把子块看作原矩阵元素,得到分块矩阵。

\subsection{运算}

\[
\begin{bmatrix}
A_1 & A_2 \\ 
A_3 & A_4
\end{bmatrix} + 
\begin{bmatrix}
B_1 & B_2 \\ 
B_3 & B_4
\end{bmatrix} = 
\begin{bmatrix}
A_1 + B_1 & A_2 + B_2 \\ 
A_3 + B_3 & A_4 + B_4
\end{bmatrix}
\]
\[k
\begin{bmatrix}
A_1 & A_2 \\ 
A_3 & A_4
\end{bmatrix} = 
\begin{bmatrix}
kA_1 & kA_2 \\ 
kA_3 & kA_4
\end{bmatrix}
\]
\[
\begin{bmatrix}
A & B \\ 
C & D
\end{bmatrix}
\begin{bmatrix}
X & Y \\ 
Z & W
\end{bmatrix} = 
\begin{bmatrix}
AX + BZ & AY + BW \\ 
CX + DZ & CY + DW
\end{bmatrix}
\]
\[
\begin{bmatrix}
A & O \\ 
O & B
\end{bmatrix}^n = 
\begin{bmatrix}
A^n & O \\ 
O & B^n
\end{bmatrix}
\]
\[
\begin{bmatrix}
A & O \\ 
O & B
\end{bmatrix}^{-1} = 
\begin{bmatrix}
A^{-1} & O \\ 
O & B^{-1}
\end{bmatrix}
\]
\[
\begin{bmatrix}
O & A \\ 
B & O
\end{bmatrix}^{-1} = 
\begin{bmatrix}
O & B^{-1} \\ 
A^{-1} & O
\end{bmatrix}
\]


\section{矩阵的逆}

\subsection{定义}
A,B为n阶方阵,若AB=BA=E,则A是可逆矩阵,B是A的逆矩阵且唯一,记作\(A^{-1}\);

A可逆的充要条件是\(|A|\neq 0\)
\mymatrix


\subsection{性质}
\[(A^{-1})^{-1} = A\]
\[\text{若k!=0,则}(kA)^{-1} = \frac{1}{k}A^{-1}\]
\[AB\text{可逆},(AB)^{-1} = B^{-1}A^{-1}\]
\[A^T\text{可逆},(A^T)^{-1} = (A^{-1})^T\]
\[|A^{-1}| = |A|^{-1} = \frac{1}{|A|}\]
\[\text{若A=BC且B,C可逆,则A可逆}A^{-1} = (BC)^{-1} = C^{-1}B^{-1}\]
\[\lambda_{A^{-1}} = \dfrac{1}{\lambda_A}\]
\[\xi_{A^{-1}} = \xi_A\]


\section{伴随矩阵}

将行列式|A|的\(n^2\)个元素的代数余子式按如下形式排列的矩阵称为A的伴随矩阵,记作\(A^*\)即
\[
\begin{bmatrix}
A_{11} & A_{21} & ... & A_{n1} \\ 
A_{12} & A_{22} & ... & A_{n2} \\ 
... & ... & ... & ... \\ 
A_{1n} & A_{2n} & ... & A_{nn}
\end{bmatrix}
\]


\subsection{性质}

\[(A^T)^* = (A^*)^T\]
\[(A^{-1})^* = (A^*)^{-1}\]
\[(AB)^* = B^*A^*\]
\[(A^*)^* = |A|^{n - 2}A\]
\[\lambda_{A^*} = \dfrac{|A|}{\lambda_A}\]
\[\xi_{A^*} = \xi_A\]

对任意n阶方阵A都有伴随矩阵\(A^*\),且有
\[AA^* = A^*A = |A|E\]
\[|A^*| = |A|^{n-1}\]

当\(|A|!= 0\)时,有
\[A^* = |A|A^{-1}\]
\[A^{-1} = \frac{1}{|A|}A^*\]
\[A = |A|(A^*)^{-1}\]
\[(kA)(kA)^* = |kA|E\]
\[A^T(A^T)^* = |A^T|E\]
\[A^{-1}(A^{-1})^* = |A^{-1}|E\]
\[A^*(A^*)^* = |A^*|E\]



\subsection{伴随矩阵求逆矩阵}
\[A^{-1} = \frac{1}{|A|}A^*\]


\section{初等变换}

非零常数乘矩阵的某一行(列);

互换矩阵中的某两行(列);

将矩阵的某一行(列)的k倍加到另一行(列);


\section{初等矩阵}
由单位矩阵经过一次初等变换得到的矩阵称为初等矩阵;

\(E_i(k)\)表示单位矩阵E的第i行(或第i列)乘常数k得到的初等矩阵;

\(E_{ij}\)表示E交换第i行和第j行(或第i列和第j列)得到的初等矩阵;

\(E_{ij}(k)\)表示E的第j行(或第i列)的k倍加到第i行(或第j列)得到的初等矩阵;

\subsection{性质}

初等矩阵的转置仍是初等矩阵
\[E_{ij}^T = E_{ij}\]
\[E_i^T(k) = E_i(k)\]
\[E_{ij}^T(k) = E_{ji}(k)\]

初等矩阵都可逆
\[(E_i(k))^{-1} = E_i(\frac{1}{k})\]
\[E_{ij}^{-1} = E_{ij}\]
\[(E_{ij}(k))^{-1} = E_{ij}(-k)\]

若A可逆,则A可表示成有限个初等矩阵的乘积;

对n阶矩阵A进行初等行变换,等价于在A左边乘相应的初等矩阵;对A进行初等列变换,等价于在A右边乘相应的初等矩阵;


\subsection{初等变换求逆矩阵}
\[
\begin{bmatrix}
A:E
\end{bmatrix}
\xrightarrow{\text{初等行变换}}
\begin{bmatrix}
E:A^{-1}
\end{bmatrix}
\]
\[
\begin{bmatrix}
A \\ 
E
\end{bmatrix}
\xrightarrow{\text{初等列变换}}
\begin{bmatrix}
E \\ 
A^{-1}
\end{bmatrix}
\]

\subsection{初等变换求矩阵方程}
\(Ax = B, x = A^{-1}B\)
\[
\begin{bmatrix}
A:B
\end{bmatrix}
\xrightarrow{\text{初等行变换}}
\begin{bmatrix}
E:A^{-1}B
\end{bmatrix}
\]
\[
\begin{bmatrix}
A \\ 
B
\end{bmatrix}
\xrightarrow{\text{初等列变换}}
\begin{bmatrix}
E \\ 
A^{-1}B
\end{bmatrix}
\]



\section{矩阵方程}


\section{等价矩阵}

设A,B是m*n阶矩阵,若存在可逆矩阵P,Q使得PAQ=B,则称A,B是等价矩阵,记作\(A \cong B\);

设A是m*n矩阵,A等价于形如\(
\begin{bmatrix}
E_r & O \\ 
O & O
\end{bmatrix}
\)的矩阵,r=r(A),称为A的等价标准形;等价标准形是唯一的,即若r(A)=r,则存在可逆矩阵P,Q使得\(PAQ = 
\begin{bmatrix}
E_r & O \\ 
O & O
\end{bmatrix}\)

\[A \text{等价于} B \Leftrightarrow r(A) = r(B)\]
\mymatrix


\section{相似矩阵}

设A,B为n阶矩阵,若
\[P^{-1}AP = B\]
则A,B为相似矩阵,对A进行相似变换得到B,记作\(A \sim B\);

\subsection{性质}
若\(A \sim B\),则
\begin{itemize}
    \item A,B特征多项式相同,特征值相同
    \item \(|A| = |B|\)
    \item \(r(A) = r(B)\)
    \item \(tr(A) = tr(B)\)
    \item \(\lambda_A = \lambda_B\)
    \item \(|\lambda E - A| = |\lambda E - B|\)
    \item \(r(\lambda E - A) = r(\lambda E - B)\)
    \item A,B各阶主子式之和分别相等
\end{itemize}
若上述条件至少一条不成立,则A不相似于B

\subsection{重要结论}
若\(A \sim B\),则
\begin{itemize}
    \item \(A^k \sim B^k, f(A) \sim f(B)\)
    \item 若A可逆,则\(A^{-1} \sim B^{-1}, f(A^{-1}) \sim f(B^{-1})\)
    \item \(A^* \sim B^*\)
    \item \(A^T \sim B^T\)
    \item \(A + kE \sim B + kE\)
\end{itemize}

若\(A \sim C, B \sim D\),则\(\begin{bmatrix}
    A & O \\ 
    O & B
\end{bmatrix} \sim \begin{bmatrix}
    C & O \\ 
    O & D
\end{bmatrix}\)

若\(P^{-1}AP = \Lambda\),其中对角矩阵\(\Lambda = \begin{bmatrix}
\lambda_1 & & & & \\ 
& \lambda_2 & \\ 
& & ... \\ 
& & & \lambda_n
\end{bmatrix}\),则\(\lambda_1, ..., \lambda_n\)为A的特征值

\paragraph{证明}
若\(A \sim B\),有可逆矩阵P,使得\(P^{-1}AP = B\)
\begin{enumerate}
    \item 由\(B^k = B...B = (P^{-1}AP)...(P^{-1}AP) = P^{-1}A^kP\),知\(A^k \sim B^k\)
    \item 两边取逆,有\(P^{-1}A^{-1}P = B^{-1}, A^{-1} \sim B^{-1}\)
    \item 两边取伴随,有\(P^*A^*(P^{-1})^* = B^*\),取\(P^*A^*(P^*)^{-1} = B^*, A^* \sim B^*\)。由于\(P^* = |P|P^{-1}\),\(|P|P^{-1}A^*\dfrac{1}{|P|}P = B^*, P^{-1}A^*P = B^*\)
    \item 两边取转置,有\(P^TA^T(P^{-1})^T = B^T, P^TA^T(P^T)^{-1} = B^T, A^T \sim B^T\),故\(A^T,\ B^T\)相似手段与上述三种不同,如\(A^2 + A^T, B^2 + B^T\)不相似
\end{enumerate}


\subsection{判别与证明}
\begin{itemize}
    \item 定义法,若存在可逆矩阵P,使得\(P^{-1}AP = B\),则\(A \sim B\)
    \item 传递性,若\(A \sim \Lambda, \Lambda \sim B\),则\(A \sim B\)
    \item 性质,若\(A \sim B\),则\(r(A) = r(B), |A| = |B|, tr(A) = tr(B), \lambda_A = \lambda_B, r(\lambda E - A) = r(\lambda E - B)\),A,B的各阶主子式之和分别相等
\end{itemize}


\section{矩阵对角化}
对n阶矩阵A,求相似变换矩阵P,使得\(P^{-1}AP = \Lambda\),\(\Lambda\)是对角矩阵,则A可\textbf{相似对角化},记\(A \sim \Lambda\),称\(\Lambda\)是\(A\)的\textbf{相似标准形};

\subsection{可相似对角化的条件}
\begin{itemize}
    \item 可对角化的充要条件是n阶矩阵A有n个线性无关的特征向量;
    \item 若n阶矩阵A的n个特征值互不相等,则A可相似对角化;
    \item n阶矩阵A可相似对角化\(\Leftrightarrow\)A对每个\(k_i\)重特征值都有\(k_i\)个线性无关的特征向量(\(r(\lambda_{k_i}E - A) = n - k_i\))
    \item 若n阶矩阵A为实对称矩阵,则可相似对角化
\end{itemize}

\subsection{判断步骤}
\begin{enumerate}
    \item 是否实对称矩阵
    \item 特征值是否都是单根,是则相似
    \item 特征值是k重根,若对应k个线性无关的特征向量(\(r(\lambda_kE - A) = n - k\)),则相似
\end{enumerate}


\subsection{对称矩阵对角化}
\paragraph{性质}
\begin{itemize}
    \item 对称矩阵特征值为实数;
    \item 设\(\lambda_1, \lambda_2\)是两个特征值,\(p_1, p_2\)是对应特征向量,若\(\lambda_1 \neq \lambda_2\),则\(p_1,p_2\)正交;
    \item 设A为n阶对称矩阵,则必存在矩阵P,使得\(PAP^{-1} = P^{-1}AP = \Lambda\);
    \item 设A为n阶对称矩阵,\(\lambda\)是A的特征方程的k重根,则\(r(A - \lambda E) = n - k\),\(\lambda\)有k个线性无关的特征向量;
\end{itemize}


\paragraph{步骤}
即求可逆矩阵P:
\begin{enumerate}
    \item 求出全部\(k_i\)重特征值\(\lambda_i,\ \ \displaystyle\sum_{i = 1}^s k_i = n\);
    \item 对每个\(\lambda_i\),求出\((A - \lambda_iE)x = 0\)的基础解系,得到\(k_i\)个线性无关特征向量,并正交化单位化,得到\(k_i\)个两两正交的单位特征向量,即n个两两正交的单位特征向量;
    \item 将n个两两正交的单位特征向量构造为正交阵P;
\end{enumerate}

\subsection{有特征值、特征向量反求A}
若可逆矩阵P,使得\(P^{-1}AP = \Lambda\)则\(A = P\Lambda P^{-1}\)

\subsection{求\(A^k\)及\(f(A)\)}
当\(A \sim \Lambda\),有\(A^k = P\Lambda^kP^{-1}, f(A) = Pf(\Lambda)P^{-1}\)


\subsection{实对称矩阵的相似对角化}

\paragraph{性质}
设A是n阶实对称矩阵
\begin{itemize}
    \item A特征值是实数,特征向量是实向量
    \item 属于不同特征值的特征向量相互正交
    \item 存在n阶正交矩阵Q使得\(Q^TAQ = Q^{-1}AQ = \begin{bmatrix}
        \lambda_1 & & \\ 
        & ... & \\ 
        & & \lambda_n
    \end{bmatrix}\)
    \item 若B是n阶实对称矩阵,则\(A \sim B \Leftrightarrow \lambda_A = \lambda_B\)
\end{itemize}

\paragraph{基本步骤}
求正交矩阵Q,Q不唯一
\begin{enumerate}
    \item 求A的特征值\(\lambda_1, ..., \lambda_n\)
    \item 求特征值对应的特征向量\(\xi_1, ..., \xi_n\)
    \item 将特征向量正交化,单位化为\(\eta_1, ..., \eta_n\)
    \item 令\(Q = [\eta_1, ..., \eta_n]\),则Q为正交矩阵,且\(Q^{-1}AQ = Q^TAQ = \Lambda\)
\end{enumerate}



\subsection{例}

\paragraph{对角化判断}
设A为3阶矩阵,已知\(|E + A| = 0, (3E - A)x = 0\text{有非零解}, E - 3A\)不可逆,问A是否相似于对角矩阵,说明理由
\subparagraph{解}
由题可知:\(|E + A| = 0, |3E - A| = 0, |E - 3A| = 0\),得A得三个特征值:\(-1, 3, \dfrac{1}{3}\),故有三个线性无关特征向量,故A相似于对角矩阵

\paragraph{特征向量定义,对角化判断}
设A为2阶矩阵且\(A^2 - A = 2E, P = [\alpha, A\alpha]\),其中\(\alpha\)是非零向量且不是A得特征向量;(1)证明\(|P| \neq 0\);(2)求\(P^{-1}AP\),判断A是否相似于对角矩阵

\subparagraph{证明1}
若\(|P| = 0\),即P为不可逆矩阵,则\(\alpha, A\alpha\)线性相关,因为\(\alpha \neq 0\),所以\(\exists \lambda_0\),使得\(A\alpha = \lambda_0\alpha\),这与\(\alpha\)不是A得特征向量矛盾,故P可逆,\(|P| = 0\)

\subparagraph{解2}
\(\because A^2 - A - 2E = 0\),即\(A^2\alpha - A\alpha - 2\alpha = 0, A^2\alpha = 2\alpha + A\alpha, \)\[\therefore AP = [A\alpha, A^2\alpha] = [A\alpha, 2\alpha + A\alpha] = [\alpha, A\alpha]\begin{bmatrix}
    0 & 2 \\ 
    1 & 1
\end{bmatrix} = P\begin{bmatrix}
    0 & 2 \\ 
    1 & 1
\end{bmatrix}\]
\[\therefore P^{-1}AP = \begin{bmatrix}
    0 & 2 \\ 
    1 & 1
\end{bmatrix}\]
\[\therefore |\lambda E - A| = \begin{vmatrix}
    \lambda & -2 \\ 
    -1 & \lambda - 1
\end{vmatrix}\]
得A得特征值为\(2, -1\),故A相似于对角矩阵\(\begin{bmatrix}
    2 & 0 \\ 
    0 & -1
\end{bmatrix}\)

\paragraph{特征值特征向量求矩阵}
设A是3阶矩阵,已知\(A\xi_i = i\xi_i, (i = 1, 2, 3)\),其中\(\xi_1 = [1, 0, 0]^T, \xi_2 = [1, 1, 0]^T, \xi_3 = [1, 1, 1]^T\),则矩阵A = ?
\subparagraph{解法1}
由题可知,A有3个互不相同的特征值,故A相似于对角矩阵,且\(\xi_1, \xi_2, \xi_3\)是3个线性无关的特征向量,故存在可逆矩阵\(P = [\xi_1, \xi_2, \xi_3]\)使得\(P^{-1}AP = \begin{bmatrix}
    1 & & \\ 
    & 2 & \\ 
    & & 3
\end{bmatrix}\),故\(A = P\begin{bmatrix}
    1 & & \\ 
    & 2 & \\ 
    & & 3
\end{bmatrix}P^{-1}\)

\subparagraph{解法2}
由题得\[[A\xi_1, A\xi_2, A\xi_3] = A[\xi_1, \xi_2, \xi_3] = [\xi_1, 2\xi_2, 3\xi_3]\]
\[\therefore\, A = [\xi_1, 2\xi_2, 3\xi_3][\xi_1, \xi_2, \xi_3]^{-1}\]

\paragraph{求特征向量}
已知\(P^{-1}AP = \begin{bmatrix}
    1 \\ 
     & 1 \\ 
     & & -1
\end{bmatrix}, P = (\alpha_1, \alpha_2, \alpha_3)\)可逆,则矩阵A关于特征值\(\lambda = 1\)的特征向量是?
\subparagraph{解}
当\(P^{-1}AP = \Lambda\)时,P的每一列都是A的相应的特征向量,故\(\lambda = 1\)的特征向量为\(k_1\alpha + k_2\alpha,\ \ k_1, k_2\)不全为0。

\paragraph{相似对角求矩阵}
已知矩阵\(A = \begin{bmatrix}
    3 & 1 & 2 \\ 
    0 & 2 & a \\ 
    0 & 0 & 3
\end{bmatrix}\)和对角矩阵相似,则\(a=\)?
\subparagraph{解}
\begin{flalign}
    A \sim \Lambda & \Leftrightarrow \lambda = 3\text{有两个线性无关特征向量} \nonumber \\ 
    & \Leftrightarrow (3E - A)x = 0\text{有两个线性无关解} \nonumber \\ 
    & \Leftrightarrow r(3E - A) = 1 \nonumber
\end{flalign}
故\(a = -2\)




\section{矩阵的迹}
方阵对角线元素的总和,记作\(tr(A)\)

\subsubsection{性质}

\(tr(A) = \)特征值之和








