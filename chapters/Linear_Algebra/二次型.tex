
\chapter{二次型}

\section{概念}

\subsection{定义}
n元变量\(x_1, ..., x_n\)的二次齐次多项式
\[f(x_1,..., x_n) = a_{11}x_1^2 + 2a_{12}x_1x_2 + ... + 2a_{1n}x_1x_n + a_{22}x_2^2 + ... + 2a_{2n}x_2x_n + ... + a_{nn}x_n^2\]
称为n元二次型,简称二次型。

只考虑系数\(a_{ij} \in R\)的情况,又称实二次型。
\begin{flalign}
    f(x_1, ..., x_n) = & a_{11}x_1^2 + a_{12}x_1x_2 + ... + a_{1n}x_1x_n \nonumber \\ 
    & + a_{21}x_2x_1 + a_{22}x_2^2 + ... + a_{2n}x_2x_n + ... \nonumber \\ 
    & + a_{n1}x_nx_1 + ... + a_{nn}x_n^2 \\ 
    = & \sum_{i = 1}^n\sum_{j = 1}^na_{ij}x_ix_j
\end{flalign}
其中(1)为完全展开式,(2)为和式。

\subsection{矩阵表示}
令\[A = \begin{bmatrix}
    a_{11} & ... & a_{1n} \\ 
    a_{21} & ... & a_{2n} \\ 
    ... & ... & ... \\ 
    a_{n1} & ... & a_{nn}
\end{bmatrix}, x = \begin{bmatrix}
    x_1 \\ 
    x_2 \\ 
    ... \\ 
    x_n
\end{bmatrix}\]
则二次型可表示为(矩阵表达式)\[f(x) = x^TAx\]
实对称矩阵A称为二次型f(x)的矩阵;二次型有不同的写法,使用对称矩阵表示的二次型写法唯一。

\section{性质}

\begin{itemize}
    \item 二次型系数矩阵A的秩为二次型\(f(x)\)的秩
\end{itemize}


\section{合同变换}

\subsection{线性变换}

\subsubsection{定义}

对n元二次型\(f(x_1, ..., x_n)\),若令
\[\begin{cases}
    x_1 = c_{11}y_1 + c_{12}y_2 + ... + c_{1n}y_n \\ 
    x_2 = c_{21}y_1 + c_{22}y_2 + ... + c_{2n}y_n \\ 
    ... \\ 
    x_n = c_{n1}y_1 + c_{n2}y_2 + ... + c_{nn}y_n
\end{cases}\tag{*}\]
记\(x = \begin{bmatrix}
    x_1 \\ 
    ... \\ 
    x_n
\end{bmatrix}, C = \begin{bmatrix}
    c_{11} & ... & ... \\ 
    ... & & ... \\ 
    c_{n1} & ... & c_{nn}
\end{bmatrix}, y = \begin{bmatrix}
    y_1 \\ 
    ... \\ 
    y_n
\end{bmatrix}\),则(*)式(线性变换)可写为\[x = Cy\]
(*)称为线性变换;若线性变换的系数矩阵C可逆\footnote{\mymatrix},则称为可逆线性变换,则\[f(x) = (Cy)^TA(Cy) = y^T(C^TAC)y\]
记\(B = C^TAC\),则\[f(x) = y^TBy = g(y)\]
即二次型f(x)通过线性变换得到二次型g(y)


\subsection{矩阵合同}

\subsubsection{定义}
设A,B为n阶矩阵,若存在可逆矩阵C,使得\[C^TAC = B\]
则称A与B合同,记作\(A \simeq B\),对应二次型\(f(x), g(y)\)为合同二次型。

在二次型背景下,A为f(x)的“形态”,B为g(y)的“形态”,而f(x)=g(y),故A,B分别代表同一事物在不同参照系\(x, y\)下的不同“形态”。

综上,二次型中,A与B的合同指同一二次型在可逆线性变换下的两个不同状态的联系。


\subsubsection{性质}
\begin{itemize}
    \item \(A \simeq A\)反身性
    \item 若\(A \simeq B, B \simeq A\)对称性
    \item 若\(A \simeq B, B \simeq C\),则\(A \simeq C\)传递性
    \item \(A \simeq B, r(A) = r(B)\),可逆线性变换不会改变二次型的秩
    \item 与对称矩阵合同的矩阵必是对称矩阵
\end{itemize}


\subsubsection{充要条件}
两个二次型合同的充要条件是有相同的正负惯性指数,或有相同的秩及正(或负)惯性指数,或有相同的正负特征值数。


\subsection{标准形与规范形}

若二次型中只有平方项,没有交叉项(交叉项系数为0)则其为标准形(一般不唯一)

若标准形中,系数取值范围为\{1, -1, 0\},则其为规范形


\subsubsection{定理}
\begin{itemize}
    \item 任何二次型均可通过配方法(作可逆线性变换\(x = Cy\))化成标准形及规范形;用矩阵描述即:任何实对称矩阵A,必存在可逆矩阵C(不唯一),使得\(C^TAC = \Lambda\)其中\[\Lambda = \text{标准形}\begin{bmatrix}
        d_1 \\ 
        & d_2 \\ 
        & & ... \\ 
        & & & d_n
    \end{bmatrix}\ ||\ \Lambda = \text{规范形}\begin{bmatrix}
        1 \\ 
        & ... \\ 
        & & -1 \\ 
        & & & ... \\ 
        & & & & 0
    \end{bmatrix}\]
    此时C的列向量一般不为A的特征向量,\(d_i\)一般不为A特征值;
    \item 任何二次型可通过正交变换\(x = Qy\)化为标准形;用矩阵描述即:任何实对称矩阵A,一定存在正交矩阵Q(不唯一),使得\(Q^{-1}AQ = Q^TAQ = \Lambda\),其中\[\Lambda = \begin{bmatrix}
        \lambda_1 \\ 
        & ... \\ 
        & & \lambda_n
    \end{bmatrix}\]
    此时Q的列向量均为A的特征向量,\(\lambda_i\)均为A的特征值;
\end{itemize}


\section{惯性定理}
无论选取怎样的可逆线性变换,将二次型化成标准形或规范形,其中正项个数p,负项个数q都是不变的,p为正惯性指数,q为负惯性指数;

若二次型的秩为r,则\(r = p + q\)




\section{计算}

\subsection{求可逆线性变换C}

\subsubsection{配方法}
\begin{enumerate}
    \item 对二次型配方
    \item 作线性变换
    \item 将二次型化成标准形,
    \item 将线性变换表示成矩阵形式,即\(x = Cy\)
\end{enumerate}

将某个变量的平方项及其有关的混合项一次配完,配成一个完全平方,减少一个未配完全平方的变量,使得总的平方项的项数小于等于变量个数。目的是保证所用变换是可逆的;

当总的完全平方项的项数小于变量个数时,如三元二次型,完全平方项个数是2,应视作\(f(x_1, x_2, x_3) = (...)^2 + a(...)^2 + 0x_3^2\),变换为\(\begin{cases}
    y_1 = ... \\ 
    y_2 = ... \\ 
    y_3 = x_3
\end{cases}\)

若有平方项,应将平方项及其混合项配成完全平方;若没有平方项,作可逆线性变换\(\begin{cases}
    x_1 = y_1 + y_2 \\ 
    x_2 = y_1 - y_2 \\ 
    x_3 = y_3
\end{cases}\),使其出现完全平方项,然后再配平方;

配方法提供了\begin{itemize}
    \item 所作的可逆线性变换
    \item 与A合同的对角矩阵
    \item 二次型(或A)的秩
    \item 正负惯性指数
    \item 是否正定
\end{itemize}


\subsection{求正交变换Q}

\subsubsection{化为标准形}
求正交矩阵Q,Q不唯一
\begin{itemize}
    \item 求A的特征值\(\lambda_1, ..., \lambda_n\)
    \item 求特征值对应的特征向量\(\xi_1, ..., \xi_n\)
    \item 将特征向量正交化,单位化为\(\eta_1, ..., \eta_n\)
    \item 令\(Q = [\eta_1, ..., \eta_n]\),则Q为正交矩阵,且\(Q^{-1}AQ = Q^TAQ = \Lambda\)
\end{itemize}

\subsubsection{化为二次型}
\begin{itemize}
    \item 分别求出使两个二次型化为对角矩阵的正交矩阵\(Q_1, Q_2\)
    \item 则\(Q_1^TAQ_1 = Q_2^TBQ_2\),故\((Q_1Q_2^T)^TA(Q_1Q_2^T) = B\)
    \item 故\(Q = Q_1Q_2^T\)为所求矩阵
\end{itemize}


\subsubsection{性质}
\begin{itemize}
    \item 只能化二次型为标准形,不能化为规范形(除非特征值\(\in \{0, -1, 1\}\))
    \item 正交变换不唯一,化成的标准形唯一(不考虑特征值顺序),为\(\lambda_1y_1^2 + ... + \lambda_ny_n^2\)
    \item 对多重根,在解方程组时同时考虑正交化
\end{itemize}


\subsection{例}

\subsubsection{正交变换化为标准形}
设二次型f在正交变换\(x = Py\)下的标准形为\(2y_1^2 + y_2^2 - y_3^2\),其中\(P = [e_1, e_2, e_3]\),若\(Q = [e_1, -e_3, e_2]\)则二次型f在\(x = Qy\)下的标准形为?(\(2y_1^2 - y_2^2 + y_3^2\))

\paragraph{解}
\(e_1, e_2, e_3\)分别是A对应于特征值2, 1, -1的特征向量,于是\(-e_3\)对应特征值-1的特征向量,故


\subsubsection{正交变换化为二次型}
设二次型\(f(x_1, x_2) = x_1^2 - 4x_1x_2 + ax_2^2\)经过正交变换\(\begin{bmatrix}
    x_1 \\ 
    x_2
\end{bmatrix} = Q\begin{bmatrix}
    y_1 \\ 
    y_2
\end{bmatrix}\)化为二次型\(g(y_1, y_2) = 4y_1^2 + 4y_1y_2 + by_2^2\)
\begin{enumerate}
    \item 求a,b值
    \item 求正交矩阵Q
\end{enumerate}

\paragraph{解1}
由题可知,二次型f,g的矩阵分别为\(A = \begin{bmatrix}
    1 & -2 \\ 
    -2 & a
\end{bmatrix}, B = \begin{bmatrix}
    4 & 2 \\ 
    2 & b
\end{bmatrix}\),且\(Q^TAQ = B\)由于Q为正交矩阵,于是有\(Q^{-1}AQ = B\),因此\(tr(A) = tr(B), |A| = |B|\),得\(a = 4, b = 1\)

\paragraph{解2}
由\(|\lambda E - A| = |\lambda E - B| = \lambda(\lambda - 5)\)得特征值为\(0, 5\)

矩阵A对应特征值\(\lambda_1 = 0\)得特征向量\(a_1\),对应特征值\(\lambda_2 = 5\)得特征向量\(a_2\)。令\(Q_1 = [a_1, a_2]\),则\(Q_1\)为正交矩阵

矩阵B对应特征值\(\lambda_1 = 0\)得特征向量\(b_1\),对应特征值\(\lambda_2 = 5\)得特征向量\(b_2\)。令\(Q_2 = [b_1, b_2]\),则\(Q_2\)为正交矩阵

由\(Q_1^TAQ_1 = Q_2^TBQ_2, \therefore\ Q = Q_1Q_2^T\)


\subsubsection{惯性指数}
设二次型\(f(x_1, x_2, x_3) = x_1^2 - x_2^2 + 2ax_1x_3 + 4x_2x_3\)的负惯性指数为1,则a的取值范围为?

\paragraph{解}
\(f = x_1^2 + 2ax_1x_3 + a^2x_3^2 - x_2^2 + 4x_2x_3 - 4x_3^2 + 4x_3^2 - a^2x_3^2 = (x_1 + ax_3)^2 - (x_2 - 2x_3)^2 + (4 - a^2)x_3^2\)

\(\because\ 4 - a^2 >= 0, \therefore\ -2 <= a <= 2\)


\section{正定二次型}

\subsection{定义}
n元二次型\(f(...) = x^TAx\),若对任意的\(x = [x_1,..., x_n]^T \neq 0\)均有\(x^TAx > 0\),则称f为正定二次型,对应矩阵A为正定矩阵。


\subsection{充要条件}
\begin{flalign}
    & n\text{元二次型}f = x^TAx\text{正定} \nonumber \\ 
    \Leftrightarrow & \text{对任意}x \neq 0, \text{有}x^TAx > 0 \nonumber \\ 
    \Leftrightarrow & f\text{的正惯性指数}p = n \nonumber \\ 
    \Leftrightarrow & \text{存在可逆矩阵}D, \text{使}A = D^TD \nonumber \\ 
    \Leftrightarrow & A \simeq E \nonumber \\ 
    \Leftrightarrow & A\text{的特征值}\lambda_i > 0 \nonumber \\ 
    \Leftrightarrow & A\text{的全部顺序主子式均大于}0 \nonumber \\ 
    \Leftrightarrow & A^{-1}\text{正定} \nonumber
\end{flalign}


\subsection{顺序主子式}
设\(A = (a_{ij})_{n * n}\),则
\[|A_k| = \begin{vmatrix}
    a_{11} & a_{12} & ... & a_{1k} \\ 
    a_{21} & a_{22} & ... & a_{2k} \\ 
    ... \\ 
    a_{k1} & a_{k2} & ... & a_{kk}
\end{vmatrix}\]
称为n阶矩阵A的k阶顺序主子式(或左上角主子式),


\subsection{必要条件}
\begin{itemize}
    \item \(a_{ii} > 0, (i = 1, ..., n)\)
    \item \(|A| > 0\)
    \item \(A^*\)正定
\end{itemize}

\subsection{例}

\subsubsection{判别正定性}
判别\(f(x_1, x_2, x_3) = 2x_1^2 + 2x_2^2 + 2x_3^2 + 2x_1x_2 + 2x_1x_3 + 2x_2x_3\)的正定性

\paragraph{方法一}
判断各阶顺序主子式\[2 > 0, \begin{vmatrix}
    2 & 1 \\ 
    1 & 2
\end{vmatrix} = 3 > 0, |A| = 4 > 0\]

\paragraph{方法二}
判断特征值是否全部大于0

\paragraph{方法三}
配方法化为标准形,判断正惯性指数p是否等于n

\paragraph{方法四}
定义验证是否对任意\(x = [x_1, x_2, x_3]^T \neq 0\)有\(x^TAx > 0\)\[f = (x_1 + x_2)^2 + (x_1 + x_3)^2 + (x_2 + x_3)^2\]
故有\(f >= 0\),且\[f = 0 \Leftrightarrow \begin{cases}
    x_1 + x_2 = 0 \\ 
    x_1 + x_3 = 0 \\ 
    x_2 + x_3 = 0
\end{cases}\tag{*}\]
方程组(*)的系数行列式\( = 2 \neq 0\),故(*)只有零解,故\(x = [x_1, x_2, x_3]^T \neq 0\)时\(f > 0\)

\subparagraph{注}
对\(f = (...)^2 + (...)^2 + ...\)可直接计算系数行列式


\subsubsection{判断二次型矩阵}
设矩阵\[B = \begin{bmatrix}
    1 & 2 & 3 \\ 
    -2 & 0 & 1 \\ 
    0 & 4 & 5
\end{bmatrix}, x = \begin{bmatrix}
    x_1 \\ 
    x_2 \\ 
    x_3
\end{bmatrix}\]问\(f = x^TBx\)是否为关于\(x_1, x_2, x_3\)的二次型?B是否为f的矩阵?写出\(f\)的矩阵表达式

\(f\)是关于\(x_1, x_2, x_3\)的二次型,B不是f的矩阵
\paragraph{方法一}
由于
\begin{flalign}
    f & = \begin{bmatrix}
        x_1, & x_2, & x_3
    \end{bmatrix}B\begin{bmatrix}
        x_1 \\ 
        x_2 \\ 
        x_3
    \end{bmatrix} \nonumber \\ 
    & = \begin{bmatrix}
        x_1, & x_2, & x_3
    \end{bmatrix}\begin{bmatrix}
        x_1 + 2x_2 + 3x_3 \\ 
        -2x_1 + x_3 \\ 
        4x_2 + 5x_3
    \end{bmatrix} \nonumber \\ 
    & = x_1^2 + 5x_3^2 + 3x_1x_3 + 5x_2x_3 \nonumber
\end{flalign}
故矩阵为\[A = \begin{bmatrix}
    1 & 0 & \dfrac{3}{2} \\ 
    0 & 0 & \dfrac{5}{2} \\ 
    \dfrac{3}{2} & \dfrac{5}{2} & 5
\end{bmatrix}\]

\paragraph{方法二}
注意到\(x^TBx\)是\(1 * 1\)矩阵,故转置不变,故有
\begin{flalign}
    f & = x^TBx = (x^TBx)^T = \dfrac{1}{2}[x^TBx + (x^TBx)^T] \nonumber \\ 
    & = \dfrac{1}{2}(x^TBx + x^TB^Tx) = \dfrac{1}{2}x^T(B + B^T)x = x^T\dfrac{B + B^T}{2}x
\end{flalign}
故\[A = \dfrac{B + B^T}{2}\]



