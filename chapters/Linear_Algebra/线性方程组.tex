
\chapter{线性方程组}

方程组的解是描述列向量组中各向量之间数量关系的系数;

\section{线性方程组与向量组}

非齐次线性方程组\[\begin{cases}
    a_{11}x_1 + ... + a_{1n}x_n = b_1 \\ 
    ... \\ 
    a_{m1}x_1 + ... + a_{mn}x_n = b_m
\end{cases}\]
系数矩阵\[A = \begin{bmatrix}
    a_{11} & ... & a_{1n} \\ 
    ... \\ 
    a_{m1} & ... & a_{mn}
\end{bmatrix}\]
其增广矩阵\[\begin{bmatrix}
    a_{11} & ... & a_{1n} & b_1 \\ 
    ... \\ 
    a_{m1} & ... & a_{mn} & b_m
\end{bmatrix}\]
向量组形式
\[\alpha_1x_1 + ... + \alpha_nx_n = \beta\]
其中\[\alpha_i = \begin{bmatrix}
    a_{1i} \\ 
    a_{2i} \\
    ... \\ 
    a_{mi}
\end{bmatrix}, \beta = \begin{bmatrix}
    b_1 \\ 
    ... \\ 
    b_m
\end{bmatrix}\]
方程组中未知数是向量组中成员系数,方程组问题是向量组问题,方程组和向量组是同一问题两种表现形式;

求线性方程组即对增广矩阵作\textbf{初等行变换},化为行阶梯性矩阵后求解;


\section{齐次线性方程组}
\[\begin{cases}
    a_{11}x_1 + ... + a_{1n}x_n = 0 \\ 
    ... \\ 
    a_{m1}x_1 + ... + a_{mn}x_n = 0
\end{cases}\]
向量形式为\[\alpha_1x_1 + ... + \alpha_nx_n = 0, \alpha_i = \begin{bmatrix}
    a_{1i} \\ 
    a_{2i} \\
    ... \\ 
    a_{mi}
\end{bmatrix}\]
矩阵形式为
\[A_{m*n}x = 0, A_{m*n} = \begin{bmatrix}
    a_{11} & ... & a_{1n} \\ 
    ... \\ 
    a_{m1} & ... & a_{mn}
\end{bmatrix}, x = \begin{bmatrix}
    x_1 \\ 
    ... \\ 
    x_n
\end{bmatrix}\]

\subsection{有解条件}
当\(r(A) = n\)(向量组线性无关)\footnote{\mymatrix}时,方程组有唯一零解;

当\(r(A) = r < n\)(向量组线性相关)时,方程组有非零解(无穷多解),且有n-r个线性无关解;


\subsection{解性质}
\begin{itemize}
    \item 若\(A\xi_1 = 0, A\xi_2 = 0\)则\(A(k_1\xi_1 + k_2\xi_2) = 0\)
    \item 若\(A_{n*n}B_{n*n} = 0\),则B中每一个列向量都是Ax=0的解;如\(|A| = 0\),\(A^*A = |A|E = O\),则A的列向量是\(A^*x = 0\)的解;
    \item 若\(Ax = 0\)存在\(a\)个线性无关解,则\(n - r(A) >= a\);
\end{itemize}


\subsection{基础解系和解结构}
\paragraph{基础解系}
:设\(\xi_1...\xi_{n - r}\)满足\begin{enumerate}
    \item 是方程组Ax = 0的解;
    \item 线性无关;
    \item 方程组Ax = 0任一解均可由\(\xi_1...\xi_{n - r}\)线性表示;
\end{enumerate}

\paragraph{通解}
:设\(\xi_1...\xi_{n - r}\)是Ax = 0的基础解系,则\(k_1\xi_1 + ... + k_{n - r}\xi_{n - r}\)是方程组Ax = 0的通解;

\subsection{求解方法步骤}
\begin{enumerate}
    \item 将系数矩阵A作初等行变换化为行阶梯型矩阵B,阶数为r即r(A) = r;
    \item 按列找出秩为r的子矩阵,剩余列未知数设为自由变量;
    \item 对子矩阵赋值,求出基础解系,写出通解;
\end{enumerate}


\subsection{例}

\subsubsection{通解}
设\(A = \begin{bmatrix}
    1 & -2 & 0 \\ 
    2 & 1 & 5 \\ 
    0 & 1 & 1
\end{bmatrix}\),B是三阶矩阵,则满足AB = O的所有B=?
\subparagraph{解}
\(B = \begin{bmatrix}
    2k & 2l & 2\lambda \\ 
    k & l & \lambda \\ 
    -k & -l & -\lambda
\end{bmatrix}\)


\section{非齐次线性方程组}
\[
\begin{cases}
a_{11}x_1 + ... + a_{1n}x_n = b_1 \\ 
... \\ 
a_{m1}x_1 + ... + a_{mn}x_n = b_m
\end{cases}
\]为m个方程,n个未知数的非齐次线性方程组;

\paragraph{向量形式}为\(x_1\alpha_1 + ... + x_n\alpha_n = \beta\),其中\(\alpha_i = \begin{bmatrix}
a_{1i} \\ 
... \\ 
a_{mi}
\end{bmatrix}, \beta = \begin{bmatrix}
b_1 \\ 
... \\ 
b_m
\end{bmatrix}\);

\paragraph{矩阵形式}为\(Ax = \beta\),其中\(A = \begin{bmatrix}
a_{11} & ... & a_{1n} \\ 
... \\ 
a_{m1} & ... & a_{mn}
\end{bmatrix}\),A的伴随矩阵为\(\begin{bmatrix}
a_{11} & ... & a_{1n} & b_1 \\ 
... & & & ... \\ 
a_{m1} & ... & a_{mn} & b_m
\end{bmatrix}\)记作\([A : \beta]\)

\subsection{有解条件}
若\(r(A) \neq r([A, \beta])\)(\(\beta\)不能由\(\alpha_1, ..., \alpha_n\)线性表示),则无解;

若\(r(A) = r([A, \beta]) = n\)(即\(\alpha_1, ..., \alpha_n\)线性无关,\(\alpha_1, ..., \alpha_n, \beta\)线性相关,则有唯一解;

若\(r(A) = r([A, \beta]) = r < n\),则有无穷多解;

若\(r(A) = m\),则\(r([A, \beta] = m = r(A)\),有解;

对于\(A^TAx = A^T\beta\),因为\([A^TA : A^T\beta] = A^T[A : \beta]\),故\(r([A^TA, A^T\beta] <= r(A^T) = r(A)\),又\(r(A) = r(A^TA) <= r([A^TA, A^T\beta]\),故\(r(A^TA) = r([A^TA, A^T\beta]\),因此有解;

对于方阵\(A_{n * n}\),若有多解,则\(|A| = 0/r(A) < n/A\text{不可逆}\)

\subsection{解性质}
设\(\eta_1, \eta_2\),\(\eta\)是非齐次线性方程组\(Ax = \beta\)的解,\(\zeta\)是对应齐次线性方程组\(Ax = 0\)的解,则\begin{itemize}
    \item \(\eta_1 - \eta_2\)是\(Ax = 0\)的解;
    \item \(k\zeta + \eta\)是\(Ax = \beta\)的解;
\end{itemize}

由\(r(A) = r([A, \beta])\)得\(r(A^T) = r(\begin{bmatrix}
A^T \\ 
\beta^T
\end{bmatrix})\)

\subsection{解结构}
通解:\(k_1\xi_1 + ... + k_{n - r}\xi_{n - r} + \eta\)


\subsection{求解方法步骤1}
\begin{enumerate}
    \item 写出\(Ax = \beta\)的导出方程组\(Ax = 0\),求出\(Ax = 0\)的通解\(k_1\xi_1 + k_2\xi_2 + ... + k_{n - r}\xi_{n - r}\);
    \item 求出\(Ax = \beta\)的一个特解\(\eta\);
    \item \(Ax = \beta\)的通解为\(k_1\xi_1 + ... + k_{n - r}\xi_{n - r} + \eta\),其中\(k_i\)为任意常数。
\end{enumerate}


\subsection{求解方法步骤2}
\begin{enumerate}
    \item 化简\(Ax = \beta\)增广矩阵;
    \item 取\(n - r\)个自由未知量,代入增广矩阵求出其余r个\(x_i\);
    \item 通解由自由未知量表示。
\end{enumerate}


\subsection{例}

\subsubsection{例1}
设\(r(A_{4 * 4}) = 2, \eta_1, \eta_2, \eta_3\)是\(Ax = b\)的3个解向量,其中\(\begin{cases}
\eta_1 - \eta_2 = \alpha_1 \\ 
\eta_1 + \eta_2 = \alpha_2 \\ 
\eta_3 + 2\eta_2 = \alpha_3
\end{cases}\),求\(Ax = b\)通解

\subparagraph{解}
\(Ax = \beta\)通解结构为\[k_1\xi_1 + k_2\xi_2 + \eta\],因为\(A(\eta_1 - \eta_2) = b - b = 0, A[3(\eta_1 + \eta_2) - 2(\eta_3 + 2\eta_2)] = 6b - 6b = 0\),故\(\eta_1 - \eta_2, 3(\eta_1 + \eta_2) - 2(\eta_3 + 2\eta_2)\)是\(Ax = 0\)的解向量,又\(A[\dfrac{1}{2}(\eta_1 + \eta_2)] = \dfrac{1}{2}(b + b) = b\),因此\(\dfrac{1}{2}(\eta_1 + \eta_2)\)是\(Ax = b\)的一个特解,因此\(Ax = b\)通解为\[k_1\alpha_1 + k_2(3\alpha_2 - 2\alpha_3) + \dfrac{1}{2}\alpha_2\]

\subsubsection{例2}
由\([\eta_1 - \eta_2, \eta_1 + \eta_2, \eta_3 + 2\eta_2] = [\eta_1, \eta_2, \eta_3]\begin{bmatrix}
1 & 1 & 0 \\ 
-1 & 1 & 2 \\ 
0 & 0 & 1
\end{bmatrix}\),故\[[\eta_1, \eta_2, \eta_3] = [\eta_1 - \eta_2, \eta_1 + \eta_2, \eta_3 + 2\eta_2]\begin{bmatrix}
1 & 1 & 0 \\ 
-1 & 1 & 2 \\ 
0 & 0 & 1
\end{bmatrix}^{-1}\],求出\(\eta_1, \eta_2, \eta_3\),的通解\[k_1(\eta_1 - \eta_2) + k_2(\eta_2 - \eta_3) + \eta_3\]


\subsubsection{解性质}
设\(A_{3 * 3}x = b\),即\(\begin{cases}
    a_{11}x_1 + ... = b_1 \\ 
    ... \\ 
    ... \\ 
\end{cases}\)有唯一解\(\xi = [1, 2, 3]^T\)。

方程组\(B_{3 * 4}x = b\),即\(\begin{cases}
    a_{11}x_1 + ... + a_{14}x_4 = b_1 \\ 
    ... \\ 
    ... \\ 
\end{cases}\)有特解\(\eta = [-2, 1, 4, 2]^T\),则\(B_{3 * 4}x = b\)的通解为?

\subparagraph{解}
\(r(A) = r(A, b) = 3,\ \therefore\ r(B) = r(B, b) = 3, \eta_1 = [1, 2, 3, 0]^T\)是\(B_{3 * 4}x = b\)的另一特解。故Bx = 0基础解系仅一个向量,为\(\eta - \eta_1\),故通解为\(k(\eta - \eta_1) + \eta\)



\section{两个方程组的公共解}

齐次线性方程组\(A_{m * n}x = 0, B_{m * n}x = 0\)的公共解是满足方程组\(\begin{bmatrix}
A \\ 
B
\end{bmatrix}x = 0\)的解;同理可求\(Ax = \alpha, Bx = \beta\)的公共解;
\begin{itemize}
    \item 若给出\(A_{m * n}x = 0\)的基础解系\(\xi_1, ..., \xi_s\)和B的具体表达式,则先写出\(Ax = 0\)的通解\(k_1\xi_1 + ... + k_s\xi_s\)代入\(Bx = 0\),求出\(k_i\)之间的关系,代回\(Ax = 0\)得通解,即得公共解;
    \item 若给出\(A_{m * n}x = 0\)的基础解系\(\xi_1, ..., \xi_s\)和B的基础解系\(\eta_1, ..., \eta_t\),则公共解\(\gamma = k_1\xi_1 + ... + k_s\xi_s = l_1\eta_1 + ... + l_t\eta_t\),即\(k_1\xi_1 + ... + k_s\xi_s - l_1\eta_1 - ... - l_t\eta_t = 0\)解此式求出\(k_i/l_j\)之间关系,代入即可求出\(\gamma\)
\end{itemize}



\section{同解方程组}

若两个方程组\(A_{m * n}x = 0, B_{s * n}x = 0\)有完全相同的解,则称它们为同解方程组;

\begin{flalign}
& Ax = 0, Bx = 0\text{同解方程组} \nonumber \\ 
\Leftrightarrow & Ax = 0\text{解满足}Bx = 0, Bx = 0\text{解满足}Ax = 0 \nonumber \\ 
\Leftrightarrow & r(A) = r(B), \text{且}Ax = 0\text{的解满足}Bx = 0 \nonumber \\ 
\Leftrightarrow & r(A) = r(B) = r(\begin{bmatrix}
A \\ 
B
\end{bmatrix}) \nonumber
\end{flalign}





